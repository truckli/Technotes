%!Mode:: "TeX:UTF-8"
\section{Linux启动过程}


\subsection{BIOS}
BIOS是英文"Basic Input Output System"的缩略语,它是一组固化到计算机内主板上一个ROM芯片上的程序,它保存着计算机最重要的基本输入输出的程序、系统设置信息、开机后自检程序和系统自启动程序。 其主要功能是为计算机提供最底层的、最直接的硬件设置和控制。

\textbf{Bootloader即引导装入程序},是由BIOS用来把操作系统内核镜像装载到RAM中所调用的一个程序。
它在系统上电时开始执行,初始化硬件设备、准备好软件环境,最后调用系统内核。
Linux在x86下的引导装入程序有Linux Loader(LILO)和GRUB,GRUB更高级,本文假定使用LILO。

硬盘的第一个扇区称为MBR(Master Boot Record),该扇区包含一个分区表和一个小程序,这个小程序用来装载被启动的操作系统所在分区的第一个扇区。Windows 98使用分区表中的active标志来标示活动分区,只有内核镜像存在活动分区中的操作系统才可被启动。而Linux的处理方式更为灵活,用一个Bootloader取代这个MBR中不完善的程序,允许用户来选择要启动的操作系统。
LILO或许在MBR上,取代了那个小程序,或许被装在每个分区的引导扇区上。

用户选择引导Linux后,LILO调用BIOS例程打印“Loading”信息,从磁盘读取内核镜像。
\verb$setup()$函数的代码在内核镜像文件的0x200处,LILO装载完内核镜像后会跳转到\verb$setup()$函数处。
\verb$setup()$最终跳转到\verb$startup_32()$函数处。

\verb$startup_32()$函数有两个,依次执行。
其中第二个\verb$startup_32()$会创建被称作进程0的内核线程(又称\textbf{idle进程或swapper进程})。进程0是内核的一部分,不执行任何磁盘程序。
进程0调用\verb$start_kernel()$。

\verb$start_kernel()$初始化内核所需的所有数据结构(此前只有少数几个数据结构被建立),激活中断,创建一个叫进程1的内核线程,又叫\textbf{init进程}。
init进程执行\verb$init()$函数,\verb$init()$完成内核初始化。\verb$init()$函数调用\verb$execve()$装入可执行程序init,结果
init内核线程变成了一个普通进程,并拥有了自己的per-process内核数据结构。在系统关闭前,init进程一直存活,因为它创建和监控在操作系统外层执行的所有进程的活动。

\subsection{init进程}

系统加电之后,首先进行的硬件自检,然后是bootloader对系统的初始化,加载内核。
内核被加载到内存中之后,就开始执行了。
一旦内核启动运行,对硬件的检测就会决定需要对哪些设备驱动程序进行初始化。
从这里开始,内核就能够挂装根文件系统。
内核挂装了根文件系统,并已初始化所有的设备驱动程序和数据结构等之后,就通过启动一个叫init的用户级程序,完成引导进程。

由0号进程创建1号进程(内核态),1号负责执行内核的部分初始化工作及进行系统配置,并创建若干个用于高速缓存和虚拟主存管理的内核线程。随后,1号进程调用execve()运行可执行程序init,并演变成用户态1号进程,即init进程。它按照配置文件/etc/initab的要求,完成系统启动工作,创建编号为1号、2号...的若干终端注册进程getty。每个getty进程设置其进程组标识号,并监视配置到系统终端的接口线路。当检测到来自终端的连接信号时,getty进程将通过函数\verb$execve()$执行注册程序login,此时用户就可输入注册名和密码进入登录过程,如果成功,由login程序再通过函数\verb$execve()$执行shell,该shell进程接收getty进程的pid,取代原来的getty进程。再由shell直接或间接地产生其他进程。

上述过程可描述为:0号进程->1号内核进程->1号内核线程->1号用户进程(init进程)->getty进程->shell进程。


\subsection{init程序}

init是 Unix 和 类Unix 系统中用来产生其它所有进程的程序。它以守护进程的方式存在,其进程号为1。

BSD init 运行存放于'/etc/rc'的初始化 shell 脚本,然后启动基于文本模式的终端(getty)或者基于图形界面的终端(窗口系统,如 X)。
这里没有运行模式的问题,因为文件 'rc' 决定了 init 如何执行。

现代的 BSD 派生系统一直支持使用 'rc.local'
文件的方式,它将在正常启动过程接近最后的时间以子脚本的方式来执行。这样做减少了整个系统无法启动的风险。然后,第三方软件包可以将它们独立的 start/stop 脚本安装到一个本地的 'rc.d' 目录中(通常这是由 ports collection/pkgsrc 完成的)。 FreeBSD 和 NetBSD 现在默认使用 rc.d ,该目录中所有的用户启动脚本,都被分成更小的子脚本,和 SysV 类似。

System V init 检查 '/etc/inittab' 文件中是否含有 'initdefault' 项。 这告诉 init
系统是否有一个默认运行模式。如果没有默认的运行模式,那么用户将进入系统控制台,手动决定进入何种运行模式。

systemd意欲取代System V和BSD风格的init的程序。
在RHEL6中采用的init程序为upstart。在RHEL7中开始采用systemd。

   
\subsection{运行级别}
\begin{itemize}
  \item 0为停机,机器关闭。 
  \item 1为单用户模式,就像Win9x下的安全模式类似。 
  \item 2为多用户模式,但是没有NFS支持。 
  \item 3为完整的多用户模式,是标准的运行级。除了需要在登录后手动启动图形界面外,与级别5相同。
  \item 4一般的发行版没定义这个级别。
  \item 5就是X11,进到X Window系统了。 
  \item 6为重启,运行init 6机器就会重启。 
\end{itemize}
在Ubuntu 14.04上尝试进入级别2,3,4,5均处于X11界面,进入级别1系统退出并卡死。

查看运行级别命令:
 \begin{verbatim}
 runlevel
\end{verbatim}
先后显示系统上一次和当前运行级别。如果不存在上一次运行级别,则用N表示。在Ubuntu上运行结果为“N 2”。

改变提供运行级别命令:
 \begin{verbatim}
 init [0123456]
\end{verbatim}


    
\subsection{内核线程}
除了idle和init进程之外,其他内核线程包括:
\begin{description}
  \item[keventd] 执行\verb$kevent_wq$工作队列中的函数。
  \item[kapmd] 处理与APM(高级电源管理)相关的事件。
  \item[kswapd] 执行内存回收。
  \item[pdflush] 刷新脏内存的内容到磁盘以回收内存。
  \item[kblockd] 执行\verb$kblockd_workqueue$工作队列中的函数。它周期性激活块设备驱动程序, 因为一些需要激活驱动程序的work已被延迟以求提高性能。
  \item[ksoftirqd] 每个CPU都有一个ksoftirqd,运行tasklet。
\end{description}




















    
    
    
    
    
    
    
    
    
    
    
    
    
