%!Mode:: "TeX:UTF-8"
\section{Glibc的malloc实现}
\label{sec:glibc-malloc}

在程序开发中,堆和栈是最常使用的两个内存区,在Linux下栈分为用户栈和内核栈,内核栈具有固定大小,而用户栈可以通过ulimit来设定,最大8M。

Glibc分配算法思想:
\begin{itemize}
\item 小于等于64字节:用pool算法分配
\item 64到512字节之间:在最佳凭配算法分配和pool算法分配中取一种合适的
\item 大于等于512字节:用最佳凭配算法分配
\item 大于等于128K:直接调用OS提供的函数(如mmap)分配
\end{itemize}

\begin{lstlisting}[language=C++]
typedef struct free_list {
  spin_lock_t lock;/* spin lock for mutual exclusion */
  header_t head;/* head of free list for this size */
  #ifdef DEBUG
  int in_use;  /* # mallocs - # frees */
  #endif DEBUG
} *free_list_t;

typedef union header {
  union header *next;
  struct free_list *fl;
} *header_t;

#define MIN_SIZE 8/* minimum block size */
#define NBUCKETS 29

/*block size 8,16,24,...,64*/
static struct free_list malloc_free_list[NBUCKETS];
\end{lstlisting}

内存碎片包括内部碎片和外部碎片。














