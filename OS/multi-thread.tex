%!Mode:: "TeX:UTF-8"
\section{线程安全与可重入性}

\subsection{线程安全}
如果一段代码是\textbf{线程安全}的,那么多个线程在同时执行它时能确保逻辑正确。

维基百科为线程安全程度分为三类:线程安全,有条件线程安全(conditionally thread safe)和线程不安全。
有条件线程安全指对不同对象的并发访问是安全的,对共享对象的访问需被保护以防竞跑(Race)。

有两类策略可用来消除数据竞跑以达到线程安全。
第一类策略致力于消除共享状态:
\begin{itemize}
\item 代码要做到可重入:部分执行后可被重新执行,并能确保原执行正确结束。
状态信息应保存于每次“执行”的本地,一般在栈上。所有非本地状态必须通过原子操作访问,并且数据结构须为可重入的。
\item 使用线程本地存储(Thread-local storage)。
\end{itemize}
第二类策略用于共享状态无法避免的情形,依赖于同步操作,包括互斥锁、原子操作、immutable对象等。
\begin{quotation}
引入自旋锁的隐患包括死锁、“活锁”和资源饥饿。
\end{quotation}

大多数Unix函数是线程安全的(malloc,free,printf),只有少数例外,如rand,asctime,ctime,strtok,gethostbyaddr,gethostbyname,inet\_ntoa。
Unix为大多数线程不安全函数提供了可重入版本,用\_r作为后缀。

\cite{csapp}总结了四类线程不安全函数:
\begin{enumerate}
\item 不保护共享变量的函数。可用同步操作来改进,不用改变原有接口。
\item 保存多次调用间的状态的函数,如stdlib.h的rand()和string.h的strtok()。
	如欲改进必须改变原接口,让调用间状态作为参数传入下一次调用,如非标准的rand\_r()和strtok\_r()。
\item 返回指向静态变量指针的函数,如ctime(), gethostbyname()。
有两种改进策略,一是改变原有接口,让调用者传入额外参数存放返回值,如ctime\_r(), gethostbyname\_r();
二是使用lock-and-copy策略再次封装。
\item 调用线程不安全函数的函数,\textbf{可能}会成为线程不安全函数。
\cite{csapp}认为如果调用了1型或3型线程不安全函数,可以在本函数中增加同步操作wrapper,使本函数做到线程安全。
我认为此举的前提是被调用的不安全函数不会在他处被调用。
\end{enumerate}


\subsection{可重入性}
 可重入概念是在单线程操作系统的时代提出的。
 一段程序或例程是可重入的,如果它能在执行完成之前被打断并作重新调用(重入),当重入的调用完成后原调用能够正确地恢复执行。
一段程序的重入,可能由于自身原因,如执行了jmp或者call,类似于子程序的递归调用;
或者由于硬件中断,UNIX系统的signal的处理,即子程序被中断处理程序或者signal处理程序调用。
重入的子程序,按照后进先出线性序依次执行。

中断服务例程必须是可重入的,它们通常被禁止访问文件系统,甚至不允许分配内存。
直接或间接执行递归的函数也需要是可重入的。

可重入性不同于幂等性(Idempotence, $f(f(x))=f(x)$)。

可重入程序可用于实现线程安全,《CSAPP》认为可重入程序是线程安全程序的真子集,但维基百科认为可重入程序未必是线程安全的,例如下面这个程序。
这是因为不同文献对可重入性的定义不同。
\begin{lstlisting}[language=C++]    
int t;
 
void swap(int *x, int *y)
{
    int s;
 
    s = t; // save global variable
    t = *x;
    *x = *y;
 
    // hardware interrupt might invoke isr() here!
    *y = t;
    t = s; // restore global variable
}
 
void isr()
{
    int x = 1, y = 2;
    swap(&x, &y);
}
\end{lstlisting}




若一个函数是可重入的,则该函数:
\begin{itemize}
\item 最好不要含有静态(全局)非常量数据,除非是通过原子操作访问。
\item 最好不要修改自身代码,除非对自身的代码有私有拷贝。
\item 不能调用(call)不可重入的函数(有呼叫(call)到的函数需满足前述条件)。
\end{itemize}


 下述例子是线程安全的,但不是可重入的:
\begin{lstlisting}[language=C++]    
int function()
{
 mutex_lock();
 ...
 function body
 ...
 mutex_unlock();
}
\end{lstlisting}

 
 
 \subsection{线程安全计数器类的实现\cite{wikipedia}}
以下Java代码为线程安全的:
\begin{lstlisting}[language=Java]  
class Counter {
    private int i = 0; 
    public synchronized void inc() {
        i++;
    }
}
\end{lstlisting}

以下C代码为线程安全的,但不可重入。如果本代码用于一个中断handler中,执行过程中再次发生中断,则第二次调用将永远旋住。
\begin{lstlisting}[language=C]    
#include <pthread.h>
 
int increment_counter ()
{
	static int counter = 0;
	static pthread_mutex_t mutex = PTHREAD_MUTEX_INITIALIZER;
 
	pthread_mutex_lock(&mutex); 
	// only allow one thread to increment at a time
	++counter;
	// store value before any other threads increment it further
	int result = counter; 
	pthread_mutex_unlock(&mutex);
 
	return result;
}
\end{lstlisting}

以下代码用C++的原子类型作无锁实现,为线程安全、可重入的:
\begin{lstlisting}[language=C++]    
#include <atomic>
 
int increment_counter ()
{
	static std::atomic<int> counter(0); 
	int result = ++counter; 
	return result;
}
\end{lstlisting}

 
 
 
 
 \subsection{Java中的线程安全}
对于 Java 类中常见的线程安全性级别,没有一种分类系统可被广泛接受。
Joshua Bloch给出了描述五类线程安全性的分类方法:
\begin{description}
\item[不可变]不可变的对象一定是线程安全的,并且永远也不需要额外的同步。
因为一个不可变的对象只要构建正确,其外部可见状态永远也不会改变,永远也不会看到它处于不一致的状态。
Java 类库中大多数基本数值类如 Integer 、 String 和 BigInteger 都是不可变的。
\item[线程安全]“线程安全”是很严格的,不管运行时环境如何排列,线程都不需要任何额外的同步。
Java的Vector、HashTable都不满足线程安全。
\item[有条件线程安全]有条件的线程安全类对于单独的操作可以是线程安全的,但是某些操作序列可能需要外部同步。
条件线程安全的最常见的例子是遍历由 Hashtable 或者 Vector 或者返回的迭代器。
由这些类返回的由这些类返回的 fail-fast 迭代器假定在迭代器进行遍历的时候底层集合不会有变化。
为了保证其他线程不会在遍历的时候改变集合,进行迭代的线程应该确保它是独占性地访问集合以实现遍历的完整性。
\item[线程兼容]线程兼容类不是线程安全的,但是可以通过正确使用同步而在并发环境中安全地使用。
这可能意味着用一个 synchronized 块包围每一个方法调用,或者创建一个包装器对象,其中每一个方法都是同步的。
\item[线程对立]线程对立类是那些不管是否调用了外部同步都不能在并发使用时安全地呈现的类。
线程对立很少见,当类修改静态数据,而静态数据会影响在其他线程中执行的其他类的行为,这时通常会出现线程对立。
\end{description}
这种系统有其局限性:各类之间的界线不是百分之百地明确,而且有些情况它没照顾到。 
 
\subsection{C++ 的线程安全}
在STL容器(和大多数厂商的愿望)里对多线程支持的黄金规则已经由SGI定义,并且在它们的STL网站上发布。
在访问不同对象的时候无需加锁,对共享对象并发读取时也是安全的。
但多个线程对共享对象进行读、写时,STL的用户必须自行执行保护。
 
 \subsection{False sharing问题}
在做多线程程序的时候,为了避免使用锁,我们通常会采用这样的数据结构:根据线程的数目,安排一个数组, 每个线程一个项,互相不冲突。
从逻辑上看这样的设计无懈可击,但是实践的过程我们会发现这样并没有提高速度。
问题在于cpu的cache line. 我们在读主存的时候,数据同时被读到L1,L2中去,而且在L1中是以cache line(通常64)字节为单位的。
每个Core都有自己的L1,L2,所以每个线程在读取自己的项的时候, 也把别人的项读进去, 所以在别人的项更新的时候,为了保持数据的一致性, core之间cache要进行同步, 这个会导致严重的性能问题. 这就是所谓的False sharing问题 。
\begin{lstlisting}[language=C++]                      
typedef union {
    erts_smp_rwmtx_t rwmtx;
    byte cache_line_align__[ERTS_ALC_CACHE_LINE_ALIGN_SIZE(
                                sizeof(erts_smp_rwmtx_t))];
} erts_meta_main_tab_lock_t;
\end{lstlisting}

或者:
\begin{lstlisting}[language=C++]                      
_declspec (align(64)) int thread1_global_variable;
__declspec (align(64)) int thread2_global_variable;
\end{lstlisting}
 
 








