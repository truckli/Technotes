%!Mode:: "TeX:UTF-8"
\section{awk语言}

AWK是一种优良的文本处理工具,是Linux及Unix环境中现有的功能最强大的数据处理引擎之一。
AWK提供了极其强大的功能:可以进行正则表达式的匹配,样式装入、流控制、数学运算符、进程控制语句甚至于内置的变量和函数。它具备了一个完整的语言所应具有的几乎所有精美特性。实际上AWK的确拥有自己的语言:AWK程序设计语言,三位创建者已将它正式定义为“样式扫描和处理语言”。它允许您创建简短的程序,这些程序读取输入文件、为数据排序、处理数据、对输入执行计算以及生成报表,还有无数其他的功能。gawk是AWK的GNU版本。

Awk提供了适应多种需要的不同解决方案,它们是:
\begin{enumerate}
\item awk命令行,在命令行中用单引号包含awk程序设计语句
\item ’\verb$awk -f filename$‘ 形式
\item 利用命令解释器调用awk程序:\verb$’#!/bin/awk -f‘$
\end{enumerate}

AWK程序是由一系列模式--动作对组成的,写做’pattern { action }‘。action包含以下部分:
\begin{verbatim}
BEGIN{ 这里面放的是执行前的语句 }
END {这里面放的是处理完所有的行后要执行的语句 }
{这里面放的是处理每一行时要执行的语句}
\end{verbatim}

强引用(单引号)和大括号用来包含shell脚本中的awk代码段.
Awk将传递进来的每行输入都分割成域. 默认情况下, 一个域指的就是使用空白分隔的一个连续字符串, 不过我们可以修改属性来改变分隔符. Awk将会分析并操作每个分割域. 因为这种特性, 所以awk非常善于处理结构化的文本文件 -- 尤其是表 -- 将数据组织成统一的块, 比如说分成行和列.

假设有如下文件:
\begin{verbatim}
$ cat netstat.txt
Proto Recv-Q Send-Q Local-Address          Foreign-Address             State
tcp        0      0 0.0.0.0:3306           0.0.0.0:*                   LISTEN
tcp        0      0 0.0.0.0:80             0.0.0.0:*                   LISTEN
tcp        0      0 127.0.0.1:9000         0.0.0.0:*                   LISTEN
tcp        0      0 coolshell.cn:80        124.205.5.146:18245         TIME_WAIT
tcp        0      0 coolshell.cn:80        61.140.101.185:37538        FIN_WAIT2
tcp        0      0 coolshell.cn:80        110.194.134.189:1032        ESTABLISHED
tcp        0      0 coolshell.cn:80        123.169.124.111:49809       ESTABLISHED
tcp        0      0 coolshell.cn:80        116.234.127.77:11502        FIN_WAIT2
tcp        0      0 coolshell.cn:80        123.169.124.111:49829       ESTABLISHED
tcp        0      0 coolshell.cn:80        183.60.215.36:36970         TIME_WAIT
tcp        0   4166 coolshell.cn:80        61.148.242.38:30901         ESTABLISHED
tcp        0      1 coolshell.cn:80        124.152.181.209:26825       FIN_WAIT1
tcp        0      0 coolshell.cn:80        110.194.134.189:4796        ESTABLISHED
tcp        0      0 coolshell.cn:80        183.60.212.163:51082        TIME_WAIT
tcp        0      1 coolshell.cn:80        208.115.113.92:50601        LAST_ACK
tcp        0      0 coolshell.cn:80        123.169.124.111:49840       ESTABLISHED
tcp        0      0 coolshell.cn:80        117.136.20.85:50025         FIN_WAIT2
tcp        0      0 :::22                  :::*                        LISTEN 
\end{verbatim}

选列打印:
\begin{verbatim}
$ awk '{print $1, $4}' netstat.txt
\end{verbatim}

awk的格式化输出,和C语言的printf语法一致。\verb|$0|表示整段记录。
\begin{verbatim}
$ awk '{printf "%-8s %-8s %-8s %-18s %-22s %-15s\n",$1,$2,$3,$4,$5,$6}' netstat.txt
Proto    Recv-Q   Send-Q   Local-Address      Foreign-Address        State
tcp      0        0        0.0.0.0:3306       0.0.0.0:*              LISTEN
tcp      0        0        0.0.0.0:80         0.0.0.0:*              LISTEN
tcp      0        0        127.0.0.1:9000     0.0.0.0:*              LISTEN
tcp      0        0        coolshell.cn:80    124.205.5.146:18245    TIME_WAIT
tcp      0        0        coolshell.cn:80    61.140.101.185:37538   FIN_WAIT2
tcp      0        0        coolshell.cn:80    110.194.134.189:1032   ESTABLISHED
tcp      0        0        coolshell.cn:80    123.169.124.111:49809  ESTABLISHED
tcp      0        0        coolshell.cn:80    116.234.127.77:11502   FIN_WAIT2
tcp      0        0        coolshell.cn:80    123.169.124.111:49829  ESTABLISHED
tcp      0        0        coolshell.cn:80    183.60.215.36:36970    TIME_WAIT
tcp      0        4166     coolshell.cn:80    61.148.242.38:30901    ESTABLISHED
tcp      0        1        coolshell.cn:80    124.152.181.209:26825  FIN_WAIT1
tcp      0        0        coolshell.cn:80    110.194.134.189:4796   ESTABLISHED
tcp      0        0        coolshell.cn:80    183.60.212.163:51082   TIME_WAIT
tcp      0        1        coolshell.cn:80    208.115.113.92:50601   LAST_ACK
tcp      0        0        coolshell.cn:80    123.169.124.111:49840  ESTABLISHED
tcp      0        0        coolshell.cn:80    117.136.20.85:50025    FIN_WAIT2
tcp      0        0        :::22              :::*                   LISTEN
\end{verbatim}

过滤功能:
\begin{verbatim}
$ awk '$3==0 && $6=="LISTEN" || NR==1 {printf "%-20s %-20s %s\n",$4,$5,$6}' netstat.txt
Local-Address        Foreign-Address      State
0.0.0.0:3306         0.0.0.0:*            LISTEN
0.0.0.0:80           0.0.0.0:*            LISTEN
127.0.0.1:9000       0.0.0.0:*            LISTEN
:::22                :::*                 LISTEN
\end{verbatim}
NR是内建变量,表示到目前为止已经处理的行数。

指定分隔符(内建变量FS),以处理不以空格为分隔符的文件:
\begin{verbatim}
$  awk  'BEGIN{FS=":"} {print $1,$3,$6}' /etc/passwd
root 0 /root
bin 1 /bin
daemon 2 /sbin
adm 3 /var/adm
lp 4 /var/spool/lpd
sync 5 /sbin
shutdown 6 /sbin
halt 7 /sbin
\end{verbatim}

字符串匹配:
\begin{verbatim}
$ awk '$6 ~ /FIN/ || NR==1 {print NR,$4,$5,$6}' OFS="\t" netstat.txt
1       Local-Address   Foreign-Address State
6       coolshell.cn:80 61.140.101.185:37538    FIN_WAIT2
9       coolshell.cn:80 116.234.127.77:11502    FIN_WAIT2
13      coolshell.cn:80 124.152.181.209:26825   FIN_WAIT1
18      coolshell.cn:80 117.136.20.85:50025     FIN_WAIT2

或运算:
$ awk '$6 ~ /FIN|TIME/ || NR==1 {print NR,$4,$5,$6}' OFS="\t" netstat.txt
1       Local-Address   Foreign-Address State
5       coolshell.cn:80 124.205.5.146:18245     TIME_WAIT
6       coolshell.cn:80 61.140.101.185:37538    FIN_WAIT2
9       coolshell.cn:80 116.234.127.77:11502    FIN_WAIT2
11      coolshell.cn:80 183.60.215.36:36970     TIME_WAIT
13      coolshell.cn:80 124.152.181.209:26825   FIN_WAIT1
15      coolshell.cn:80 183.60.212.163:51082    TIME_WAIT
18      coolshell.cn:80 117.136.20.85:50025     FIN_WAIT2

模式取反:
$ awk '$6 !~ /WAIT/ || NR==1 {print NR,$4,$5,$6}' OFS="\t" netstat.txt
1       Local-Address   Foreign-Address State
2       0.0.0.0:3306    0.0.0.0:*       LISTEN
3       0.0.0.0:80      0.0.0.0:*       LISTEN
4       127.0.0.1:9000  0.0.0.0:*       LISTEN
7       coolshell.cn:80 110.194.134.189:1032    ESTABLISHED
8       coolshell.cn:80 123.169.124.111:49809   ESTABLISHED
10      coolshell.cn:80 123.169.124.111:49829   ESTABLISHED
12      coolshell.cn:80 61.148.242.38:30901     ESTABLISHED
14      coolshell.cn:80 110.194.134.189:4796    ESTABLISHED
16      coolshell.cn:80 208.115.113.92:50601    LAST_ACK
17      coolshell.cn:80 123.169.124.111:49840   ESTABLISHED
19      :::22   :::*    LISTEN
\end{verbatim}

其实awk可以像grep一样的去匹配第一行,就像这样:
\begin{verbatim}
$ awk '/LISTEN/' netstat.txt
tcp        0      0 0.0.0.0:3306            0.0.0.0:*               LISTEN
tcp        0      0 0.0.0.0:80              0.0.0.0:*               LISTEN
tcp        0      0 127.0.0.1:9000          0.0.0.0:*               LISTEN
tcp        0      0 :::22                   :::*                    LISTEN
\end{verbatim}

另一个文本例子:
\begin{verbatim}
$ cat score.txt
Marry   2143 78 84 77
Jack    2321 66 78 45
Tom     2122 48 77 71
Mike    2537 87 97 95
Bob     2415 40 57 62

$ cat cal.awk
#!/bin/awk -f
#运行前
BEGIN {
    math = 0
    english = 0
    computer = 0

    printf "NAME    NO.   MATH  ENGLISH  COMPUTER   TOTAL\n"
    printf "---------------------------------------------\n"
}
#运行中
{
    math+=$3
    english+=$4
    computer+=$5
    printf "%-6s %-6s %4d %8d %8d %8d\n", $1, $2, $3,$4,$5, $3+$4+$5
}
#运行后
END {
    printf "---------------------------------------------\n"
    printf "  TOTAL:%10d %8d %8d \n", math, english, computer
    printf "AVERAGE:%10.2f %8.2f %8.2f\n", math/NR, english/NR, computer/NR
}

$ awk -f cal.awk score.txt
NAME    NO.   MATH  ENGLISH  COMPUTER   TOTAL
---------------------------------------------
Marry  2143     78       84       77      239
Jack   2321     66       78       45      189
Tom    2122     48       77       71      196
Mike   2537     87       97       95      279
Bob    2415     40       57       62      159
---------------------------------------------
  TOTAL:       319      393      350
AVERAGE:     63.80    78.60    70.00
\end{verbatim}

统计当前目录下有多少源程序文件:
\begin{verbatim}
$ ls -l  *.cpp *.c *.h | awk '{sum+=$5} END {print sum}'
2511401
\end{verbatim}


统计各个connection状态:
\begin{verbatim}
$ awk 'NR!=1{a[$6]++;} END {for (i in a) print i ", " a[i];}' netstat.txt
TIME_WAIT, 3
FIN_WAIT1, 1
ESTABLISHED, 6
FIN_WAIT2, 3
LAST_ACK, 1
LISTEN, 4
\end{verbatim}

统计每个用户的进程的占了多少内存:
\begin{verbatim}
$ ps aux | awk 'NR!=1{a[$1]+=$6;} END { for(i in a) print i ", " a[i]"KB";}'
dbus, 540KB
mysql, 99928KB
www, 3264924KB
root, 63644KB
hchen, 6020KB
\end{verbatim}




