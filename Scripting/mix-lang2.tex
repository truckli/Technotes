%!Mode:: "TeX:UTF-8"

\section{Mixed Language Programming}
\subsection{Mixing C with C++}
\begin{verbatim}
#ifdef __cplusplus
extern C {}
\end{verbatim}

\subsection{Calling C from Python with ctypes}

\url{https://segmentfault.com/a/1190000000479951}

\url{https://www.zhihu.com/question/23003213}

\href{https://docs.python.org/2/library/ctypes.html##ctypes-tutorial}{Ctypes Tutorial}

\href{https://docs.python.org/2/library/ctypes.html#ctypes-reference}{Ctypes Reference}

 
 creating a dynamic library from a .o file:

\begin{verbatim}
ar crv libctest.a c_test.o
gcc -shared -fPCI -o libctest.so c_test.o
\end{verbatim}

ctypes example:
\begin{verbatim}
from ctypes import *
libc = CDLL("libc.so.6")
libc.printf("hello world\n")
libget = CDLL("libgetinfo.so")
libget.nisc_get_rx_free_ratio.restype = POINTER(c_double * 24) 
ratios = libget.nisc_get_rx_free_ratio().contents
\end{verbatim}

\subsubsection{Returning a pointer to a structure}

C part:
\begin{verbatim}
typedef struct pag_stats_node
{
	double core_usage;
	double core_freq;
	uint32_t buf_size;
} stats_node_t;

stats_node_t st;
stats_node_t *test()
{
	st.core_usage = 1;
	st.core_freq = 2;
	st.buf_size = 3;
	return &st;
}
\end{verbatim}

Python part:
\begin{verbatim}
class stat(Structure):
     _fields_=[("u",c_double),
              ("f",c_double),
              ("b",c_uint)]

libget.test.restype = POINTER(stat)
statv = libget.test()
print statv.contents.u
print statv.contents.f
print statv.contents.b
\end{verbatim}


\subsection{Calling Python from C}
\begin{verbatim}
#include <Python.h>

int main(int argc, char *argv[])
{
  Py_SetProgramName(argv[0]);
  Py_Initialize();
  PyRun_SimpleString("print 'Hello Python!'\n");
  Py_Finalize();
  return 0;
}
\end{verbatim}

To compile:
\begin{verbatim}
gcc my_python.c -o my_python -I/usr/include/python2.7/ -lpython2.7
\end{verbatim}

\subsection{Calling Java from Python with Py4j}

Solutions include Pyjnius/Jnius, JCC, javabridge, Jpype and Py4j.
Jpype works pretty well and is proven in many projects (such as
python-boilerpipe), but Pyjnius is faster and simpler than JPype.
Py4j is a bit hard to use, as you need to start a gateway, adding another layer
of fragility.

Here is a simple example for Py4j. 
Suppose you have a Java project in which your Python code requires access to
MyClass.

\begin{lstlisting}[language=Java]
// In MyClass.java
package py4j.examples;

public class MyClass {
  public int addOne(int x) {
    return x+1;
  }
}

\end{lstlisting}

To enable Python access, first you write a entry point class in the same
project:

\begin{lstlisting}[language=Java]
package py4j.examples;

import py4j.GatewayServer;

public class EntryPoint {
    public static void main(String[] args) {
        GatewayServer gatewayServer = new GatewayServer(new EntryPoint());
        gatewayServer.start();
        System.out.println("Gateway Server Started ... ");
    }
}
\end{lstlisting}

Then you compile the project into entry.jar and run it:

\begin{verbatim}
java -cp entry.jar py4j.examples.EntryPoint
\end{verbatim}

Now your Python code can access MyClass via this entry point:

\begin{lstlisting}
from py4j.java_gateway import JavaGateway
gateway = JavaGateway()
jobj = gateway.jvm.py4j.examples.MyClass()
print jobj.addOne(5)
\end{lstlisting}







