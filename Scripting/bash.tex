%!Mode:: "TeX:UTF-8"
\section{Bash脚本的基本使用}

\subsection{变量}

一些特殊的位置变量:
\verb|$*|表示所有参数,\verb|$@|类似,

\verb|$#|表示所有参数的个数。

\verb|$n|为位置参数,特别地,\verb|$0|为脚本名称。

\verb|$?|表示上一条命令的退出状态。

\verb|$$|表示本shell进程的PID 
。






eval用于得到变量的"值的值"
\begin{verbatim}
foo=10
x=foo
y=’$’$x
echo $y
\end{verbatim}
会得到"\$foo",而
\begin{verbatim}
foo=10
x=foo
eval y=’$’$x
echo $y
\end{verbatim}
会得到10。

另一个eval用法:
\begin{verbatim}
op=1;value_$op=110;echo $value_1
\end{verbatim}



set为当前上下文设置参数变量。

\begin{verbatim}
#!/bin/sh
echo the date is $(date)
set $(date)
echo The month is $2
exit 0
\end{verbatim}

unset命令取消一个变量:
\begin{verbatim}
foo=”Hello World”
echo $foo
unset foo
echo $foo
\end{verbatim}

export命令让变量在子shell可见。

倒引号 (backticks, \verb|``|)用于命令替换,但\verb|$(...)|是另一种(更新的)方式:
\begin{verbatim}
output=$(sed -n /"$1"/p $file)
\end{verbatim}

\verb|$#|表示参数个数,
\verb|$@|在Bash中表示所有参数集合,在参数不确定时非常有用。完美世界面试时曾考我这个知识。


\subsection{条件测试}
\begin{verbatim}
#!/bin/sh
echo “Is it morning? Please answer yes or no”
read timeofday
if [ “$timeofday” = “yes” ]
then
  echo “Good morning”
elif [  “$timeofday” = “no” ]; then
  echo “Good afternoon”
else
  echo “Sorry, $timeofday not recognized. Enter yes or no”
  exit 1
fi

exit 0
\end{verbatim}
注意,\verb$timeofday$加双引号是为了以防其为空串时不合语法。

命令的连接同C语言:\verb$ $$, || $。
\begin{verbatim}
if [ -f file_one ] && echo “hello” && [ -f file_two ] && echo “ there”
then
  echo “in if”
else
  echo “in else”
fi
\end{verbatim}

测试条件包括:
\begin{description}
    \item[string1 = string2] 字符串相等
    \item[-n string] 字符串非空
    \item[-z string] 字符串空
    \item[expression1 -eq expression2] 表达式相等
    \item[expression1 -ne expression2] 表达式不等
    \item[expression1 -gt expression2] 大于。如\verb|a=3; if [ "$a" -gt 0 ] ...|
    \item[expression1 -ge expression2] 大于等于
    \item[-d file]是否为目录
    \item[-f file]是否为普通文件
    \item[-r file]文件可读,类似有-w,-x选项
    \item[-s file]non-zero size文件
    \item[-S file]文件是个socket
    \item[d]删除行
    \item[p]打印行
\end{description}

\subsection{循环}
\begin{verbatim}
#!/bin/sh
for foo in bar fud 43
do
echo $foo
done
exit 0
\end{verbatim}

\begin{verbatim}
for file in $(ls f*.sh); do
  lpr $file
done
exit 0
\end{verbatim}

\begin{verbatim}
#!/bin/sh
echo “Enter password”
read trythis
while [ “$trythis” != “secret” ]; do
  echo “Sorry, try again”
  read trythis
done
exit 0
\end{verbatim}





\subsection{算术运算}
主要有三种语法实现算术运输,expr,let和双小括号。第一种方式比较旧。
\begin{verbatim}
1.x=`expr $x + 1`或x=$(expr $x + 1),运算符两侧需要空格
2.x=$(($x+1))或x=$((x+1))或((x+=1))或((x += 1)),在双小括号中'$'和空格可选
3.let x=$x+1或let "x+=1"或 let "x += 1"
\end{verbatim}
对于expr,注意乘法符号\verb$*$可能需要转义。

\subsection{字符串运算}
有多种方式表示字符串长度:
\begin{verbatim}
${#string}
expr length $string
expr "$string" : '.*'
\end{verbatim}

\verb|expr index $string $substring|用于匹配,返回匹配位置。

\verb|${string:pos}|用于截取子串,从pos所示位置开始。

\verb|${string:pos:len}|用于截取子串,从pos所示位置开始,并指定子串长度。

\verb|${string#substring}|用于删除子串,前向最短匹配。

\verb|${string##substring}|用于删除子串,前向最短匹配。

\verb|${string%substring}|用于删除子串,后向最短匹配。

\verb|${string%%substring}|用于删除子串,后向最长匹配。

\verb|${string/substring/replacement}|用于替换第一个匹配。

\verb|${string//substring/replacement}|用于替换所有匹配。



\begin{verbatim}

\end{verbatim}

\begin{verbatim}

\end{verbatim}


\subsection{函数}
\begin{verbatim}
#!/bin/sh
foo() {
  local loc_var=23 # Declared as local variable
  echo “$1”
}

echo “script starting”
foo "haha"
echo “script ended”
exit 0
\end{verbatim}


\subsection{逐行读取}
\begin{verbatim}
方法一,指定换行符读取:
#! /bin/bash  
  
IFS="  
"  
  
for LINE in `cat /etc/passwd`  
do   
        echo $LINE 
done
 
方法二,文件重定向给read处理:
#! /bin/bash  
  
cat /etc/passwd | while read LINE  
do
        echo $LINE 
done
 
 
方法三,用read读取文件重定向:
#! /bin/bash  
  
while read LINE
do
        echo $LINE 
done < /etc/passwd
\end{verbatim}








