%!Mode:: "TeX:UTF-8"
\section{正则表达式}

常用元字符:
\begin{itemize}
\item \verb$\w$:匹配字母或数字或下划线或汉字。在英文中等于\verb$[a-z0-9A-Z_]$
\item \verb$\W$:匹配不是字母或数字或下划线或汉字的字符
\item \verb$\s$:匹配任意的空白符
\item \verb$\S$:匹配任意的非空白符
\item \verb$\d$:匹配数字
\item \verb$\D$:匹配非数字字符
\item \verb$\b$:匹配单词的开始或结束。它的前一个字符和后一个字符不全是(一个是,一个不是或不存在)\verb$\w$
\item \verb$\B$:匹配非\verb$\b$字符
\item \verb$[aeiou]$:匹配某元音字母
\item \verb$[^aeiou]$:匹配非\verb$[aeiou]$字符
\item \verb$^$:匹配字符串的开始
\item \verb|$|:匹配字符串的结束
\item \verb$.$: 匹配除换行符以外的任意字符
\end{itemize}

常用限定符:
\begin{itemize}
\item \verb$*$: 零次或多次
\item \verb$+$: 一次或多次
\item \verb$?$:零次或一次
\item \verb${n}$:n次。如\verb$\b\w{6}\b$匹配长度为6的单词。
\item \verb${n,}$:至少n次
\item \verb${n,m}$:n次到m次
\end{itemize}

正则表达式里的分枝条件指的是有几种规则,如果满足其中任意一种规则都应该当成匹配,但前面的规则优先:
\verb$\d{5}-\d{4}|\d{5}$这个表达式用于匹配美国的邮政编码。美国邮编的规则是5位数字,或者用连字号间隔的9位数字。
如果写成\verb$\d{5}|\d{5}-\d{4}$则只能匹配5位数字,因为前面的规则短路包含了后面的规则。

\verb$0\d{2}-\d{8}|0\d{3}-\d{7}$这个表达式能匹配两种以连字号分隔的电话号码:
一种是三位区号,8位本地号(如010-12345678),一种是4位区号,7位本地号(0376-2233445)。

\verb$((2[0-4]\d|25[0-5]|[01]?\d\d?)\.){3}(2[0-4]\d|25[0-5]|[01]?\d\d?)$匹配合法的IPv4地址。
小括号的另一种用途是通过语法\verb$(?#comment)$来包含注释。例如:\verb$2[0-4]\d(?#200-249)|25[0-5](?#250-255)|[01]?\d\d?(?#0-199)$。

正则表达式默认贪婪匹配,a.*b,它将会匹配\textbf{最长的}以a开始,以b结束的字符串。
前面给出的限定符都可以被转化为懒惰匹配模式,只要在它后面加上一个问号。\verb$a.*?b$匹配最短的,以a开始,以b结束的字符串。
如果把它应用于aabab的话,它会匹配aab(第一到第三个字符)和ab(第四到第五个字符)。
为什么第一个匹配是aab(第一到第三个字符)而不是ab(第二到第三个字符)?
简单地说,因为正则表达式有另一条规则,比懒惰/贪婪规则的优先级更高:最先开始的匹配拥有最高的优先权。
























