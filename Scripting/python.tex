%!Mode:: "TeX:UTF-8"

\section{Python Examples}
\subsection{Python版本查询}
\begin{verbatim}
shell执行:python -V
Python脚本:import sys;print sys.version
\end{verbatim}

\subsection{Python包}
额外包安装:
\begin{verbatim}
pip install package
\end{verbatim}

升级pip:
\begin{verbatim}
python -m pip install --upgrade --force pip
\end{verbatim}


\subsection{Python类型判断}
\begin{verbatim}
>>>a=1
>>> isinstance(a,int)
True
>>> isinstance(a,(int,float))
True
>>> type(a)
<type 'int'>
>>> type(int())
<type 'int'>
>>> class AClass():
...     pass
... 
>>> a=AClass
>>> type(a)
<type 'classobj'>
>>> isinstance(a, AClass)
False
>>> b=AClass()
>>> isinstance(b, AClass)
True
\end{verbatim}

dir(A)列出A所属类的所有成员方法。

\subsection{Python特殊矩阵}
\begin{verbatim}
buckets = [0] * 100 #这种矩阵可能会出现赋值异常,即会保持所有值恒等。
buckets = [[0 for col in range(5)] for row in range(10)]

w, h = 100, 100
bucket = [[None] * w for i in range(h)]

import numpy
zarray = numpy.zeros(100)


\end{verbatim}

\subsection{lambda运算}
“Lambda 表达式”(lambda expression)是一个匿名函数.

\begin{lstlisting}
g = lambda x:x*2
print g(3)
 
li=[{"age":20,"name":"def"},{"age":25,"name":"abc"},{"age":10,"name":"ghi"}]
li=sorted(li,key=lambda x:x["age"])
print(li)
\end{lstlisting}





\section{Python脚本常用功能}
常用对象和函数
\begin{lstlisting}
    sys.argv
    len(sys.argv)
    os.system()
    time.sleep()
    sys.exit()
    os.path.exists()
    os.path.isfile()
    os.path.split()
    os.path.basename()
    __file__
\end{lstlisting}


\subsection{正则匹配}
Use \verb|re.search(pattern, string, flags=0)| API.

E.g:
\begin{lstlisting}
import re
m = re.search('(?<=abc)def', 'abcdef')
print m.group(0)
\end{lstlisting}

\subsection{检查文件是否存在}
os.path.isfile(filename)

检查文件是否可执行:
\begin{verbatim}
fpath = commands.getoutput('which %s'% handler)
if not (os.path.isfile(fpath) and os.access(fpath, os.X_OK)):
\end{verbatim}

\subsection{Current Script Name}
\begin{lstlisting}
os.path.basename(sys.argv[0])
os.path.basename(__file__)
os.path.split(sys.argv[0])[1]
\end{lstlisting}

\subsection{Current Path}
\begin{lstlisting}
    os.path.dirname(__file__)
\end{lstlisting}


\subsection{List files in some folder}
\begin{lstlisting}
	ifaces = os.listdir('/sys/class/net')
\end{lstlisting}

\subsection{执行外部程序}
\begin{verbatim}
os.system(cmd)
commands.getstatusoutput(cmd)
subprocess.Popen([],...)
subprocess.check_output(["echo", "Hello World!"])
pexpect.spawn()
\end{verbatim}
subprocess模块定义了Popen类,试图取代os.system,os.spawn,os.popen, popen2,commands等模块。除Popen外,还定义了一些简洁函数call, check\_call, check\_output.详见subprocess文档。

pexpect包含的spawn类具有强大的交互功能,适用于ssh,scp等工具。

如果需要非阻塞式地获取进程的输出,似乎只能用subprocess.Popen或pexpect.spawn。

For Python 2.7+, get command output like this:
\begin{lstlisting}
output=subprocess.check_output(["echo", "Hello World!"])
\end{lstlisting}

Before Python 2.7, this can get command output:
\begin{lstlisting}
output = subprocess.Popen(['echo', 'Hello World'], stdout=subprocess.PIPE).communicate()[0]
\end{lstlisting}



\subsection{遍历stdin和文件}

文件的read()方法:读取整个文件内容。
\begin{verbatim}
import sys
text = sys.stdin.read()
words = text.split()
\end{verbatim}

文件的readline()方法:

\begin{verbatim}
f = open(filename)
while True:
line = f.readline()
if not line: break
process(line)
f.close()
\end{verbatim}

文件的readlines方法:
\begin{verbatim}
f = open(filename)
for line in f.readlines():
process(line)
f.close()
\end{verbatim}

文件迭代器:
\begin{verbatim}
f = open(filename)
for line in f:
process(line)
f.close()
\end{verbatim}

\section{Python科研仿真程序常用功能}

\subsection{浮点除法}
\begin{lstlisting}
from __future__ import division
\end{lstlisting}
即可让除法默认为浮点除法。
\subsection{随机数}
random模块提供了random,uniform, shuffle等函数。


\section{Excel File Read and Write}
Relevant libs include xlrd,xlwt, xlutils,Python-xlsx and PyXLSX,openpyxl.
Site \url{www.python-excel.org} contains pointers to information available about working with Excel files in the Python
programming language.

\textbf{openpyxl} is the recommended package for reading and writing Excel 2010
files (ie: .xlsx). 
\textbf{xlrd} and \textbf{xlwt} packages are for reading from and writing to
older Excel files (ie: .xls).
 

xlrd example:
\begin{lstlisting}
import xlrd
data = xlrd.open_workbook('2016.xls')
table = data.sheets()[0]
nrows = table.nrows
ncols = table.ncols
val = table.cell(3, 2).value
\end{lstlisting}

xlwt example:
\begin{lstlisting}
import xlwt
book = xlwt.Workbook()
sheet1 = book.add_sheet('Sheet 1')
sheet1.write(3, 2, 'hello')
\end{lstlisting}



\section{Python时间与日期} 

Python的datetime包提供了日期与时间相关的操作。
\begin{verbatim}
#datetime提供了date,datetime,timedelta等类
from datetime import * 
#日期差计算
date(2012,7,28)-date(2012,7,25)
date.today()-date(2012,7,25)
#时间差计算
#datetime参数中,时、分、秒、微秒可选
datetime.now()-datetime(2012,7,25,09,23,45,23333)
#日期推算
#timedelta参数依次为天、秒、微秒,后两者可选
datetime(2012,7,25)+timedelta(2) #2天后的日期
date(2012,7,25)+timedelta(2) #2天后的日期
\end{verbatim}

两个date对象相减,得到的是timedelta类型的对象,如果想返回整数,则有:
\begin{verbatim}
(date.today()-date(2012,7,25)).days
\end{verbatim}

calendar模块提供了查询平闰年和星期的功能,如
\begin{verbatim}
calendar.isleap(2000)
calendar.weekday(2000, 1, 1) #周一是0,周日是6
\end{verbatim}
calendar也能产生日历字符串如:
\begin{verbatim}
print calendar.month(2000, 1) #月历
print calendar.calendar(2000) #年历
\end{verbatim}

\label{sec:pythonTimeCalc}


\section{Python字符编码问题}
\subsection{Python中文字符输出}
\begin{verbatim}
#!/usr/bin/python
# -*- coding: utf-8 -*-

u"中文¥ chinese"
\end{verbatim}

\subsection{Python Unicode序列与字符串}
unicode(str,encoding)函数将字符串str转换为unicode序列。

str.encode(encoding)将某种编码的字符串转为另一种编码的字符串。如:
\begin{verbatim}
str = u'你'
str2 = str.encode('gbk')
str3 = str.encode('utf8')#此处也作'utf-8'
\end{verbatim}


一个unicode数值可表示为字符串\verb|u'%c'%i|.中文“你”的unicode序列为0x4f60,unicode字符串为\verb|u'\u4f60'|。假设unicode数字序列用链表uarray表示,则其转换为unicode编码的字符串的过程为:
\begin{verbatim}
str = ''
for ucode in uarray: str += '%c'%ucode
\end{verbatim}

\label{subsec:PythonUnicode}

\section{Terminal Handling: pexpect, cmd}

\begin{lstlisting}
import pexpect
child = pexpect.spawn("ssh root@172.16.0.120 -p 2222")
child.logfile = open("/tmp/mylog", "w")
child.expect(".*assword:")
child.send("XXXXXXX\r")
child.expect(".*\$ ")
child.sendline("ls\r")
child.expect(".*\$ ")
\end{lstlisting}


\begin{lstlisting}
import cmd
import string, sys

class CLI(cmd.Cmd):

    def __init__(self):
        cmd.Cmd.__init__(self)
        self.prompt = '> '

    def do_hello(self, arg):
        print "hello again", arg, "!"

    def help_hello(self):
        print "syntax: hello [message]",
        print "-- prints a hello message"

    def do_quit(self, arg):
        sys.exit(1)

    def help_quit(self):
        print "syntax: quit",
        print "-- terminates the application"

    # shortcuts
    do_q = do_quit

#
# try it out

cli = CLI()
\end{lstlisting}

























