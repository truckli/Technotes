%!Mode:: "TeX:UTF-8"

\section{trait技法}

特性萃取技术依赖于trait模板、category类体系和重载函数体系。

标准STL定义了\verb$struct iterator_traits$,其中的\verb$iterator_category$字段可取五种类型,对应五类迭代器。
\begin{lstlisting}[language=C++]

struct input_iterator_tag {};
struct output_iterator_tag {};
struct forward_iterator_tag : public input_iterator_tag {};
struct bidirectional_iterator_tag : public forward_iterator_tag {};
struct random_access_iterator_tag : public bidirectional_iterator_tag {};

template <class Iterator>
struct iterator_traits {
  typedef typename Iterator::iterator_category iterator_category;
  typedef typename Iterator::value_type        value_type;
  typedef typename Iterator::difference_type   difference_type;
  typedef typename Iterator::pointer           pointer;
  typedef typename Iterator::reference         reference;
};

\end{lstlisting}

迭代器类和迭代器tag类是两个相关联的类体系结构。
以下是两个迭代器类:
\begin{lstlisting}[language=C++]

template <class T, class Distance> struct input_iterator {
  typedef input_iterator_tag iterator_category;
  typedef T                  value_type;
  typedef Distance           difference_type;
  typedef T*                 pointer;
  typedef T&                 reference;
};

template <class T, class Distance> struct random_access_iterator {
  typedef random_access_iterator_tag iterator_category;
  typedef T                          value_type;
  typedef Distance                   difference_type;
  typedef T*                         pointer;
  typedef T&                         reference;
};

\end{lstlisting}
还有一个通用的迭代器类:
\begin{lstlisting}[language=C++]


struct iterator {
  typedef Category  iterator_category;
  typedef T         value_type;
  typedef Distance  difference_type;
  typedef Pointer   pointer;
  typedef Reference reference;
};

\end{lstlisting}

内置指针也是一种迭代器:
\begin{lstlisting}[language=C++]

template <class T>
struct iterator_traits<T*> {
  typedef random_access_iterator_tag iterator_category;
  typedef T                          value_type;
  typedef ptrdiff_t                  difference_type;
  typedef T*                         pointer;
  typedef T&                         reference;
};

\end{lstlisting}

这是一个便利函数,返回一个迭代器类对应的\verb$iterator_category$类实例,实现了两个类体系结构的对应:
\begin{lstlisting}[language=C++]

template <class Iterator>
inline typename iterator_traits<Iterator>::iterator_category
iterator_category(const Iterator&) {
  typedef typename iterator_traits<Iterator>::iterator_category category;
  return category();
}

\end{lstlisting}

于是可根据不同的\verb$iterator_category$类实例实现不同的操作:
\begin{lstlisting}[language=C++]

template <class InputIterator, class Distance>
inline void advance(InputIterator& i, Distance n) {
  __advance(i, n, iterator_category(i));
}

template <class InputIterator, class Distance>
inline void __advance(InputIterator& i, Distance n, input_iterator_tag) {
  while (n--) ++i;
}

template <class RandomAccessIterator, class Distance>
inline void __advance(RandomAccessIterator& i, Distance n, 
                      random_access_iterator_tag) {
  i += n;
}


\end{lstlisting}

SGI STL定义了非标准的$type_traits$把这种技法扩大到了迭代器以外的世界。

\begin{lstlisting}[language=C++]
template <class ForwardIterator>
inline void
__destroy_aux(ForwardIterator first, ForwardIterator last, __false_type) {
  for ( ; first < last; ++first)
    destroy(&*first);
}

template <class ForwardIterator> 
inline void __destroy_aux(ForwardIterator, ForwardIterator, __true_type) {}

template <class ForwardIterator, class T>
inline void __destroy(ForwardIterator first, ForwardIterator last, T*) {
  typedef typename __type_traits<T>::has_trivial_destructor trivial_destructor;
  __destroy_aux(first, last, trivial_destructor());
}

\end{lstlisting}








