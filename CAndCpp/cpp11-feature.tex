%!Mode:: "TeX:UTF-8"
\section{C++ 2011}

\url{http://developer.51cto.com/art/201112/305880.htm}

C++11是曾经被叫做C++0x,是对目前C++语言的扩展和修正,C++11不仅包含核心语言的新机能,而且扩展了C++的标准程序库(STL),并入了大部分的C++ Technical Report 1(TR1)程序库(数学的特殊函数除外)。

C++之前的标准有C++98和C++03,二者差别很小。C++14旨在作为C++11的一个小扩展,主要提供漏洞修复和小的改进。
例如,在C++11中,lambda函数的形式参数需要被声明为具体的类型。C++14放宽了这一要求,允许lambda函数的形式参数声明中使用类型说明符auto。

C++11包括大量的新特性:它现在支持Lambda表达式,对象类型自动推断,统一的初始化语法,委托构造函数,deleted和defaulted函数声明,nullptr,以及最重要的右值引用。


\subsection{自动类型推导}
\begin{lstlisting}[language=C++]
auto a; // 错误,auto是通过初始化表达式进行类型推导,如果没有初始化表达式,就无法确定a的类型  
auto d = 1.0;  
auto str = "Hello World";  
auto ch = 'A';  
auto func = less<int>();  
vector<int> iv;  
auto ite = iv.begin();  
auto p = new foo() // 对自定义类型进行类型推导 
\end{lstlisting}

\textbf{decltype}实际上有点像\textbf{auto}的反函数,auto可以让你声明一个变量,而decltype则可以从一个变量或表达式中得到类型,有实例如下:
\begin{lstlisting}[language=C++]
int x = 3;  
decltype(x) y = x;

template <typename Creator>  
auto processProduct(const Creator& creator) -> decltype(creator.makeObject()) {  
    auto val = creator.makeObject();  
    // do somthing with val  
} 
\end{lstlisting}

\subsection{nullptr}
nullptr是为了解决原来C++中NULL的二义性问题而引进的一种新的类型,对于一些重载函数,NULL的类型不确定是整型还是指针。
\begin{lstlisting}[language=C++]
void f(int); //1  
void f(char *);//2  
f(0); //C++03: which f is called?
f(nullptr) //C++11: unambiguous, calls 2
\end{lstlisting}

\subsection{序列for循环}
\begin{lstlisting}[language=C++]
map<string, int> m{{"a", 1}, {"b", 2}, {"c", 3}};  
for (auto p : m){  
    cout<<p.first<<" : "<<p.second<<endl;  
} 
\end{lstlisting}

\subsection{Lambda表达式}
lambda表达式类似Javascript中的闭包,它可以用于创建并定义匿名的函数对象,以简化编程工作。Lambda的语法如下:
\begin{verbatim}
[函数对象参数](操作符重载函数参数)->返回值类型{函数体}
\end{verbatim}


\subsection{变长参数的模板}
由于在C++11中引入了变长参数模板,所以发明了新的数据类型:tuple,tuple是一个N元组,可以传入1个, 2个甚至多个不同类型的数据:
\begin{lstlisting}[language=C++]
auto t1 = make_tuple(1, 2.0, "C++ 11");  
auto t2 = make_tuple(1, 2.0, "C++ 11", {1, 0, 2}); 
\end{lstlisting}
这样就避免了从前的pair中嵌套pair的丑陋做法,使得代码更加整洁。

\subsection{右值引用}
左值是一个指向某内存空间的表达式,并且我们可以用\&操作符获得该内存空间的地址。右值就是非左值的表达式。

在 C++03及之前的标准,临时对象(称为右值"R-values",位于赋值运算符之右)无法被改变,在 C 中亦同(且被视为无法和 const T\& 做出区分)。
如果一个表达式返回一个临时变量,则该表达式是右值。

C++11 增加一个新的非常数引用(reference)类型,称作右值引用(R-value reference),标记为T \&\&。普通的引用现在被称为左值引用。
右值引用所引用的临时对象可以在该临时对象被初始化之后做修改,这是为了允许move语义。

右值引用至少解决了这两个问题:实现move语义;完美转发(Perfect forwarding)。
右值引用允许函数在编译期根据参数是左值还是右值来建立分支。

\begin{lstlisting}[language=C++]
void foo(X& x); // 左值引用重载
void foo(X&& x); // 右值引用重载
X x;
X foobar();
foo(x);//参数是左值
foo(foobar());//参数是右值
\end{lstlisting}

右值引用类型既可以被当作左值也可以被当作右值,判断的标准是,如果它有名字,那就是左值,否则就是右值。
\begin{lstlisting}[language=C++]
void foo(X&& x)
{
   X anotherX = x; // x被当做左值,调用X(X const & rhs)
}
/**********************************/
X&& goo();
X x = goo(); // 调用X(X&& rhs),goo的返回值没有名字
/**********************************/
Base(Base const & rhs); // non-move semantics
Base(Base&& rhs); // move semantics
Derived(Derived const & rhs) 
   : Base(rhs)
{
   // Derived-specific stuff
}
Derived(Derived&& rhs) 
   : Base(rhs) // 错误:rhs是个左值
{
   // ...
}
Derived(Derived&& rhs) 
   : Base(std::move(rhs)) // good, calls Base(Base&& rhs)
{
   // Derived-specific stuff
}
\end{lstlisting}


\subsection{统一形的初始化}
在引入C++11之前,只有数组能使用初始化列表,其他容器想要使用初始化列表,只能用以下方法:
\begin{lstlisting}[language=C++]
int arr[3] = {1, 2, 3}  
vector<int> v(arr, arr + 3); 
\end{lstlisting}

在C++11中,我们可以使用以下语法来进行替换:
\begin{lstlisting}[language=C++]
int arr[3]{1, 2, 3};  
vector<int> iv{1, 2, 3};
vector vs<string>={ "first", "second", "third"};
map<int, string>{{1, "a"}, {2, "b"}};  
string str{"Hello World"}; 
\end{lstlisting}

\subsection{委托构造函数}
C++11允许构造函数调用其他构造函数,这种做法称作委托或转接(delegation)。
\begin{lstlisting}[language=C++]
class SomeType {
  int number;
  string name;
  SomeType( int i, string& s ) : number(i), name(s){}
public:
  SomeType( )           : SomeType( 0, "invalid" ){}
  SomeType( int i )     : SomeType( i, "guest" ){}
  SomeType( string& s ) : SomeType( 1, s ){ PostInit(); }
};
\end{lstlisting}

\subsection{使用或禁用对象的默认函数}
在传统C++中,若使用者没有提供, 则编译器会自动为对象生成默认构造函数(default constructor)、 复制构造函数(copy constructor),赋值运算符(copy assignment operator operator=) 以及析构函数(destructor)。
问题在于原先的c++无法精确地控制这些默认函数的生成。
比方说,要让类型不能被拷贝,必须将复制构造函数与赋值运算符声明为private,并不去定义它们。
尝试使用这些未定义的函数会导致编译期或链接期的错误。
但这种手法并不是一个理想的解决方案。
此外,编译器产生的默认构造函数与使用者定义的构造函数无法同时存在。
\textbf{若用户定义了任何构造函数,编译器便不会生成默认构造函数; 但有时同时带有上述两者提供的构造函数也是很有用的}。


C++11 允许显式地表明采用或拒用编译器提供的内置函数。例如要求类型带有默认构造函数,可以用以下的语法:
\begin{lstlisting}[language=C++]
struct A   
{   
	A()=default;
    virtual ~A()=default;
};
\end{lstlisting}
Deleted函数对防止对象复制很有用,回想一下C++自动为类声明一个副本构造函数和一个赋值操作符,要禁用复制,声明这两个特殊的成员函数=delete即可:
\begin{lstlisting}[language=C++]
struct NonCopyable
{
  NonCopyable & operator=(const NonCopyable&) = delete;
  NonCopyable(const NonCopyable&) = delete;
  NonCopyable() = default;
};
\end{lstlisting}
禁止类型以operator new配置内存:
\begin{lstlisting}[language=C++]
struct NonNewable
{
  void *operator new(std::size_t) = delete;
};
\end{lstlisting}

\subsection{线程库}
站在程序员的角度来看,C++11最重要的新功能毫无疑问是并行操作,C++11拥有一个代表执行线程的线程类,在并行环境中用于同步,async()函数模板启动并行任务,为线程独特的数据声明\verb$thread_local$存储类型。如果你想找C++11线程库的快速教程,请阅读Anthony William的“C++0x中更简单的多线程”。

\subsection{常量表达式}
C++11新标准规定,允许将变量声明为constexpr类型以便由编译器来验证变量的值是否是一个常量表达式。声明为constexpr的变量一定是一个常量,而且必须用常量表达式初始化:
\begin{lstlisting}[language=C++]
constexpr int GetFive() {return 5;}
int some_value[GetFive() + 5];
constexpr int mf = 20; // 20是常量表达式  
constexpr int limit = mf + 1;// mf + 1是常量表达式  
constexpr int sz = size();//当size是一个constexpr函数时才可编译
\end{lstlisting}

\subsection{新的智能指针类}
C++98只定义了一个智能指针类\verb$auto_ptr$,它现在已经被废弃了,C++11引入了新的智能指针类\verb$shared_ptr$和最近添加的\verb$unique_ptr$,两者都兼容其它标准库组件,因此你可以在标准容器内安全保存这些智能指针,并使用标准算法操作它们。

\subsection{新的算法}
C++11标准库定义了新的算法: \verb$all_of(),any_of(),none_of(),copy_n(),iota()$操作:
\begin{lstlisting}[language=C++]
#include <algorithm>      
//are all of the elements positive?  
all_of(first, first+n, ispositive()); //false  
//is there at least one positive element?   
any_of(first, first+n, ispositive());//true    
// are none of the elements positive?    
none_of(first, first+n, ispositive()); //false 
int source[5]={0,12,34,50,80};  
int target[5];  
//copy 5 elements from source to target  
copy_n(source,5,target);
\end{lstlisting}

算法iota()创建了一个值顺序递增的范围,好像分配一个初始值给*first,然后使用前缀++使值递增,在下面的代码中,iota()分配连续值{10,11,12,13,14}给数组arr,并将{‘a’,’b’,’c’}分配给char数组c。
\begin{lstlisting}[language=C++]
include <numeric> 
int a[5]={0};  
char c[3]={0};  
iota(a, a+5, 10); //changes a to {10,11,12,13,14}  
iota(c, c+3, 'a'); //{'a','b','c'}
\end{lstlisting}

C++11仍然缺乏一些有用的库,如XML API,套接字,GUI,反射以及前面提到的一个合适的自动垃圾回收器,但C++11的确也带来了许多新特性,让C++变得更加安全,高效,易学易用。

\subsection{Boost}
除了加入C++11的功能之外,boost还包含其他的实用库,如crc,uuid,有理数,环形数组,内存池,system,filesystem,python支持等。


