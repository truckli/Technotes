%!Mode:: "TeX:UTF-8"
\section{SGI STL源码学习笔记}

vector动态增长是乘以2的关系,未必是2的幂。
c[idx]没有边界检查,而c.at(idx)方式访问元素会抛出range\_error。

STL三大组件:容器,算法,迭代器。SGI STL又加上仿函数,适配器,配置器。

STL序列式容器:vector,deque,list, forward\_list,priority\_queue,stack适配器,queue适配器(SGI定义了非标准的slist,以算法形式存在的heap)。

STL关联式容器:set,map,multiset,multimap(SGI定义了非标准的hashtable,hashset,hashmap等,以算法形式存在的RBTree)。

有了hashmap(unordered\_map),为什么还要保留基于红黑树的map?
第一,平衡二叉树的查找时间未必比常数查找时间慢多少(100W记录只需20次比较),而哈希表未必有多快(设计不当,以及哈希函数耗时);
第二,哈希表占用大量的内存空间,是以空间换时间的方法。

insert操作执行前插并范围刚插入的值,erase则返回被删除的下一个值。以下代码删除偶数,复制奇数:
\begin{lstlisting}[language=C++] 
vector<int> vi = {0,1,2,3,4,5,6,7,8,9};
auto iter = vi.begin(); // call begin , not cbegin because we’re changing vi
while (iter != vi.end()) {
	if (*iter % 2) {
		iter = vi.insert(iter, *iter); // duplicate the current element
		iter += 2; // advance past this element and the one inserted before it
	} else
		iter = vi.erase(iter);// remove even elements	
	// don’t advance the iterator; iter denotes the element after the one we erased
}
\end{lstlisting}




