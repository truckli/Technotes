%!Mode:: "TeX:UTF-8"
\section{对象大小与对齐}

\subsection{C++常见类型的大小}
\begin{itemize}
    \item string:4 (注:类的大小依具体实现而异)
    \item istream:144
    \item vector:13
    \item 空类:1 (注:空类作为派生类的基类部分时可以大小为零)
    \item 多重继承空类:1
    \item 虚继承空类:4. 因为涉及虚表(\cite{pibible}P59)
\end{itemize}

\subsection{sizeof(字符串)}
字符数组长度(使用sizeof()求取)等于字符串长度加1, 因为末尾的零也占一位。
而指向字符串的char*变量的长度则为指针长度(比如4)。
这是字符串与字符数组的区别之一。二者都可以用静态字符串变量初始化。
静态字符串也可用sizeof()求取长度。


\begin{lstlisting}[language=C++]

#include <iostream>
using namespace std;

size_t string_len(const char str[])
{
    return sizeof(str);
}

size_t string8_len(const char str[8])
{
    return sizeof(str);
}

int main(int argc, char* argv[])
{
#define STATIC_STRING "123456789"
    const char sc[] = STATIC_STRING;
    const char *dc = STATIC_STRING;

    cout << sizeof(STATIC_STRING) << endl; //10
    cout << sizeof(sc) << endl; //10
    cout << sizeof(dc) << endl; //4
    cout << string_len(sc) << endl; //4
    cout << string8_len(sc) << endl; //4

    return 0;
}                                           

\end{lstlisting}


\subsection{sizeof(C二维数组)}
\begin{lstlisting}[language=C++]
#include <stdlib.h>
#include <stdint.h>
#include <stdio.h>

int a[3][5] = {
    1, 2, 3, 4, 5, 
    6, 7, 8, 9, 10, 
    11, 12, 13, 14, 15
};

int main(int argc, char* argv[])
{
    printf("sizeof(a) = %d\n", sizeof(a));
    printf("sizeof(*a) = %d\n", sizeof(*a));
    printf("sizeof(**a) = %d\n", sizeof(**a));
    printf("**(a+1) = %d\n", **(a+1));
    printf("**(a+2) = %d\n", **(a+2));
    printf("**(a+3) = %d\n", **(a+3));
    printf("**(a+4) = %d\n", **(a+4));
    return 0;
}

/*
输出结果:
sizeof(a) = 60
sizeof(*a) = 20
sizeof(**a) = 4
**(a+1) = 6
**(a+2) = 11
**(a+3) = 0
**(a+4) = 0
*/                        
\end{lstlisting}

a的类型是什么?
如果\verb$int daytab[2][13]$作为函数参数,那么函数声明应形如
\begin{lstlisting}[language=C++]
void f(int daytab[2][13]) { ... }
vod f(int daytab[][13]) { ... }
void f(int (*daytab)[13]) { ... }          
\end{lstlisting}
第三种声明解释了a+1实际的地址偏移并非sizeof(a), 而是20。

\subsection{成员地址偏移}
stddef.h提供以下宏可用于求取类型为t的对象的成员变量m相对于对象起始字节的地址偏移\cite{pibible}:
\begin{lstlisting}[language=C++]
 #define offsetof(t,m) ((size_t)&((t *)0)->m)
\end{lstlisting}
将地址零强制转换为所关心的类型,则各字段的地址等于其相对于0的偏移。只要不引用各字段,就不会发生段错误。


类似地,内核代码linux/kernel.h中提供的container\_of宏定义将type类型的成员变量member的指针ptr转为结构体地址。
\begin{lstlisting}[language=C++]
  #define container_of(ptr, type, member) ({                      \
        const typeof( ((type *)0)->member ) *__mptr = (ptr);    \
        (type *)( (char *)__mptr - offsetof(type,member) );})
\end{lstlisting}
其中,typeof是gcc的C语言扩展保留字,用于声明变量类型。





\subsection{C变量对齐规则}
内存中的数据对齐准则,指数据所在内存地址必须是数据长度的整数倍(\cite{pibible}P50)。当在栈上分配变量空间时,这一点由编译器保证,且不同的编译器会输出不同的结果。当在堆上分配大片空间,由程序员故意在非变量长度整数倍的地址上存放一变量,这一点可能无法保证。

对于成员变量对齐规则,有
参\cite{pibible}P49。
\begin{enumerate}
    \item 如果成员均小于处理器位数,以最长成员为对齐单位
    \item 如果存在成员达到或超过处理器位数,则以处理器位数为对齐单位
    \item 类型相同的连续元素位于连续的空间内,如同数组(只要第一个元素满足内存中的数据对齐规则,其后元素必然满足对齐规则)。
    \item 对象的大小为对齐单位的整数倍
\end{enumerate}

VC++中加上\#pragma pack(n)的设定能改变默认的数据对齐长度,n需为2的幂,如果没有参数n,相当于取消上一次设定。

使用g++测试,总结出如下规律:
    成员变量如果是结构体,则分解开考虑,相当于所有成员的长度只能是1,2,4,8。如果32位机上出现了4字节或8字节的变量,则结构长度为4字节的倍数,而每个成员均须保证自身是对齐的(假设结构体起始于4字节倍数地址)。如果最长的成员长2字节,则结构长度为2字节的倍数。如果是由单字节变量聚合成,则无需考虑对齐。

还有一种总结:
\begin{enumerate}
\item 原则1、普通数据成员对齐规则:第一个数据成员放在offset为0的地方,以后每个数据成员存储的起始位置要从该成员大小的整数倍开始,或对齐单位整数倍,取较小者
\item 原则2、结构体成员对齐规则:如果一个结构里有某些结构体成员,则该结构体成员要从其内部最大元素大小的整数倍地址开始存储。(struct a里存有struct b,b里有char,int,double等元素,那b应该从8的整数倍开始存储。)
\item 原则3、结构体大小对齐规则:结构体大小也就是sizeof的结果,必须是其内部成员中最大的对齐参数的整数倍
\end{enumerate}

另外,在栈上分配结构体空间,是如何确定起始地址的?可以猜想,如果确保结构体起始于4字节倍数的地址,则可以保证每个成员都是对齐的。但实际分配原则应比此宽松。

以下为一些测试例:
\begin{lstlisting}[language=C++]
#include <stdlib.h>
#include <stdint.h>
#include <stdio.h>

/*#pragma pack(2)*/
struct char_only{
    char f1;
};

struct char3{
    char f1:3;
};

struct int3{
    int f1:3;
};


struct int3_int2{
    int f1:3;
    int f2:2;
};

struct int3_int2_char{
    int f1:3;
    int f2:2;
    char f3;
};

struct int3_char {
    int f1:3;
    char f2;
};

struct int24_char {
    int f1:24;
    char f2;
};

struct int25_char {
    int f1:25;
    char f2;
};

struct int_char {
    int f1;
    char f2;
};

struct char_int{
    char f1;
    int f2;
};

struct char_short_int{
    char f1;
    short f2;
    int f3;
};

struct char_int_short{
    char f1;
    int f3;
    short f2;
};

struct int_double_float{
    int f1;
    double f3;
    float f2;
};


struct float_struct_int_double_float{
    float f1;
    struct int_double_float f2;
};

struct int_double_char{
    int f1;
    double f3;
    char f2;
};



struct char_short_int_double{
    char f1;
    short f2;
    int f3;
    double f4;
};

struct char_short_char_double{
    char f1;//2
    short f2;//2
    char f3;//4
    double f4;//8
};

struct char_char_double{
    char f1;//1
    char f3;//3
    double f4;//8
};

#define PRINT(x) printf("%s = %d\n", #x, x)

int main(int argc, char* argv[])
{
    PRINT(sizeof(int));
    PRINT(sizeof(double));
    PRINT(sizeof(struct char_only));
    PRINT(sizeof(struct char3));
    PRINT(sizeof(struct int3));
    PRINT(sizeof(struct int3_int2));
    PRINT(sizeof(struct int3_int2_char));
    PRINT(sizeof(struct int3_char));
    PRINT(sizeof(struct int24_char));
    PRINT(sizeof(struct int25_char));
    PRINT(sizeof(struct int_char));
    PRINT(sizeof(struct int_double_float));
    PRINT(sizeof(struct float_struct_int_double_float));
    PRINT(sizeof(struct int_double_char));
    PRINT(sizeof(struct char_int));
    PRINT(sizeof(struct char_short_int));
    PRINT(sizeof(struct char_int_short));
    PRINT(sizeof(struct char_short_int_double));
    PRINT(sizeof(struct char_short_char_double));
    PRINT(sizeof(struct char_char_double));
    return 0;
}
\end{lstlisting}

输出结果:
\begin{verbatim}
sizeof(int) = 4
sizeof(double) = 8
sizeof(struct char_only) = 1
sizeof(struct char3) = 1
sizeof(struct int3) = 4
sizeof(struct int3_int2) = 4
sizeof(struct int3_int2_char) = 4
sizeof(struct int3_char) = 4
sizeof(struct int24_char) = 4
sizeof(struct int25_char) = 8
sizeof(struct int_char) = 8
sizeof(struct int_double_float) = 16
sizeof(struct float_struct_int_double_float) = 20
sizeof(struct int_double_char) = 16
sizeof(struct char_int) = 8
sizeof(struct char_short_int) = 8
sizeof(struct char_int_short) = 12
sizeof(struct char_short_int_double) = 16
sizeof(struct char_short_char_double) = 16
sizeof(struct char_char_double) = 12
\end{verbatim}
如果使用了\#pragma pack(2), 则输出结果:
\begin{verbatim}
sizeof(int) = 4
sizeof(double) = 8
sizeof(struct char_only) = 1
sizeof(struct char3) = 1
sizeof(struct int3) = 2
sizeof(struct int3_int2) = 2
sizeof(struct int3_int2_char) = 2
sizeof(struct int3_char) = 2
sizeof(struct int24_char) = 4
sizeof(struct int25_char) = 6
sizeof(struct int_char) = 6
sizeof(struct int_double_float) = 16
sizeof(struct float_struct_int_double_float) = 20
sizeof(struct int_double_char) = 14
sizeof(struct char_int) = 6
sizeof(struct char_short_int) = 8
sizeof(struct char_int_short) = 8
sizeof(struct char_short_int_double) = 16
sizeof(struct char_short_char_double) = 14
sizeof(struct char_char_double) = 10

\end{verbatim}


使用位域的主要目的是压缩存储,其大致规则为:  \\
 1) 如果相邻位域字段的类型相同,且其位宽之和小于类型的sizeof大小,则后面的字段将紧邻前一个字段存储,直到不能容纳为止;  \\
 2) 如果相邻位域字段的类型相同,但其位宽之和大于类型的sizeof大小,则后面的字段将从新的存储单元开始,其偏移量为其类型大小的整数倍;  \\
 3) 如果相邻的位域字段的类型不同,则各编译器的具体实现有差异,VC6采取不压缩方式,Dev-C++采取压缩方式;  \\
 4) 如果位域字段之间穿插着非位域字段,则不进行压缩;  \\
 5) 整个结构体的总大小为最宽基本类型成员大小的整数倍。\\

结构,联合,或者类的数据成员,第一个放在偏移为0的地方,以后每个数据成员的对齐,按照\#pragma pack指定的数值和这个数据成员自身长度中,比较小的那个进行。
 也就是说,当\#pragma pack的值等于或超过所有数据成员长度的时候,这个值的大小将不产生任何效果。
 而结构整体的对齐,则按照结构体中最大的数据成员 和 \#pragma pack指定值 之间,较小的那个进行。

\#pragma pack(1)表示完全没有对齐考虑,紧凑布局。

当数据定义中出现\__declspec(align())时,指定类型的对齐长度还要用自身长度和这里指定的数值比较,然后取其中较大的。
最终类/结构的对齐长度也需要和这个数值比较,然后取其中较大的。
可以这样理解,\__declspec(align()) 和 \#pragma pack是一对兄弟,前者规定了对齐的最小值,后者规定了对齐的最大值,两者同时出现时,前者拥有更高的优先级。


\begin{lstlisting}[language=C++]
struct test{
    int f1;
    double f2;
    int f3[0];
};
\end{lstlisting}
上述f3是不占内存的,但可作指针用,只是一个指向结尾的指针。

\begin{lstlisting}[language=C++]
struct char1{
    char f3[1];
};
\end{lstlisting}
该结构体占用的空间为1,即f3占用1字节内存,但可作指针用。


\subsection{C++多继承下的内存布局}

\begin{lstlisting}[language=C++]
#include <cassert>
#include <cstdlib>
#include <climits>
#include <cstring>
#include <stack>
#include <vector>
#include <iostream>
#include <algorithm>
#include <iterator>  
using namespace std;

#define COUT(x) cout << #x << " = " << x << endl

struct Base1{
    int b1; 
    virtual void f() {}
};

struct Base2{
    int b2; 
    virtual void g() {}
};

struct Derived: public Base1, public Base2{
    int d1; 
    int d2; 
    virtual void h() {}
    virtual void f() {}
};


int main(void) {
    Derived d;
    Base1 *bp1 = &d; 
    Base2 *bp2 = &d; 
    COUT(bp1);
    COUT(bp2);
    COUT(&d);
    COUT(&d.d2);
    return 0;
}

/*
输出结果:
bp1 = 0xbfc4b038
bp2 = 0xbfc4b040
&d = 0xbfc4b038
&d.d2 = 0xbfc4b04c
*/      
\end{lstlisting}


\subsection{Structure Packing}

\href{http://www.catb.org/esr/structure-packing/}{The Lost Art of C Structure Packing}

\href{https://github.com/ludx/The-Lost-Art-of-C-Structure-Packing/blob/master/README.md}{The Lost Art of C Structure Packing(Chinese Version)}



\begin{lstlisting}[language=C++]
#pragma pack(push) //save state
#pragma pack(4)//4-byte alignment
struct test
{
char m1;
double m4;
int m3;
};
#pragma pack(pop)//restore state

\end{lstlisting}

Another example:
\begin{lstlisting}[language=C++]
#pragma pack(push, 1) // exact fit - no padding
struct MyStruct
{
  char b; 
  int a; 
  int array[2];
};
#pragma pack(pop) //back to whatever the previous packing mode was

\end{lstlisting}
Without the pragma directive, the size of the structure is 16 bytes - with the packing of 1 - the size is 13 bytes

Gcc's attribute \textbf{aligned} specifies a minimum alignment for the variable
or structure field, measured in bytes. For example, the declaration: 
\begin{lstlisting}[language=C++]
int x __attribute__ ((aligned (16))) = 0;
\end{lstlisting}
causes the compiler to allocate the global variable x on a 16-byte boundary.

This attribute also specifies a minimum alignment (in bytes) for variables of
the specified type(structure or union type). For example, the declarations:
\begin{lstlisting}[language=C++]
struct S { short f[3]; } __attribute__ ((aligned (8)));
typedef int more_aligned_int __attribute__ ((aligned (8)));
\end{lstlisting}
force the compiler to insure (as far as it can) that each variable whose type is struct S or \verb|more_aligned_int| will be allocated and aligned at least on a 8-byte boundary.

The \textbf{packed} attribute specifies that a variable or structure field
should have the smallest possible alignment—one byte for a variable, and one bit for a field, unless you specify a larger value with the aligned attribute.
Here is a structure in which the field x is packed, so that it immediately follows a:

\begin{lstlisting}[language=C++]
struct foo {
    char a;
    int x[2] __attribute__ ((packed));
};
\end{lstlisting}




