%!Mode:: "TeX:UTF-8"
\section{const,static与volatile}

\subsection{const用途}
\begin{itemize}
    \item 定义const常量
    \item 修饰函数参数和返回值
    \item 修饰函数体(C++ only),定义恒态函数,不改动成员变量,除非成员变量使用了mutable修饰符
\end{itemize}

logical constness举例:
\begin{lstlisting}[language=C++]
class CTextBlock {
public:
  ...
  std::size_t length() const;
private:
  char *pText;
  mutable std::size_t textLength;         // these data members may
  mutable bool lengthIsValid;             // always be modified, even in
};                                        // const member functions

std::size_t CTextBlock::length() const
{
  if (!lengthIsValid) {
    textLength = std::strlen(pText);      // now fine
    lengthIsValid = true;                 // also fine
  }

  return textLength;
}
\end{lstlisting}


\subsection{const在C与C++的区别}
\begin{itemize}
    \item C默认const是外部连接的,而C++则默认是内部连接的。C++不允许\verb|const int a;|这种写法,而C中相当于声明语句\verb|extern const int a;|
    \item C中const量总是占用内存,且全局可见,不能当作编译期间的常量(如定义数组长度)。因此C中const用途非常有限。
\end{itemize}

一般而言,定义引用(reference)须用相同类型的变量作初始化,但const引用则可以用一般右值进行初始化。
\begin{lstlisting}[language=C++]
    int i = 3;
    int &t = i;
    const int &r = i + 6;
\end{lstlisting}
这是因为一般的引用可看做一个地址,而const引用则可以创建一个占用内存的对象。

\subsection{用const构造类成员数组}
在类中可以用constexpr声明数组。用const代替constexpr也可以。
\begin{lstlisting}[language=C++]
static constexpr int period = 30;// period is a constant expression
double daily_tbl[period];
\end{lstlisting}

\subsection{static用途}
\begin{itemize}
    \item 静态全局变量/函数
    \item 静态局部变量
    \item 类成员变量/函数
\end{itemize}


static关键词只用于声明,不用于定义;static成员(变量或函数)用通过类名和对象均可访问,在派生类中亦可访问(public, protected)。
non-const static成员变量不能在类体中初始化,必须在外部定义时初始化;const static成员变量可在类体中初始化,当常量表达式使用,但仍需在外部定义。
static成员变量的类型可以是所属类的类型。

\subsection{volatile修饰符}
 volatile 关键字是一种类型修饰符,用它声明的类型变量表示可以被某些编译器未知的因素更改,比如:操作系统、硬件或者其它线程等。
 遇到这个关键字声明的变量,编译器应该对访问该变量的代码就不再进行优化,从而可以提供对特殊地址的稳定访问。
 当两个线程都要用到某一个变量且该变量的值会被改变时,应该用volatile声明。
 然而volatile在C/C++中的语义其实很模糊,有的编译器未必会理会这一提示。参考\ref{subsec:MemBarrier}。

\begin{lstlisting}[language=C++]
class MyThread : public Thread {
	public:
		virtual void run() {
			while (!_stopped) {
			  //do something
			}
	    }
		void stop() {
			 _stopped = true;
		}
	private:
		bool volatile _stopped;
 };

MyThread thread;
thread.run();
\end{lstlisting}

\url{https://www.kernel.org/doc/Documentation/volatile-considered-harmful.txt}认为volatile应该被栅栏操作替代。
volatile没有实际意义,只有降低性能的危害。如果有人认为volatile非用不可,很有可能意味着其代码有bug。



