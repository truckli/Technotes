%!Mode:: "TeX:UTF-8"
%!Mode:: "TeX:UTF-8"

\section{针对C语言的轮子}
Glib库为Gnome项目的基础,可独立用于其他项目做轮子。
Glib项目的test目录可作为demo以供学习。

\subsection{boost}

在Linux下编译:
\begin{verbatim}
./bootstrap.sh
./b2
\end{verbatim}

在mingw环境下编译:
\begin{verbatim}
./bootstrap.bat mingw
./b2.exe --toolset=gcc --build-type=complete
\end{verbatim}


\subsection{日志调试}

Glib的功能中包含了日志调试,但是不甚便利。
默认情况下,debug和info级别的日志默认是不打开的,其他级别的日志都是打开的。
可以通过设置环境变量\verb$G_MESSAGES\_DEBUG=all$打开所有日志。
如果需要对级别进行精细调控,需要自定义日志处理handler,自己的handler可以调用默认的\verb$g_log_default_handler()$。
glib并未提供定义好的给定阈值控制法。

GitHub项目\href{https://github.com/HardySimpson/zlog}{zlog}提供了较好的日志功能。
快速配置时,将配置文件myzlog.conf,zlog.h头文件和静态库文件置入当前项目路径。编译时要同时链接pthread库。
参考该项目的README.md文件。
\begin{verbatim}
#include <stdio.h>
    #include "zlog.h"

    int main()
    {

       if (dzlog_init("myzlog.conf", "my_cat")) {
          printf("init failed\n");
          return -1;
      }

      dzlog_info("%d: %d, %d", high_freq_bcounts, ok, bad);
      zlog_fini();
      return 0;
  }

\end{verbatim}
