%!Mode:: "TeX:UTF-8"

\section{errno与错误}
\subsection{errno}
在C++中errno被定义为宏,最终展开成为一个int型左值。
在C语言中,errno可以是一个有外部链接的整型变量。
程序启动时errno被初始化为0,任意函数均可改变其值,因此在调用需要查错的函数前应先将errno设置为0。


支持多线程的库应该在每个线程定义errno,对于C11和C++11兼容的库这是基本要求。

stdio.h定义了perror检查errno变量,打印出对应的描述性文字,可以配置打印前缀。
\begin{lstlisting}[language=C]
void perror(const char *str);
\end{lstlisting}

string.h定义了strerror将整型错误码转为描述性文字,其返回值为静态分配的字符串,不应被修改,多次调用strerror可能会覆盖其内容。
\begin{lstlisting}[language=C]
char *strerror(int errnum);
\end{lstlisting}

以下例子说明了perror的用法:
\begin{lstlisting}[language=C++]
/* perror example */
#include <stdio.h>

int main ()
{
  FILE * pFile;
  pFile=fopen ("unexist.ent","rb");
  if (pFile==NULL)
    perror ("The following error occurred");
  else
    fclose (pFile);
  return 0;
}
\end{lstlisting}

以下例子说明了strerror的用法:
\begin{lstlisting}[language=C++]
/* strerror example : error list */
#include <stdio.h>
#include <string.h>
#include <errno.h>

int main ()
{
  FILE * pFile;
  pFile = fopen ("unexist.ent","r");
  if (pFile == NULL)
    printf ("Error opening file unexist.ent: %s\n",strerror(errno));
  return 0;
}
\end{lstlisting} 




