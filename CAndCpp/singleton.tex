%!Mode:: "TeX:UTF-8"
\section{单例模式的实现问题}

\subsection{使用静态成员的实现}
\begin{lstlisting}[language=C++] 
class CMySingleton
{
public:
  static CMySingleton& Instance()
  {
    static CMySingleton singleton;
    return singleton;
  }

// Other non-static member functions
private:
  CMySingleton() {}                                  // Private constructor
  ~CMySingleton() {}
  CMySingleton(const CMySingleton&);                 // Prevent copy-construction
  CMySingleton& operator=(const CMySingleton&);      // Prevent assignment
};
\end{lstlisting}

这种实现方式的缺陷是我们无法控制多个单例的构造析构顺序。
当多个单例对象在析构过程中相互依赖时,可能会导致错误。
\begin{lstlisting}[language=C++] 
struct A
{
  A() { B::Instance(); C::Instance().call(); }
};

struct B
{
  ~B() { C::Instance().call(); }
  static B& Instance() { static B MI; return MI; }
};

struct C
{
  static C& Instance() { static C MI; return MI; }
  void call() {}
};

A globalA;
\end{lstlisting}
A的构造函数会导致B和C的单例被依次构造。
静态对象的析构顺序与其构造顺序相反,因此C先被析构。
B在析构时会调用C的方法,导致未定义行为。

\subsection{使用指针的实现}
使用不死鸟模式(Phoenix Singleton)解决多个单例相互依赖的问题。

\begin{lstlisting}[language=C++] 
// in s.hpp
class S
{
public:
  static S& Instance(); // already defined
private:
  static void CleanUp();
  S(); // later, because that's where the work takes place
  ~S() { /* anything ? */ }
  // not copyable
  S(S const&);
  S& operator=(S const&);
  static S* MInstance;
};

// in s.cpp
S* S::MInstance = 0;
S& S::Instance() { if (MInstance == 0) MInstance = new S(); return *MInstance; }
S::S() { atexit(&CleanUp); }
S::CleanUp() { delete MInstance; MInstance = 0; } // Note the = 0 bit!!!

\end{lstlisting}



\subsection{线程安全单例的实现}
一种广为人知的做法是使用所谓的Double-Checked Locking:

\begin{lstlisting}[language=C++]                      

 class Singleton
{
private:
    static Singleton *volatile m_instance;
    Singleton(){}
public:
    static Singleton *getInstance();
};

Singleton* volatile Singleton::m_instance = 0;

Singleton* Singleton::getInstance()
{
    if(NULL == m_instance)
    {
        Lock();//借用其它类来实现,如boost
        if(NULL == m_instance)
        {
            m_instance = new Singleton;
        }
        UnLock();
    }
    return m_instance;
} 
\end{lstlisting}

Double-Checked Locking机制不是一个完美的解决方案,
首先,编译器可能会乱序优化,将Singleton对象的初始化调整到m\_instance赋值之后,使得某个进程得到的m\_instance可能只是未完成构造的裸内存。
其次,在多处理器(multiprocessors)的情况下,超线程技术必然会混合执行指令,指令的执行顺序更无法保障。
Java会将整个getInstance标为Syncronized,除此别无他法实现线程安全。
C++也采取类似做法,引入相关同步对象(synchronization object):
\begin{lstlisting}[language=C++]                      
1 #ifndef SINGLETON_H
 2 #define SINGLETON_H
 3 
 4 #include "synobj.h"
 5 
 6 class Singleton {
 7 public:
 8     static Singleton& Instance() { // Unique point of access
 9         Lock lock(_mutex);
10         if (0 == _instance) {
11             _instance = new Singleton();
12             atexit(Destroy); // Register Destroy function
13         }
14         return *_instance;
15     }
16     void DoSomething(){}
17 private:
18     static void Destroy() { // Destroy the only instance
19         if ( _instance != 0 ) {
20             delete _instance;
21             _instance = 0;
22         }
23     }
24     Singleton(){} // Prevent clients from creating a new Singleton
25     ~Singleton(){} // Prevent clients from deleting a Singleton
26     Singleton(const Singleton&); // Prevent clients from copying a Singleton
27     Singleton& operator=(const Singleton&);
28 private:
29     static Mutex _mutex;
30     static Singleton *_instance; // The one and only instance
31 };
32 
33 #endif/*SINGLETON_H*/
\end{lstlisting}



\subsection{Monoid模式}
Monoid模式是Flyweight模式的退化,也可看做是在单例模式上应用代理模式。
思想是所有实例共享相同的状态。
\begin{lstlisting}[language=C++] 
class Monoid
{
public:
  void foo() { if (State* i = Instance()) i->foo(); }
  void bar() { if (State* i = Instance()) i->bar(); }
private:
  struct State {};
  static State* Instance();
  static void CleanUp();
  static bool MDestroyed;
  static State* MInstance;
};

// .cpp
bool Monoid::MDestroyed = false;
State* Monoid::MInstance = 0;

State* Monoid::Instance()
{
  if (!MDestroyed && !MInstance)
  {
    MInstance = new State();
    atexit(&CleanUp);
  }
  return MInstance;
}

void Monoid::CleanUp()
{
  delete MInstance;
  MInstance = 0;
  MDestroyed = true;
}
\end{lstlisting}


