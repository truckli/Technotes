%!Mode:: "TeX:UTF-8"
\section{NoSQL}


NoSQL有时也称作Not Only SQL的缩写,是对不同于传统的关联式数据库的数据库管理系统的统称。
两者存在许多显著的不同点,其中最重要的是NoSQL不使用SQL作为查询语言。
其数据存储可以不需要固定的表格模式,也经常会避免使用SQL的JOIN操作,一般有水平可扩展性的特征。
NoSQL的实现具有二个特征:使用硬盘,或者把随机存储器作存储载体。
少数NoSQL系统部署了分布式结构,通常使用分散式杂凑表(DHT)将数据以冗余方式保存在多台服务器上。依此,扩充系统时候添加服务器更容易,并且扩大了对服务器失效的承受能程度。


\subsection{BigTable}
BigTable是Google设计的分布式数据存储系统,用来处理海量的数据的一种非关系型的数据库。

BigTable是非关系的数据库,是一个稀疏的、分布式的、持久化存储的多维度排序Map。Bigtable的设计目的是可靠的处理PB级别的数据,并且能够部署到上千台机器上。Bigtable已经实现了下面的几个目标:适用性广泛、可扩展、高性能和高可用性。Bigtable已经在超过60个Google的产品和项目上得到了应用,包括 Google Analytics、GoogleFinance、Orkut、Personalized Search、Writely和GoogleEarth。这些产品对Bigtable提出了迥异的需求,有的需要高吞吐量的批处理,有的则需要及时响应,快速返回数据给最终用户。它们使用的Bigtable集群的配置也有很大的差异,有的集群只有几台服务器,而有的则需要上千台服务器、存储几百TB的数据。

在很多方面,Bigtable和数据库很类似:它使用了很多数据库的实现策略。并行数据库和内存数据库已经具备可扩展性和高性能,但是Bigtable提供了一个和这些系统完全不同的接口。Bigtable不支持完整的关系数据模型;与之相反,Bigtable为客户提供了简单的数据模型,利用这个模型,客户可以动态控制数据的分布和格式(alex注:也就是对BigTable而言,数据是没有格式的,用数据库领域的术语说,就是数据没有Schema,用户自己去定义Schema),用户也可以自己推测(alex注:reasonabout)底层存储数据的位置相关性(alex注:位置相关性可以这样理解,比如树状结构,具有相同前缀的数据的存放位置接近。在读取的时候,可以把这些数据一次读取出来)。数据的下标是行和列的名字,名字可以是任意的字符串。Bigtable将存储的数据都视为字符串,但是Bigtable本身不去解析这些字符串,客户程序通常会在把各种结构化或者半结构化的数据串行化到这些字符串里。通过仔细选择数据的模式,客户可以控制数据的位置相关性。最后,可以通过BigTable的模式参数来控制数据是存放在内存中、还是硬盘上。

\subsection{HBase}

HBase是一个开源的非关系型分布式数据库(NoSQL),它参考了谷歌的BigTable建模,实现的编程语言为Java。
它是Apache软件基金会Hadoop项目的一部分,运行于HDFS文件系统之上,为Hadoop提供类似于BigTable规模的服务。HBase在列上实现了BigTable论文提到的压缩算法、内存操作和布隆过滤器。

HBase – Hadoop Database,是一个高可靠性、高性能、面向列、可伸缩的分布式存储系统,利用HBase技术可在廉价PC Server上搭建起大规模结构化存储集群。
HBase是Google Bigtable的开源实现,类似Google Bigtable利用GFS作为其文件存储系统,HBase利用Hadoop HDFS作为其文件存储系统;Google运行MapReduce来处理Bigtable中的海量数据,HBase同样利用Hadoop MapReduce来处理HBase中的海量数据;Google Bigtable利用 Chubby作为协同服务,HBase利用Zookeeper作为对应。
  
  
\subsection{mongodb}

MongoDB是一个高性能,开源,无模式的文档型数据库,是当前NoSql数据库中比较热门的一种。它在许多场景下可用于替代传统的关系型数据库或键/值存储方式。Mongo使用C++开发。

Mongo DB很好的实现了面向对象的思想(OO思想),在Mongo DB中 每一条记录都是一个Document对象。Mongo DB最大的优势在于所有的数据持久操作都无需开发人员手动编写SQL语句,直接调用方法就可以轻松的实现CRUD操作。

NoSQL数据库与传统的关系型数据库相比,它具有操作简单、完全免费、源码公开、随时下载等特点,并可以用于各种商业目的。
这使NoSQL产品广泛应用于各种大型门户网站和专业网站,大大降低了运营成本。
2010年,随着互联网Web2.0网站的兴起,NoSQL在国内掀起一阵热潮,其中风头最劲的莫过于MongoDB了。
越来越多的业界公司已经将MongoDB投入实际的生产环境,很多创业团队也将MongoDB作为自己的首选数据库,创造出非常之多的移动互联网应用。

MongoDB的文档模型自由灵活,可以让你在开发过程中畅顺无比。对于大数据量、高并发、弱事务的互联网应用,MongoDB可以应对自如。MongoDB内置的水平扩展机制提供了从百万到十亿级别的数据量处理能力,完全可以满足Web2.0和移动互联网的数据存储需求,其开箱即用的特性也大大降低了中小型网站的运维成本。

适用场合:
\begin{itemize}
\item 网站数据:Mongo非常适合实时的插入,更新与查询,并具备网站实时数据存储所需的复制及高度伸缩性。
\item 缓存:由于性能很高,Mongo也适合作为信息基础设施的缓存层。在系统重启之后,由Mongo搭建的持久化缓存层可以避免下层的数据源 过载。
\item 大尺寸,低价值的数据:使用传统的关系型数据库存储一些数据时可能会比较昂贵,在此之前,很多时候程序员往往会选择传统的文件进行存储。
\item 高伸缩性的场景:Mongo非常适合由数十或数百台服务器组成的数据库。Mongo的路线图中已经包含对MapReduce引擎的内置支持。
\item 用于对象及JSON数据的存储:Mongo的BSON数据格式非常适合文档化格式的存储及查询。
\end{itemize}


\subsection{Redis}
Redis是一个开源的使用ANSI C语言编写、支持网络、可基于内存亦可持久化的日志型、Key-Value数据库,并提供多种语言的API。
从2010年3月15日起,Redis的开发工作由VMware主持。

Redis是一个key-value存储系统。和Memcached类似,它支持存储的value类型相对更多,包括string(字符串)、list(链表)、set(集合)、zset(sorted set --有序集合)和hash(哈希类型)。这些数据类型都支持push/pop、add/remove及取交集并集和差集及更丰富的操作,而且这些操作都是原子性的。在此基础上,redis支持各种不同方式的排序。与memcached一样,为了保证效率,数据都是缓存在内存中。
区别的是redis会周期性的把更新的数据写入磁盘或者把修改操作写入追加的记录文件,并且在此基础上实现了master-slave(主从)同步。

Redis的外围由一个键、值映射的字典构成。与其他非关系型数据库主要不同在于:Redis中值的类型不仅限于字符串,还支持如下抽象数据类型:
\begin{itemize}
\item 字符串列表
\item 无序不重复的字符串集合
\item 有序不重复的字符串集合
\item 键、值都为字符串的哈希表
\end{itemize}
值的类型决定了值本身支持的操作。Redis支持不同无序、有序的列表,无序、有序的集合间的交集、并集等高级服务器端原子操作。

Redis通常将全部的数据存储在内存中。2.4版本后可配置为使用虚拟内存,一部分数据集存储在硬盘上,但这个特性废弃了。
目前通过两种方式实现持久化:
\begin{itemize}
\item 使用快照,一种半持久耐用模式。不时的将数据集以异步方式从内存以RDB格式写入硬盘。
\item 1.1版本开始使用更安全的AOF格式替代,一种只能追加的日志类型。将数据集修改操作记录起来。Redis能够在后台对只可追加的记录作修改来避免无限增长的日志。
\end{itemize}


Redis支持主从同步。数据可以从主服务器向任意数量的从服务器上同步,从服务器可以是关联其他从服务器的主服务器。这使得Redis可执行单层树复制。从盘可以有意无意的对数据进行写操作。由于完全实现了发布/订阅机制,使得从数据库在任何地方同步树时,可订阅一个频道并接收主服务器完整的消息发布记录。同步对读取操作的可扩展性和数据冗余很有帮助。


\subsection{Casssandra}
Cassandra最初由Facebook开发,后来成了Apache开源项目,它是一个网络社交云计算方面理想的数据库。它集成了其他的流行工具如Solr,现在已经成为一个完全成熟的大型数据存储工具。Cassandra是一个混合型的非关系的数据库,类似于Google的BigTable。其主要功能比Dynomite(分布式的Key-Value存储系统)更丰富,但支持度却不如文档存储MongoDB。Cassandra的主要特点就是它不是一个数据库,而是由一堆数据库节点共同构成的一个分布式网络服务,对Cassandra的一个写操作,会被复制到其他节点上去,而对Cassandra的读操作,也会被路由到某个节点上面去读取。在最近的一次测试中,Netflix建立了一个288个节点的集群。

\subsection{DynamoDB}
DynamoDB是亚马逊的key-value模式的存储平台,可用性和扩展性都很好,性能也不错:读写访问中99.9\%的响应时间都在300ms内。DynamoDB的NoSQL解决方案,也是使用键/值对存储的模式,平且通过服务器把所有的数据存储在SSD上的三个不同的区域。如果有更高的传输需求,DynamoDB也可以在后台添加更多的服务器。

\subsection{CouchDB}
CouchDB是用Erlang开发的面向文档的数据库系统,不过它不是一个传统的关系数据库,而是面向文档的数据库,其数据存储方式有点类似lucene的index文件格式,CouchDB最大的意义在于它是一个面向web应用的新一代存储系统。作为一个分布式的数据库,CouchDB可以把存储系统分布到n台物理的节点上面,并且很好的协调和同步节点之间的数据读写一致性。CouchDB支持REST API,可以让用户使用JavaScript来操作CouchDB数据库,也可以用JavaScript编写查询语句,可以想像一下,用AJAX技术结合CouchDB开发出来的CMS系统会是多么的简单和方便。


\subsection{Lucene/Solr}
Lucene是Apache软件基金会4 jakarta项目组的一个子项目,这是一个开放源代码的全文检索引擎工具包,就是说它不是一个完整的全文检索引擎,而是一个全文检索引擎的架构。不过大多数人并不认同Lucene是一个数据库,因为大多数人只是用它来检索大量的文本块,不过它的确采用了与其他NoSQL数据存储相似的模型。如果说查询并不是仅仅局限于精确的匹配,而是寻找出那些出现在块中的字或者字段的话,毫无疑问,Lucene/Solr是最好的查询方式。














