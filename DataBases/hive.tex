\section{Hive}

To start hive in CLI mode, type \verb$hive$ or

\begin{verbatim}
beeline -u jdbc:hive2://
\end{verbatim}


Check built-in functions:

Hive Function Meta commands

\begin{itemize}
  \item 
SHOW FUNCTIONS– lists Hive functions and operators
  \item 
DESCRIBE FUNCTION [function name]– displays short description of the function
  \item 
DESCRIBE FUNCTION EXTENDED [function name]– access extended description of the function
\end{itemize}


\url{https://www.qubole.com/resources/cheatsheet/hive-function-cheat-sheet/}



\subsection{basic commands}

Database/Schema create:

\begin{verbatim}
  create database NAME;
\end{verbatim}

Table create:
\begin{verbatim}
CREATE EXTERNAL TABLE test_p(
id int,
name string 
)
partitioned by (date STRING)
ROW FORMAT DELIMITED FIELDS TERMINATED BY '\,' LINES TERMINATED BY '\n' 
STORED AS TEXTFILE;
\end{verbatim}


Data loading:

\begin{verbatim}
LOAD DATA [LOCAL] INPATH 'filepath' [OVERWRITE] INTO TABLE tablename [PARTITION (partcol1=val1, partcol2=val2 ...)]
\end{verbatim}


Data insert:
\begin{verbatim}
insert overwrite table session_hour_partition_save PARTITION (date='20170101',hour='01') select ....
\end{verbatim}

From Hive to HDFS:
\begin{verbatim}
insert overwrite  directory '/user/hive/tmp' select * from test;
\end{verbatim}

\subsection{Mapjoin: a trick to accelerate small table joins}


  Mapjoin is a little-known feature of Hive. 
  It allows a table to be loaded into memory so that a (very fast) join could be performed entirely within a mapper without having to use a Map/Reduce step. 
  If your queries frequently rely on small table joins (e.g. cities or countries, etc.) you might see a very substantial speed-up from using mapjoins.
There are two ways to enable it. 
First is by using a hint, which looks like \verb$/*+ MAPJOIN(aliasname), MAPJOIN(anothertable) */$. 
This C-style comment should be placed immediately following the SELECT. 
It directs Hive to load aliasname (which is a table or alias of the query) into memory.


\begin{verbatim}
  SELECT /*+ MAPJOIN(c) */ * FROM orders o JOIN cities c ON (o.city_id = c.id);
\end{verbatim}

Another (better, in my opinion) way to turn on mapjoins is to let Hive do it automatically.
 Simply set hive.auto.convert.join to true in your config, 
 and Hive will automatically use mapjoins for any tables smaller than hive.mapjoin.smalltable.filesize (default is 25MB).

Mapjoins have a limitation in that the same table or alias cannot be used to join on different columns in the same query. 
(This makes sense because presumably Hive uses a HashMap keyed on the column(s) used in the join, and such a HashMap would be of no use for a join on different keys).
The workaround is very simple - do not use the same aliases in your query.

