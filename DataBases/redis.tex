%!Mode:: "TeX:UTF-8"
\section{Redis}

\subsection{Installation and Start}

To install on CentOS 6:
\begin{verbatim}
yum install epel-release
yum install redis
service redis start
\end{verbatim}

Installation on OSX:

\begin{verbatim}
brew install redis
## To have launchd start redis now and restart at login:
brew services start redis
## Or, if you don't want/need a background service you can just run:
redis-server /usr/local/etc/redis.conf
\end{verbatim}


To start a redis client connecting to local server:
\begin{verbatim}
redis-cli
redis-cli -h 172.30.35.1 
\end{verbatim}

To allow remote access, modify redis.conf:
\begin{verbatim}
bind 0.0.0.0
\end{verbatim}

Before connecting to a remote database, we can first try to ping it:
\begin{verbatim}
redis-cli -h 172.30.35.1 ping
\end{verbatim}

Simple manipulation:
\begin{verbatim}
redis 127.0.0.1:6379> set age 22
OK
redis 127.0.0.1:6379> incr age
(integer) 23
redis 127.0.0.1:6379> get age
"24"
redis 127.0.0.1:6379> set name 'ka'
OK
redis 127.0.0.1:6379> get name
"ka"
\end{verbatim}

We can delete a key by using 
\begin{verbatim}
DEL keynname
\end{verbatim}


To list all keys matching a wildcard, type like this:
\begin{verbatim}
KEYS *
\end{verbatim}

\subsection{Programming Redis with Python}
Python API is provided with the \textbf{redis} package.

\begin{verbatim}
import redis
r = redis.Redis(host='localhost', port=6379, db=0)
x = r.get('age')
r.set('foo', 'bar')
y = r.get('foo')
\end{verbatim}









