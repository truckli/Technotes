%!Mode:: "TeX:UTF-8"
\section{MySQL}


\subsection{Deploying MySQL}

Starting MySQL in Mac OS X:
\begin{lstlisting}
mysql.server start
\end{lstlisting}
If that fails, just type \verb|mysqld|.

To Create a database in MySQL,
\begin{lstlisting}
create database samp_db character set gbk;
CREATE DATABASE IF NOT EXISTS mdss DEFAULT CHARSET utf8 COLLATE utf8_general_ci;
\end{lstlisting}


To create a table, the official site of MySQL gives an example:
\begin{lstlisting}
 CREATE TABLE pet (name VARCHAR(20), owner VARCHAR(20), species VARCHAR(20), sex CHAR(1), birth DATE, death DATE);
\end{lstlisting}

Read meta data:
\begin{lstlisting}
desc haah;
describe haah;
show columns from haah;
SHOW CREATE TABLE tablename;
show full columns from haah;
\end{lstlisting}

Schema can be specified when login,
\begin{lstlisting}
mysql -D dbname -h hostname -u username -p
\end{lstlisting}

SQL statements or a script can also be specified during the login command,
\begin{lstlisting}
mysql -D samp_db -u root  < h.sql
mysql -D samp_db -u root  -e 'select * from haah;'
\end{lstlisting}


\subsection{Configuration}

Check timeout settings,
\begin{verbatim}
show variables where variable_name like '%timeout';
\end{verbatim}


You change default value in MySQL configuration file

\begin{verbatim}
[mysqld]
connect_timeout=100
\end{verbatim}

Or set this in statements like
\begin{verbatim}
con.query('SET GLOBAL connect_timeout=28800')
con.query('SET GLOBAL wait_timeout=28800')
con.query('SET GLOBAL interactive_timeout=28800')
\end{verbatim}

\subsection{security management}

If you have never set a root password for MySQL, the server does not require a password at all for connecting as root. To set up a root password for the first time, use the mysqladmin command at the shell prompt as follows:

\begin{verbatim}
mysqladmin -u root password NEWPASS 
\end{verbatim}

If you want to change (or update) a root password to the new password 'newpass', then you need to use the following command:

\begin{verbatim}
mysqladmin -u root -p OLDPASS NEWPASS 
\end{verbatim}

\subsection{MySQL with Python}

Installation on Windows:
\begin{verbatim}
install using wheel
pip install wheel
download from http://www.lfd.uci.edu/~gohlke/pythonlibs/#mysql-python
pip install mysqlclient-1.3.8-cp36-cp36m-win_amd64.whl
\end{verbatim}


\subsection{Data Export and Import with CSV files}

\begin{verbatim}

mysql -uroot -p123456 -hmdss18 -Dmdss -e "SELECT * FROM i_ddos_current_events
INTO OUTFILE '/tmp/i_ddos_current_events.csv'
FIELDS TERMINATED BY ','
ENCLOSED BY '\"'
LINES TERMINATED BY '\n';"

scp root@mdss18:/tmp/*_ddos_*.csv root@mdss27:/tmp

mysql -uroot -p123456 -hmdss27 -Dmdss -e "
TRUNCATE i_ddos_current_events;
LOAD DATA INFILE '/tmp/i_ddos_current_events.csv' REPLACE
INTO TABLE i_ddos_current_events
FIELDS TERMINATED BY ','
ENCLOSED BY '\"'
LINES TERMINATED BY '\n';

\end{verbatim}


\subsection{Data Migration via SQL Scripts}

\begin{verbatim}
mysqldump -uroot -p123456 -hmdss18 mdss i_ddos_flood_events --where 'id in
(select id from i_ddos_current_events)'
 --lock-tables=false --no-create-info --insert-ignore 
 > /tmp/i_ddos_flood_events.sql
 
 mysql -uroot -p123456 -hmdss27 -Dmdss --force < /tmp/i_ddos_flood_events.sql
\end{verbatim}

\subsection{MySQL Ping}
\begin{verbatim}
/* ping */ select 1
\end{verbatim}



