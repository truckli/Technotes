%!Mode:: "TeX:UTF-8"
\section{SQL}
SQL是介于关系代数和关系演算之间的结构化查询语言,已经成为关系数据库的标准语言。


SQL is a set-based, declarative query language, not an imperative language like C or BASIC. However, there are extensions to Standard SQL which add procedural programming language functionality, such as control-of-flow constructs. SQL/PSM (SQL/Persistent Stored Modules) is an ISO standard mainly defining an extension of SQL with a procedural language for use in stored procedures. However, the major SQL vendors have historically included their own proprietary procedural extensions. 

SQL包含3个部分:
\begin{itemize}
    \item 
“数据定义语言”(DDL : Data Definition Language)是SQL语言集中,负责数据结构定义与数据库对象定义的语言,由CREATE、ALTER与DROP三个语法所组成,最早是由 Codasyl (Conference on Data Systems Languages) 数据模型开始,现在被纳入 SQL 指令中作为其中一个子集。
    \item 
“数据操纵语言”(DML : Data Manipulation Language)负责对数据库对象运行数据访问工作的指令集,以INSERT、UPDATE、DELETE三种指令为核心,分别代表插入、更新与删除。Performing read-only queries of data is sometimes also considered a component of DML.有很多开发人员都把加上SQL的SELECT语句的四大指令以“CRUD”来称呼。SELECT \ldots INTO是非标准的持久化操作。
    \item 
“数据控制语言”(DCL : Data Control Language)是一种可对数据访问权进行控制的指令,它可以控制特定用户帐户对数据表、查看表、预存程序、用户自定义函数等数据库对象的控制权。由 GRANT 和 REVOKE 两个指令组成。
\end{itemize}

\begin{figure}[htpb]
    \begin{center}
        \includegraphics[keepaspectratio,width=0.5\paperwidth]{Pictures/SqlANATOMY.png}
    \end{center}
    \caption{SQL language elements}
    \label{fig:SQL lan elem}
\end{figure}

某些数据库强制要求语句后有分号。

\subsection{真值逻辑}

空值(关键字NULL),关系数据库中对数据属性未知或缺失的一种标识。数据库表主键的取值不能为空值。在SQL的Where条件式去判断字段是否为Null时,where id = null 是无法正确执行的,必须写成 where id is null, 其否定条件是is not null。

三值逻辑:true, false, unknown。false AND unknown为false,true OR unknown为true。与null比较运算的结果是unknown.

expr1 IS NOT DISTINCT FROM expr2 表示二者的值相同或二者都为NULL。其否定为IS DISTINCT FROM。


\subsection{ANSI SQL数据类型}
\begin{itemize}
   \item 
 CHARACTER(n) or CHAR(n): fixed-width n-character string, padded with spaces as needed
   \item 
CHARACTER VARYING(n) or VARCHAR(n): variable-width string with a maximum size of n characters
   \item 
NATIONAL CHARACTER(n) or NCHAR(n): fixed width string supporting an international character set
   \item 
NATIONAL CHARACTER VARYING(n) or NVARCHAR(n): variable-width NCHAR string
   \item 
BIT(n): an array of n bits
   \item 
BIT VARYING(n): an array of up to n bits
   \item 
INTEGER and SMALLINT
   \item 
FLOAT, REAL and DOUBLE PRECISION
   \item 
NUMERIC(precision, scale) or DECIMAL(precision, scale),precision为有效数字位数,scale为小数位数
   \item 
DATE: for date values (e.g. 2011-05-03)
   \item 
TIME: for time values (e.g. 15:51:36). The granularity of the time value is usually a tick (100 nanoseconds).
   \item 
TIME WITH TIME ZONE or TIMETZ: the same as TIME, but including details about the time zone in question.
   \item 
TIMESTAMP: This is a DATE and a TIME put together in one variable (e.g. 2011-05-03 15:51:36).
   \item 
TIMESTAMP WITH TIME ZONE or TIMESTAMPTZ: the same as TIMESTAMP, but including details about the time zone in question.
\end{itemize}





\subsection{关键词}
关键词\textbf{DISTINCT}用于返回唯一不同的值。
\begin{verbatim}
SELECT DISTINCT 列名称 FROM 表名称
\end{verbatim}


\textbf{IN}操作符允许我们 WHERE子句中规定多个值。

\begin{verbatim}
SELECT column_name(s)
FROM table_name
WHERE column_name IN (value1,value2,...)
\end{verbatim}

操作符 \textbf{BETWEEN ... AND} 会选取介于两个值之间的数据范围。这些值可以是数值、文本或者日期。然而区间的开闭性因厂商而异。

\begin{verbatim}
SELECT column_name(s)
FROM table_name
WHERE column_name
BETWEEN value1 AND value2
\end{verbatim}

通过使用AS,可以为列名称和表名称指定别名(Alias)。

\begin{verbatim}
SELECT column_name(s)
FROM table_name
AS alias_name
\end{verbatim}

TOP 子句用于规定要返回的记录的数目。并非所有的数据库系统都支持 TOP 子句。
SQL Server 的语法: 
\begin{verbatim}
SELECT TOP number|percent column_name(s)
FROM table_name
\end{verbatim}

MySQL 语法:
\begin{verbatim}
SELECT column_name(s)
FROM table_name
LIMIT number
\end{verbatim}

Oracle 语法:
\begin{verbatim}
SELECT column_name(s)
FROM table_name
WHERE ROWNUM <= number
\end{verbatim}


\subsection{举例}

关系定义、删除与修改:
\begin{lstlisting}[language=SQL]
CREATE TABLE My_table(
 my_field1 INT,
 my_field2 VARCHAR(50),
 my_field3 DATE NOT NULL,
 PRIMARY KEY (my_field1, my_field2)
);

CREATE TABLE employees (
    id            INTEGER      PRIMARY KEY,
    first_name    VARCHAR(50)  NULL,
    last_name     VARCHAR(75)  NOT NULL,
    dateofbirth   DATE         NULL
);

CREATE TABLE MovieStar (
  name CHAR(30),
  address VARCHAR(255),
  gender CHAR(1),
  birthdate DATE
);

DROP TABLE My_table;

ALTER TABLE MovieStar ADD phone CHAR(3) NOT NULL;
ALTER TABLE MovieStar DROP birthdate;
\end{lstlisting}

索引创建与删除:
\begin{lstlisting}[language=SQL]
CREATE INDEX YearIndex ON Movie(year);
CREATE INDEX keyIndex ON Movie(title,year);
DROP INDEX YearIndex;
\end{lstlisting}

视图创建与删除:
\begin{lstlisting}[language=SQL]
CREATE VIEW ParamountMovie AS
  SELECT title, year
  FROM Movie
  WHERE studioName = 'Paramount';
  
CREATE VIEW MovieProd(movieTitle, prodName) AS
  SELECT title, year
  FROM Movie, MovieExe
  WHERE producerC# = cert#;
  
DROP VIEW ParamountMovie;
\end{lstlisting}

数据更新:
\begin{lstlisting}[language=SQL]
INSERT INTO My_table
 (field1, field2, field3)
 VALUES
 ('test', 'N', NULL);

UPDATE My_table
 SET field1 = 'updated value'
 WHERE field2 = 'N';

DELETE FROM My_table
 WHERE field2 = 'N';
 
TRUNCATE TABLE My_table;--清零
\end{lstlisting}

权限操作:
\begin{lstlisting}[language=SQL]
GRANT SELECT, UPDATE
 ON My_table
 TO some_user, another_user;
 
REVOKE SELECT, UPDATE
 ON My_table
 FROM some_user, another_user;
\end{lstlisting}

数据查询:
\begin{lstlisting}[language=SQL]
SELECT *
 FROM Book
 WHERE price > 100.00
 ORDER BY title;

 SELECT * FROM Persons
WHERE City LIKE 'N%'
--从"Persons" 表中选取居住在以 "N" 开始的城市里的人
--提示:"%" 可用于定义通配符(模式中缺少的字母)。


SELECT Book.title AS Title,
 COUNT(*) AS Authors
 FROM Book JOIN Book_author
 ON Book.isbn = Book_author.isbn
 GROUP BY Book.title;

SELECT * FROM Persons
WHERE LastName IN ('Adams','Carter')

SELECT * FROM Persons
WHERE LastName
BETWEEN 'Adams' AND 'Carter'

SELECT title,
 COUNT(*) AS Authors
 FROM Book NATURAL JOIN Book_author
 GROUP BY title;

SELECT isbn,
 title,
 price,
 price * 0.06 AS sales_tax
 FROM Book
 WHERE price > 100.00
 ORDER BY title;

SELECT isbn, title, price
 FROM Book
 WHERE price < (SELECT AVG(price) FROM Book)
 ORDER BY title;

SELECT Customer,SUM(OrderPrice) FROM Orders
GROUP BY Customer
-- 在 SQL 中增加 HAVING子句原因是,WHERE 关键字无法与合计函数一起使用。
HAVING SUM(OrderPrice)<2000

SELECT CASE WHEN i IS NULL THEN 'Null Result'  
-- This will be returned when i is NULL
            WHEN     i = 0 THEN 'Zero'         
-- This will be returned when i = 0
            WHEN     i = 1 THEN 'One'          
-- This will be returned when i = 1
            END
FROM t;

\end{lstlisting}


事务:
\begin{lstlisting}[language=SQL]
START TRANSACTION;
 UPDATE Account SET amount=amount-200 WHERE account_number=1234;
 UPDATE Account SET amount=amount+200 WHERE account_number=2345;
 
IF ERRORS=0 COMMIT;
IF ERRORS<>0 ROLLBACK;

CREATE TABLE tbl_1(id INT);
 INSERT INTO tbl_1(id) VALUES(1);
 INSERT INTO tbl_1(id) VALUES(2);
COMMIT;
 UPDATE tbl_1 SET id=200 WHERE id=1;
SAVEPOINT id_1upd;
 UPDATE tbl_1 SET id=1000 WHERE id=2;
ROLLBACK TO id_1upd;
 SELECT id FROM tbl_1;
\end{lstlisting}


\subsection{SQL 分页查询}

\begin{lstlisting}[language=SQL]
-- SQL SERVER
SELECT TOP 页大小 * FROM table1 WHERE id NOT IN ( SELECT TOP 页大小*(页数-1) id FROM table1 ORDER BY id ) ORDER BY id
--MYSQL
SELECT * FROM TT LIMIT startline, linecount
\end{lstlisting}















