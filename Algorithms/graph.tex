%!Mode:: "TeX:UTF-8"
\section{图相关算法}
图有邻接表和邻接矩阵两种表示方式。对于稠密图,$E=\Theta(|V|^2)$。

广度优先搜索(BFS)伪代码:
\begin{lstlisting}[language=bash]
BFS(G, s)
  for u ← V[G] - {s}
       do color[u] ← WHITE
          d[u] ← INFINITY
          p[u] ← NIL
  color[s] ← GRAY
  d[s] ← 0
  p[s] ← NIL
  Q ← Ø
  ENQUEUE(Q, s)
  while Q != Ø
      do u ← DEQUEUE(Q)
         for v ← Adj[u] 
             do if color[v] = WHITE
                   then color[v] ← GRAY
                        d[v] ← d[u] + 1
                        p[v] ← u
                        ENQUEUE(Q, v)
         color[u] ← BLACK
\end{lstlisting}
由于每个顶点出队、入队各一次,扫描邻接表时每个边被扫描一次(聚集分析),因此BFS的运行时间为$O(V+E)$。

%DFS是对前序遍历的推广。
%如果G是一棵树,由于$|E|=\Theta(|V|)$,那么该树的所有顶点在总时间$|E|$都能被访问到。
深度优先搜索(DFS)伪代码:
\begin{lstlisting}[language=bash]
DFS(G)
  for u ← V[G]
       do color[u] ← WHITE
          p[u] ← NIL
  time ← 0
  for u ← V[G]
       do if color[u] = WHITE
             then DFS-VISIT(u)

DFS-VISIT(u)
  color[u] ← GRAY
  d[u] ← ++time
  for v ← Adj[u] 
       do if color[v] = WHITE
             then p[v] ← u
                 DFS-VISIT(v)
  color[u] ← BLACK
  f[u] ← ++time
\end{lstlisting}
每个结点调用DFS-VISIT一次。利用聚集分析可知,DFS的运行时间为$O(V+E)$。

\subsection{单源最短路径}
对于路径无权值情形,使用广度优先搜索即可。

如果路径有\textbf{非负}权值,可使用Dijkstra算法。Dijkstra算法是贪心算法的很好的例子。




\begin{lstlisting}[language=bash]
DIJKSTRA(G, w, s)
  for u ← V[G] do 
      d[u] ←  infinity
      p[u] ← NIL
  d[s] ← 0
  S ← Ø
  Q ← V[G] 
  while Q != Ø
      do u ← EXTRACT-MIN(Q)
         S ← S + {u}
         for v ← Adj[u] do
             if d[v] > d[u] + w(u, v)
                then d[v] ← d[u] + w(u, v)
                     p[v] ← u
\end{lstlisting}
上述Q相当于未知节点集合。


不用堆时复杂度$O(V^2)$,对稠密图来说是最优的;
使用二叉堆时,运行时间$O((V+E)\lg V)$:这里松弛操作执行了$|E|$次,每次都会破坏优先级队列,导致$\lg V$的开销;
EXTRACT-MIN执行了$|E|$次,也导致$\lg V$的开销的开销。对于稀疏图,$E=o(V^2/ \lg V)$, 则相对$O(V^2)$的实现而言构成改进。
使用斐波那契堆时,松弛操作只有常数开销,运行时间能达到$O(V\lg V + E)$。

\subsection{最小生成树}
这里的最小生成树算法一般考虑无向图,因为有向图中找生成树比较困难。
只有连通图才有生成树。对于最小生成树,贪心算法是成立的,Prim和Kruskal都是贪心算法,区别是在于最小边的选取上。

Prim算法使得生成树一步一步增长,每一步,都对应一个已经添加到树上的顶点集,而其余顶点尚未加到树上。
Prim算法同Dijkstra算法的结构非常相似,除了簿记量不同,因此也有相同的时间复杂度。

key[v]是v到已知顶点(半拉子生成树)的最短的边,p[u]是导致key[v]改变的最后的顶点。
\begin{lstlisting}[language=bash]
MST-PRIM(G, w, r)
   for u ← V [G]
        do key[u] ← infinity
           p[u] ← NIL
   key[r] ← 0
   Q ← V [G]
   while Q != Ø 
        do u ← EXTRACT-MIN(Q)
           for v ← Adj[u] 
               do if v in Q and w(u, v) < key[v]
                    then p[v] ← u
                         key[v] ← w(u, v)
\end{lstlisting}

Kruskal算法连续地选择最小权值边,当不产生回路时就选中。它处理的是森林,起初有$|V|$个单结点树,算法终止时只有一颗树了。


Kruskal算法运行时间取决于不相交集合结构如何实现。
最坏运行时间$O(|E|\log|E|)$,由于$|E| < |V|^2)$,
也可表述为$O(|E|\log|V|)$。

\begin{lstlisting}[language=bash]
MST-KRUSKAL(G, w)
  A ← Ø
  for v ← V[G]
       do MAKE-SET(v)
  sort(E, w)
  for each edge (u, v) ← E, taken in nondecreasing order by weight
       do if FIND-SET(u) != FIND-SET(v)
             then A ← A + {(u, v)}
                  UNION(u, v)
  return A
\end{lstlisting}





















