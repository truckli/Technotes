%!Mode:: "TeX:UTF-8"
\section{二分查找代码示例}

注意事项:除以2的操作可改为右移操作。

程序可以有多种写法,主要有如下区别:
\begin{itemize}
    \item 不变式区间的开闭性。不变式:待搜数据t不存在,或属于不变式区间。区间可以为[l,u]或(l,u]。区间的开闭性影响迭代更新的方式。
    \item 每次循环的比较操作次数,可以为1次或两次。每次循环如果作两次判断,效率较低,但循环次数可能会减少,因为可以直接发现目标退出循环。可以将等于比较合并到大于或小于比较中来减少判断。
\end{itemize}


当区间长度至少为3时,下轮迭代区间必能能够缩小。当区间长度为2或1时,按照\verb|m=(u+l)/2|计算到的中点m即为左端点l,如果区间端点按照l=m方式来更新,会导致区间不能缩小。当区间长度为1时,m同时等于l和u,当区间右端点按照u=m来更新时,也会发生区间不能缩小的情形,如编程不慎则会造成死循环。

如果采用全闭区间-两次比较的方式编程,u和l都不会直接赋为m,则只需确保区间长度大于零即可,无死循环之虞。如果区间存在开端,或者只进行一次比较,则需要慎重。

可以对算法提出额外要求,即当存在多个满足条件的值时,应能返回下标最低者。

如果数组长度为固定值,如1000,可以通过循环展开以进一步优化代码。
虽然二分搜索可以在log时间内完成搜索,但搜索后如果要插入新元素仍需要线性时间。不过,基于数组的连续操作具有很好的cache友好性。

\subsection{半闭区间,单次比较}
\cite{pp}9.3。 x[-1],x[n]作哨兵,但未真正访问。u和l都会直接赋为m,因此循环条件是区间长度至少为3,即l+1 != u。t包含于(l,u]。左边为开,能够保证返回最低下标。
\label{codes:binsort}

\begin{lstlisting}[language=C]

int binarysearch3(DataType t)
{	
	int l = -1, u = n, m;
	while (l+1 != u) {
		m = (l + u) >> 1;
		if (x[m] < t) l = m;
		else u = m;
	}
	if (u >= n || x[u] != t) return -1;
	return u;
}
\end{lstlisting}

然而,我并未发现为何将u初始化为n,而不是n-1。我认为初始化为n-1更简洁,相应代码为:
\begin{lstlisting}[language=Python]

#!/usr/bin/env python

"""
binary sort implementation
"""

def BinSort(x, t):
    l = -1
    u = len(x) - 1
    while u - l >= 2:
        m = l + (u - l) / 2;
        if x[m] < t:
            l = m
        else:
            u = m
    
    if x[u] == t:
        return u
    else: 
        return -1



if __name__ == '__main__':

    assert BinSort([2], 3) == -1
    assert BinSort([2], 2) == 0
    assert BinSort([2], 1) == -1
    assert BinSort([2, 3], 1) == -1
    assert BinSort([2, 3], 4) == -1
    assert BinSort([2, 3], 2) == 0
    assert BinSort([2, 3], 3) == 1
    assert BinSort([2, 3, 4], 3) == 1
    assert BinSort([2, 3, 4], 2) == 0
    assert BinSort([2, 3, 4], 4) == 2
    assert BinSort([2, 3, 4], 5) == -1
    assert BinSort([2, 3, 4], 1) == -1
    assert BinSort([2, 3, 4], 2.5) == -1
    assert BinSort([2, 3, 4], 3.5) == -1
    assert BinSort([2, 3, 4, 5], 3.5) == -1
    assert BinSort([2, 3, 4, 5], 4.5) == -1
    assert BinSort([2, 3, 4, 5], 4) == 2
    assert BinSort([2, 3, 4, 5], 0) == -1

\end{lstlisting}

\subsection{半闭区间双次比较}
将单次比较改为双次比较没有难度。只需要多加一次相等比较。

\begin{lstlisting}[language=C]

int binarysearch3(DataType t)
{	
	int l = -1, u = n, m;
	while (l+1 != u) {
		m = (l + u) / 2;
		if (x[m] < t) l = m;
		else if x[m] &=& t return m;
		else u = m;
	}
	if (u >= n || x[u] != t) return -1;
	return u;
}
\end{lstlisting}

\subsection{闭区间单比较}
\cite{self}一段Python代码:
\begin{lstlisting}[language=Python]

def binsearch(x, n, t):
    #range size >= 1
    assert n > 0

    if x[0] > t or x[n-1] < t: return -1;

    l = 0
    u = n-1
    while (u-l >= 1): 
        #x[l] <= t <= x[u], l == 0 or x[l-1] < t, range size >= 2
        m = l + (u - l) / 2;
        print "[%d - %d], %d"%(l,u,m)
        if x[m] < t: 
            l = m + 1
        else: 
            u = m
    
    #range size is 1 
    if x[l] == t: return l
    return -1

\end{lstlisting}

\cite{bop}3.11也给出了闭区间单次比较算法的实现,该实现不满足返回最小下标的条件。代码没有对输入参数进行检查,如果输入区间为零,程序功能可能与期望不符。书中认为求midIndex不可以相加除以2,以防溢出。其实如此大数本不太可能作为数组的长度。
因为l按照来l=m方式更新,所以区间长度为2时即需要退出循环。更新u的方式也过于保守,是按照开区间的方式进行更新。

其代码的Python等效版本写为:
\begin{lstlisting}[language=Python]

def binsearch(x,l,u,t):
    while (l < u - 1): #range lengh >= 3 
        m = l + (u - l) / 2;
        print "[%d - %d], %d"%(l,u,m)
        if x[m] <= t:
            l = m
        else: 
            u = m # better be u = m-1
    
    #range size is 2 or 1 
    if x[u] == t: return u
    if x[l] == t: return l
    return -1

\end{lstlisting}

原书程序为:

\begin{lstlisting}[language=C]
int bisearch(char** arr, int b, int e, char* v)
{
    int minIndex = b, maxIndex = e, midIndex;
    //循环结束有两种情况:
    //若minIndex为偶数则minIndex ==  maxIndex
    //否则就是minIndex ==  maxIndex -1
    while (minIndex < maxIndex -1)
    {
	minIndex = minIndex + (maxIndex - minIndex) / 2;
	if (strcmp(arr[minIndex], v) <= 0)
	{
	    minIndex = midIndex;
	}
	else
	{
	    //不需要minIndex - 1, 防止minIndex == maxIndex
	    maxIndex = midIndex;
	}
    }

    if (!strcmp(arr[maxIndex],v))//先判断序号最大的值
    {
	return maxIndex;
    }
    else if (!strcmp(arr[minIndex],v))
    {
	return minIndex;
    }
    else
    {
	return -1;
    }
}
\end{lstlisting}


\subsection{闭区间双次比较}
\cite{pp}5.1。t包含于[l,u]。u和l都不会赋值为m。循环条件是区间长度至少为1,即l<=u。退出循环则意味着无解。如果一开始元素个数就为零,自然无法进入循环,算作无解。
\begin{lstlisting}[language=C]
int binarysearch2(DataType t)
{	
	int l, u, m;
	l = 0;
	u = n-1;
	while (l <= u) //区间大于等于1 
	{
		m = (l + u) / 2;
		if (x[m] < t)
			l = m+1;
		else if (x[m] == t)
			return m;
		else /* x[m] > t */
			u = m-1;
	}

	return -1;
}


\end{lstlisting}

\cite{sedgewick}P99也给出了这种算法:

\begin{lstlisting}[language=Java]
private int rank(int key)
{ // Binary search.
    int lo = 0;
    int hi = a.length - 1;
    while (lo <= hi)
    { // Key is in a[lo..hi] or not present.
	int mid = lo + (hi - lo) / 2;
	if (key < a[mid]) hi = mid - 1;
	else if (key > a[mid]) lo = mid + 1;
	else
	return mid;
    }
    return -1;
}

\end{lstlisting}








