%!Mode:: "TeX:UTF-8"

\section{快速排序}
\label{codes:quicksort}
就地不稳定排序,最差时间$O(N^2)$,平价时间$O(Nlog(N))$,空间复杂度$O(logN)$(递归造成的空间开销)。


有许多版本,包括就地非稳定版本和稳定非就地版本。
网上有一些非权威程序员给出了非递归版本,自行实现栈结构以模拟递归调用行为。

\cite{pp}和\cite{sedgewick}提出了以下几种划分方案:
\begin{itemize}
    \item 
	Lomuto划分, x[l]存放t,[l+1,m]小于t,(m,i)大于等于t,[i,u]未知。
    \item 
	双向划分, x[l]存放t,[l+1,m]小于等于t, (m,i)未知,[i,u]大于等于t。
    \item 
	三路划分(\cite{pp}11.5.11称为宽支点划分)。[l, lt)部分小于t,[lt,i)等于t,[i,gt]未知,(gt,u]大于t。
\end{itemize}


对快速排序pivot的选择,早期常使用最左元素,导致对有序数组性能很差。R.Sedgewick提出\cite{wikipedia}选择pivot的方案:
\begin{itemize}
    \item 随机元素
    \item 中间元素
    \item 最左、最右和中间元素的均值
\end{itemize}

他同时提出了两种优化:
\begin{itemize}
    \item 先递归较短的那半数组,以保证最多使用$O(logN)$空间。较长的那半数组使用尾部递归或遍历,可能不额外增加堆栈空间。
    \item 数组段较短时不再排序,最终使用插入排序扫一遍,插入排序对于近似排好的数组很高效。
\end{itemize}

代码示例参考\ref{codes:quicksort}。




《算法导论》给出的分割算法没有判断输入条件,是因为边界条件已经在QUICKSORT代码中判断过了,
但这样的话\textbf{如果要求单独写分割算法可能引起误解},同时必须解释本节伪代码\textbf{数组下标从1开始}。

\begin{lstlisting}[language=C++]
PARTITION(A,l,u) 
 assert l < u
 exchange A[rand(l,u)] <-> A[u]
 m <- l - 1 
 for i <- l to u-1 do
    if A[i] <= A[u] then exchange A[++m] <-> A[i] 
 exchange A[++m] <-> A[u] 
 return m
\end{lstlisting}
模拟了尾递归的快速排序:
\begin{lstlisting}[language=C++]
TAIL-RECURSIVE-QUICKSORT(A,l,u)
  while l < u do
    m <- PARTITION(A,l,u)
    if m < (l+u)/2 do
      TAIL-RECURSIVE-QUICKSORT(A,l,m-1)
      l <- m+1
    else
      TAIL-RECURSIVE-QUICKSORT(A,m+1,u)
      u <- m-1
\end{lstlisting}



其他代码有:

\begin{lstlisting}[language=C++]
/* Sedgewick's version of Lomuto, with sentinel */
void qsort2(int l, int u)
{	int i, m;
	if (l >= u)
		return;
	m = i = u+1;
	do {
		do i--; while (x[i] < x[l]);
		swap(--m, i);
	} while (i > l);
	qsort2(l, m-1);
	qsort2(m+1, u);
}
\end{lstlisting}


\begin{lstlisting}[language=C++]
/* Two-way partitioning */
void qsort3(int l, int u)
{	int i, j;
	DType t;
	if (l >= u)
		return;
	t = x[l];
	i = l;
	j = u+1;
	for (;;) {
		do i++; while (i <= u && x[i] < t);
		do j--; while (x[j] > t);
		if (i > j)
			break;
		swap(i, j);
	}
	swap(l, j);
	qsort3(l, j-1);
	qsort3(j+1, u);
}
\end{lstlisting}


\begin{lstlisting}[language=C++]
/* qsort3 + randomization + isort small subarrays + swap inline */
int cutoff = 50;
void qsort4(int l, int u)
{	int i, j;
	DType t, temp;
	if (u - l < cutoff) return;
	swap(l, randint(l, u));
	t = x[l];
	i = l;
	j = u+1;
	for (;;) {
		do i++; while (i <= u && x[i] < t);
		do j--; while (x[j] > t);
		if (i > j)
			break;
		temp = x[i]; x[i] = x[j]; x[j] = temp;
	}
	swap(l, j);
	qsort4(l, j-1);
	qsort4(j+1, u);
}

\end{lstlisting}

\cite{sedgewick}提出了三路分割,用于应对有大量keys重复的情形,并证明三路分割的快速排序具有熵最优性。该算法使用了$(2lg2)NH$次比较,H为香农熵,$H=-\sum{p_{k}lnp_k}$。


\begin{lstlisting}[language=Java]

public class Quick3way
{
    private static void sort(Comparable[] a, int lo, int hi)
    { // See page 289 for public sort() that calls this method.
	if (hi <= lo) return;
	int lt = lo, i = lo+1, gt = hi;
	Comparable v = a[lo];
	while (i <= gt)
	{
	    int cmp = a[i].compareTo(v);
	    if (cmp < 0) exch(a, lt++, i++);
	    else if (cmp > 0) exch(a, i, gt--);
	    else
	    i++;
	} // Now a[lo..lt-1] < v = a[lt..gt] < a[gt+1..hi].
	sort(a, lo, lt - 1);
	sort(a, gt + 1, hi);
    }
}
\end{lstlisting}

\section{归并、插入、冒泡排序}

\textbf{归并排序}为\textbf{非就地、稳定}排序。
时间复杂度为$\Theta(N \log N)$(最好、最坏),空间复杂度为$O(N)$。对于链表,适合使用归并排序,此时不需要额外存储空间。
快速排序用于链表时性能很差,堆排序几乎不可行。
数组也可实现就地稳定的归并排序,但时间复杂度退化到Quasilinear($O(N \log^2 N)$)。

归并排序可实现为自顶向下式(递归式,《算法导论》风格)和自底向上式。
自然归并排序(Natural merge sort)是自底向上式的变种,利用了既已有序的子序列。



\begin{lstlisting}[language=bash]
MERGE_SORT(A,l,u)
  if l < u
    m <- (l+u)/2
    MERGE_SORT(A,l,m)
    MERGE_SORT(A,m+1,u)
    n1 <- m - l + 1, n2 <- u - m
    create B1[1..n1+1] <- [A[l,m],INFINITY]
    create B2[1..n2+1] <- [A[m+1,u],INFINITY]
    i <- j <- 1
    for k <- l to u do
      if B1[i] < B2[j] then
        A[k] <- B2[i++]
      else
        A[k] <- B1[j++]   
\end{lstlisting}

\textbf{插入排序}的优点:稳定,就地,on-line,对小数据集效率高,对近似排序的数据性能优异。在所有二次复杂度排序算法中是最快的。
最好(如,已排序)、平均、最坏(降序数组)情形下的时间复杂度分别为线性、二次方、二次方。
\begin{lstlisting}[language=bash]
INSERTION-SORT(A,n)
  for i <- 2 to n do
    key <- A[i]
    j <- i - 1
    while j > 0 and A[j] > key do
      A[j+1] <- A[j]
      j <- j-1
    A[j+1] <- key
\end{lstlisting}

\textbf{冒泡排序}优点是简单(适合教学)有趣,但很多科学家还是提倡连教学都不要用它。

\begin{lstlisting}[language=bash]
BUBBLE-SORT(A)
    last <- length[A] + 1
    repeat
       newn <- 1
       for i <- 2 to last - 1 do
          if A[i-1] > A[i] then
             exchange A[i-1] <-> A[i]
             newn <- i
       last <- newn
    until last = 1
\end{lstlisting}
上述代码中,不变式为:last为上次排好的元素,last及其以上元素已不需要排序,本次至少能排好A[last-1]。
冒泡排序就地、稳定,在最好的情形下为线性运行时间,如同插入排序,但比后者慢。




\section{堆排序}
堆排序(Heapsort)是指利用堆数据结构所设计的一种排序算法,隶属于基于选择的排序算法。堆是一个近似完全二叉树的结构。
其时间复杂度在最好、最差情形均为$O(Nlog(N))$。堆排序是就地不稳定排序,一般比快速排序算法慢,但是最差复杂度要更好。


基于\cite{pp}14.2给出的两个关键函数siftup和siftdown,仅用几行程序就可实现堆排序。
\begin{lstlisting}[language=Python]
    #make heap
    for i in range(2, len(x)):
	siftup(x, i) #必须是大顶堆
    
    #sort heap
    for i in range(len(x)-1, 1, -1):
	tmp = x[1]; x[1] = x[i]; x[i] = tmp
	siftdown(x, i-1)
\end{lstlisting}
对\cite{pp}14.2节siftup的优化包括x[0]置哨兵,放置一个大数;x[i]上浮时,不交换i,p两个位置,即将swap展开,只置i,不置p。siftdown很难使用哨兵,但也可以通过展开swap来进行优化。

《算法导论》给出的MAX-HEAPIFY方法相当于siftdown, 其构造堆的操作使用的是siftdown而非siftup:
\begin{lstlisting}[language=bash]
MAX-HEAPIFY(A,i)
  l <- 2i,r <- 2i+1,largest <- i
  if l <= heap-size[A] and A[l] > A[i] then
    largest <- l
  if r <= heap-size[A] and A[r] > A[largest] then
    largest <- r
  if largest != i then
    exchange A[largest] <-> a[i]
    MAX-HEAPIFY(A,largest)
\end{lstlisting}

BUILD-MAX-HEAP(A)操作的运行时间仅为\textbf{线性}:
$\sum \limits_{h=0} ^{\lfloor \lg n \rfloor} \lceil \frac{n}{2^{h+1}} \rceil O(h) = O(n \sum \limits_{h=0} ^{\lfloor \lg n \rfloor} \frac{h}{2^h}) = O(n \sum \limits_{h=0} ^{\infty} \frac{h}{2^h}) = O(n)$。
\begin{lstlisting}[language=bash]
BUILD-MAX-HEAP(A)
  heap-size[A] <- length[A]
  for i <- round(heap-size[A]/2)  downto 1 do
    MAX-HEAPIFY(A,i)

HEAP-SORT(A)
  BUILD-MAX-HEAP(A)
  for i <- length[A] downto 2 do
    exchange A[1] <-> A[i]
    heap-size[A] <- heap-size[A] - 1
    MAX-HEAPIFY(A,1)
\end{lstlisting}

C++标准库提供来堆操作,包括make\_heap, push\_heap, pop\_heap,sort\_heap等。可以用两行代码实现堆排序:
\begin{lstlisting}[language=Python]
    make_heap(a, a+n)
    sort_heap(a, a+n)
\end{lstlisting}

C++标准库也提供了对优先级队列priority\_queue的支持。

\section{排序算法实现}
\textbf{STL的sort实现了内省排序}。内省排序(intro sort)集成了快速排序和堆排序。这个排序算法首先从快速排序开始,当递归深度超过一定深度(深度为排序元素数量的对数值)后转为堆排序。采用这个方法,内省排序既能在常规数据集上实现快速排序的高性能,又能在最坏情况下仍保持$O(n\lg n)$的时间复杂度。

SGI STL的list容器的sort方法使用了自底向上归并排序,由于链表不方便使用2-2-2-2-4-4-8式均匀向上归并,这里临时分配64个临时list,分别管理长度为1、2、4、8\dots 的中间链表,原链表的元素逐一插入中间链表,中间链表如同二进制数值一般满则进位,最终归并所有中间链表即可。

C语言中,stdlib.h则提供了\textbf{qsort库函数}:
\begin{lstlisting}[language=C]
void qsort(void *base, size_t nmemb, size_t size,
    int (*compar)(const void *, const void *));
\end{lstlisting}

uClibc库使用了希尔排序。\textbf{希尔排序}又称缩减增量排序。其优点是运行较快(平均时间为亚二次复杂度$o(N^2)$),代码短且省内存(适合嵌入式系统)。
其时间复杂度取决于增量序列,且极难分析。Shell序列($\lfloor N/2^{k}\rfloor $)性能比较差,最差复杂度为$\Theta(N^2)$。
Sedgewick给出了一些最差复杂度$O(N^{4/3})$的序列。许多序列有$O(N^{3/2})$最差复杂度。
Pratt给出了最差复杂度$O(N \log^{2} N)$的序列为目前的最佳序列。

\textbf{Timsort}集成了归并排序和插入排序。Timsort于2002年由Tim Peters为Python设计,现已用于Java 7, Android和GNU Octave排列非内置类型。
Timsort分析既已有序(非降序和严格降序)的子数组,称为run。如果run短于某阈值则按进行插入排序使其达到阈值。严格降序的run被翻转成严格升序run。



\section{线性时间排序}
\textbf{计数排序}假定元素值介于0到k之间的整数,运行时间为$\Theta(n+k)$。
当$k=O(n)$时,运行时间为$\Theta(n)$。
计数排序是\textbf{稳定}的,需要$O(k)$临时存储,亦无法做到就地排序。
\begin{lstlisting}[language=bash]
COUNTING-SORT(A,B,k)
  create C[0..k] <- 0
  
  for i <- 1 to length[A] do
    C[A[i]] <-  C[A[i]] + 1
    
  for j <- 1 to k do
    C[j] <- C[j] + C[j-1]
    
  j <- 1
  while C[j] = 0 do
    j <- j + 1
    
  for i <- length[A] downto 1 do
    B[C[A[i]]] <- A[i]
     C[A[i]] <-  C[A[i]] - 1
\end{lstlisting}
之所以要在最后一个循环中从后向前遍历A[i],是为了保证稳定性:对于相同取值的元素,名次逐渐减小。

\textbf{基数排序}(Radix Sort)是\textbf{稳定排序},依赖于另一种针对整数的稳定排序,从低位到高位依次对每一位数排序。
给定n个k进制d位数,如果所用的整数排序算法需要$\Theta(n+k)$时间,则基数排序需要$\Theta(d(n+k))$时间。当k不太大时,可选择计数排序作为其辅助排序算法。

\textbf{桶排序}算法,如果输入数据满足一个特性:\textbf{各桶尺寸的平方和与总元素数呈线性关系},则桶排序算法以期望线性时间运行。
此时,桶内排序算法可使用\textbf{插入排序}:桶内元素不会太多,且桶多以链表方式实现,用插入排序可保证稳定性。



\section{排序算法总结}
\subsection{排序算法分类}
\label{subsec:sortAthmClass}

根据\cite{acp},排序算法包括内部排序和外部排序(不完全放入RAM)。内部排序分类:插入,交换,选择,合并,分布。
根据\cite{ita},排序算法包括基于比较的排序和非基于比较的排序(可以达到线性时间复杂度)。
\cite{weijipedia}给出了丰富的讲解材料。
\begin{verbatim}
http://en.wikipedia.org/wiki/Sorting_algorithm
\end{verbatim}

\begin{description}
    \item[基于插入的排序] 
        \hfill \\
	\begin{description}
	    \item[直接插入]就地稳定。$O(N^2)$。参\cite{ita}2.2。适用于近似有序数组。快速排序和插入排序可以联合使用,性能极佳。
	    \item[希尔(Shell)排序]就地,不稳定。\cite{acp}5.2.1D。也称递减增量排序算法,不需要大量的辅助空间,不稳定,复杂度取决于降序序列的选取,可能是$O(Nlg^{2}(N))$,$O(NlogN)$, $N^{\frac{6}{5}}$。
	    \item[图书馆排序]有n个元素待排序,这些元素被插入到拥有(1+e)n个元素的数组中。
	    \item[(二叉)树排序]稳定。线性空间复杂度。先构造二叉查找树,再in-order遍历之。从文件中读数时比较方便。平均时间$O(NlogN)$,二叉树不平衡时最差$O(N^2)$。
	    \item[其他]二路插入(使用二分法查找合适度位置,但未能减小移动数据度开销),表插入(以链表实现,减小数据复制开销),地址计算插入等。
	\end{description}
    \item[基于交换的排序] 
        \hfill \\
	\begin{description}
	    \item[冒泡]稳定就地排序。$O(N^2)$。数据交换过多,\cite{acp}认为不值得推荐。
	    \item[鸡尾酒排序]稳定就地排序。冒泡排序的变种。别名:双向冒泡排序, 鸡尾酒搅拌排序, 搅拌排序,涟漪排序, 来回排序。
	    \item[快速排序]就地不稳定。最差时间$O(N^2)$,平价时间$O(Nlog(N))$,空间$O(logN)$。\cite{sedgewick}认为快速排序是最快的。
	    \item[奇偶排序]奇偶换位排序,或砖排序。最初发明用于有本地互连的并行计算。
	    \item[Gnome Sort]又称stupid sort。类似插入排序,但通过同前面元素交换实现插入。
	    \item[梳子排序]冒泡排序的变种。不稳定。如同快速排序和合并排序,梳排序的效率在开始时最佳,接近结束时,即进入泡沫排序时最差。如果间距变得太小时(例如小于10),改用诸如插入排序或鸡尾酒排序等算法,则可提升整体效能。
	    \item[其他]合并交换(在可并行比较计算时有用),基数交换,地址计算插入等。
	\end{description}
    \item[基于选择的排序]
        \hfill \\
	\begin{description}
	    \item[直接选择]就地不稳定。$O(N^2)$。
	    \item[堆排序]就地不稳定。时间$O(Nlog(N))$,空间$O(1)$。\cite{ita}认为堆排序不如快速排序,但有其他用途,如实现优先级队列。	
	    \item[锦标赛排序]基于堆,待排数据初存于叶子节点,决出胜者,将胜者从其上升路径中删除,在存放胜者的叶子节点放置一个大数。时间$O(Nlog(N))$,需要额外空间作为堆。主要用于外排序的多路合并步骤。

	\end{description}
    \item[基于合并的排序] 
        \hfill \\
	\textbf{归并排序},非就地,稳定。\cite{ita}2.3节Merge例程先将待合并数组的两个有序子数组拷贝到外部,再拷回原数组使有序。
	比较操作的次数介于$(nlogn)/2$和$nlogn-n+1$,赋值操作的次数是$2nlogn$,空间复杂度为:Θ (n)\cite{weijipedia}。
	\cite{acp}提出了如下两个非递归的归并排序算法,额外分配了与原数组等长的缓冲区。
	\begin{description}
	    \item[自然的两路合并]\cite{acp}5.2.4算法N。
	    \item[直接两路合并]\cite{acp}5.2.4算法S。
	\end{description}
	另外,还有就地不稳定版本的合并排序(\cite{acp}Vol3 习题5.2.5,$O(NlgN)$时间,流程复杂)。也有就地稳定版本(\cite{acp}Vol3习题)的合并排序,流程简单,时间复杂度有所增加,失去了意义。对于就地不稳定版本,分为分块、预排序、两两归并、扫尾四部。分为m+2块,除最后一块外,其他快大小都是$sqrt(n)$。块内排序。块排序,依据块的首元素,如首元素相同则依据末尾元素。然后执行多次AUX排序。\url{http://blog.ibread.net/345/in-place-merge-sort/}.
    \item[基于分布的排序] \hfill \\

	\begin{description}
	    \item[桶排序]\cite{ita}8.4。稳定?,最差时间$O(n+k)$,平均时间$O(n+k)$,需要额外空间存放桶。It is a distribution sort, and is a cousin of radix sort in the most to least significant digit flavour. Bucket sort is a generalization of pigeonhole sort. 
	    \item[鸽巢排序]鸽巢排序(Pigeonhole sort), 也被称作基数分类\cite{weijipedia}(count sort\cite{wikipedia}), 时空复杂度均$O(n+k)$,适用于$k=O(n)$情形,否则不如桶排序。计数排序中元素移动一次,鸽巢排序则移动两次(入、出鸽巢)\cite{wikipedia}.
	    \item[比较计数]\cite{acp}5.2算法C,统计比该记录小的记录的个数。时间$O(N^2)$,空间$O(N)$。无大用。
	    \item[分布计数]\cite{acp}5.2算法D。\cite{ita}8.2。稳定排序。时间$O(n+k)a$,空间$O(k+n)$,k为分布空间大小,需要额外空间存放辅助表和排序结果。若$k=O(n)$,则时空均为$O(n)$。涉及的操作包括prefix sum(cumulative sum).
	    \item[基数排序]\cite{ita}8.3。若每个基数位使用计数排序,则时间$O(d(n+k))$,d为位数,k为基数。
	\end{description}
\end{description}

另外,bogosort(猴子排序)通过随机洗牌来排序,时间复杂度无上限,仅供了解。一个笑话说:量子计算机能够以 O(n) 的复杂度更有效地实现Bogo排序。


\subsection{空间复杂度}
总之,如果不是就地排序,则至少需要$O(N)$的额外空间,如归并排序。快速排序为就地排序,但需要$O(logN)$的空间。
归并排序的主要缺点就是额外空间问题。

\subsection{稳定性}
\cite{weijipedia}
1.稳定的排序
\begin{enumerate}
	\item 冒泡排序, 鸡尾酒排序
	\item 插入排序
	\item 归并排序
	\item 二叉树排序
	\item 鸽巢排序
	\item 桶排序
	\item 计数排序
	\item 基数排序
	\item Gnome sort
	\item 图书馆排序
	\end{enumerate}

	2.不稳定的排序

	\begin{enumerate}
	\item 选择排序
	\item 希尔排序
	\item Comb sort
	\item 组合排序
	\item 堆排序
	\item Smoothsort
	\item 快速排序
	\item Introsort
	\item Patience sort
    \end{enumerate}

\cite{sedgewick}习题2.5.18提出了强制稳定性概念,比如用额外的字段记录每个数据的位置,在不稳定排序操作之后用此字段恢复原来的相对次序。
原地分区版本的快速排序算法是不稳定的。其他变种是可以通过牺牲性能和空间来维护稳定性的\cite{weijipedia}。

综上,基本上可以认为,如果排序算法涉及到非相邻元素间的互换(冒泡算法是相邻互换),那么就是不稳定的。


\subsection{外部排序}
主要算法\cite{wikipedia}:
\begin{itemize}
    \item 
        \textbf{外部合并排序}\cite{wikipedia}.涉及的数量概念是pass(取决于算法设计,不宜过多)和way(取决与数据量与RAM空间比值)。pass是步骤数,pass1将子文件排序并存放于way路临时文件,pass2进行多路合并。如果way过多,pass2会因磁盘寻道频繁而效率低下。此时可增加pass数,pass2合并way路为更少的路数,pass3将现有路数合并成1路。
    \item 
	\textbf{外部桶排序},基于分布的排序。桶排序和合并排序具有数学上的对偶性\cite{weijipedia}。k趟算法可以在kn的时间开销和\verb|n/k|空间开销内完成对最多n个小于n的无重复正整数的排序(选自\cite{pp}1.5答案,每趟使用位图法进行排序)。
\end{itemize}
\cite{pp}1.3的归并排序、多趟排序分别指这里的外部合并排序和外部桶排序。

此外,还有一些不需要临时文件的原地算法。 Merge sort is used in external sorting; heapsort is not. Locality of reference is the issue.



\subsection{对整数排序}
\cite{wikipedia}
题目中研究的对象如果是整数,可以在其值域做文章。计算机中的整数取值空间有限且可数。
主要算法:
\begin{itemize}
    \item 
van Emde Boas tree,\cite{ita}第3版第20章,\cite{wikipedia}。
    \item 
non-conservative "packed sorting" algorithm
    \item 
	位图排序,要求key可表示为整数且互异。bit在bit序列中的位置代表整数的值,而bit的值代表存在性。如果整数出现多次\cite{pp}1.6.6,或者需要统计其他属性(而不仅仅是存在性),可以考虑用多个bit位表示每个整数,bit值代表该属性。如\cite{pp}1.6.8,美国免费电话号(800.887.888)存储问题,可以用3个bit分别表示以800,887,888前缀的特定7位号码是否已经存在\cite{self}。
    \item 
Bucket sort, counting sort, radix sort
\end{itemize}
