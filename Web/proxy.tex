%!Mode:: "TeX:UTF-8"
\section{代理服务器}

\subsection{反向代理}
反向代理(Reverse Proxy)方式是指以代理服务器来接受internet上的连接请求,然后将请求转发给内部网络上的服务器,并将从服务器上得到的结果返回给internet上请求连接的客户端,此时代理服务器对外就表现为一个服务器。

安全反向代理有许多用途:
可以提供从防火墙外部代理服务器到防火墙内部安全内容服务器的加密连接;
可以允许客户机安全地连接到代理服务器,从而有利于安全地传输信息(如信用卡号)。


正向、反向代理区别:
\begin{description}
\item[用途]正向代理的典型用途是为在防火墙内的局域网客户端提供访问Internet的途径。正向代理还可以使用缓冲特性减少网络使用率。反向代理的典型用途是将防火墙后面的服务器提供给Internet用户访问。反向代理还可以为后端的多台服务器提供负载平衡,或为后端较慢的服务器提供缓冲服务。另外,反向代理还可以启用高级URL策略和管理技术,从而使处于不同web服务器系统的web页面同时存在于同一个URL空间下。
\item[安全性]正向代理允许客户端通过它访问任意网站并且隐藏客户端自身,因此你必须采取安全措施以确保仅为经过授权的客户端提供服务。
反向代理对外都是透明的,访问者并不知道自己访问的是一个代理。
\end{description}

\subsection{翻墙}
长城防火墙(The Great Firewall)屏蔽网站有以下几种方式:
\begin{enumerate}
\item DNS污染
\item 特定IP地址屏蔽
\item TCP连接重置 
\item 间歇性完全封锁
\end{enumerate}

用户和代理服务器之间建立VPN来传输加密数据,数据又通过代理转发给目的服务器,从而实现翻墙访问。


\subsection{匿名}

代理服务器根据匿名程度区分:
\begin{description}
\item[高度匿名代理]高度匿名代理会将我们的数据包原封不动的转发,在服务端看来就好像真的是一个普通客户端在访问,而记录的IP是代理服务器的IP。
\item[普通匿名代理]普通匿名代理会在数据包上做一些改动,服务端上有可能发现这是个代理服务器,也有一定几率追查到你的真实IP。
代理服务器通常会加入的HTTP头有HTTP\_VIA和HTTP\_X\_FORWARDED\_FOR。
\item[透明代理]透明代理不但改动了我们的数据包,还会告诉服务器你的真实IP。这种代理除了能用缓存技术帮你提高浏览速度,能用内容过滤提高你的安全性之外,并无其他显著作用。(最常见的例子是:内网中的硬件防火墙)。透明代理的意思是客户端根本不需要知道有代理服务器的存在,它改编你的request fields(报文),并会传送真实IP,多用于路由器的NAT转发中。注意,加密的透明代理则是属于匿名代理,意思是不用设置使用代理了,例如Garden 2程序。
\item[间谍代理]间谍代理指组织或个人创建的,用于记录使用者传输的数据,然后进行研究、监控等目的代理服务器。
\end{description}
















