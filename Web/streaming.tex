%!Mode:: "TeX:UTF-8"
\section{流媒体防盗链技术}

《CDN技术详解》一书6.4.6节列举出了7种流媒体防盗链技术:

\begin{enumerate}
\item HTTP Referer检查
\item 验证登陆信息。登陆Session ID在Cookie中,也可将该登陆信息提取出来植入URL。这个方法不适合免登陆观看的视频内容。注意防盗链不是限制用户的获取,而是限制用户在其他网站获取。
\item Cookie中预先存入验证信息,比如一个字符串或数字
\item 使用Post方法下载。将下载链接换成一个表单和一个按钮,用户点击按钮。服务器验证是否是Post方法。\textit{我的理解是,点击URL后返回的是一个表单。盗链网站没法对表单进行手动处理。}
\item 使用图形验证码。\textit{我的理解是,同上一条类似,盗链系统无法做到让用户进行手动操作}。
\item 使用动态密钥。点击资源后,服务器计算一个临时的key,植入URL。用户再利用该URL进行访问。\textit{那么这个URL怎么返回给用户?通过302?盗链网站处理不了302?亦或是初次点击网址时返回的是Javascript代码?}
\item 在内容中插入随机字节。比如mp3的tag区,rar、zip的备注区,这样整个文件的哈希值会改变。其实是上一条方法的特例,即URL植入的临时key是哈希值。这个key既能用于合法性校验,又能进行哈希校验。\textit{服务器为什么要执行哈希校验?这不是客户端干的事情吗?}
\end{enumerate}


蓝汛HPCC实现了通用时间戳防盗链模块,我的理解是,URL中的时间戳那段每隔一段时间就更新一次,盗链网站无法做到及时更新。
更改URL意味着同时改变前端显示的URL和服务器记录的URL。客户端看到的URL未必是写死在html中的,可能是根据时间算出来的,为动态URL。
合法网站能够动态计算URL,加入时间戳,而非法网站只是点击一个死的URL。同理,服务器也不存一个死的url,也只是验证用户发来的URL。

湖南卫视要求蓝汛实现的防盗链算法中,需要在URL中植入时间戳和一段key,这个key是对路径、时间戳和密钥的哈希。密钥是一段简单的字符串,几乎是英文。如何把密钥传给客户端?太不安全。可能需要在点播时另外再建立一个连接,将原始URL(其实就是告诉服务端我想看哪个视频)发给服务器,服务器在服务端算好密钥然后生成新的URL返回给客户端,客户端用新的URL而非原始URL进行访问。但湖南卫视并未让蓝汛计算这个key,只让他验证这个key,说明这个key是在客户端计算的。



搜狐的服务端防盗链算法中,要求如果发现包含某个非法字符串,就返回404。我猜测其中原理是写死的网址都是包含这个非法串的,但是如果是在合法的网页下点击,实际发出的URL会去掉这个非法串。


综上,防盗链主要用到三种思路:
\begin{enumerate}
\item 检查盗链网站的HTTP referer。据说网页播放器不会传递Referer,不能使用这个手段。
\item 检查盗链网站的Cookie。不适合开放的资源。
\item 假定盗链网站不支持某些操作,如输入图形验证码,点击表单页的按钮,redirect到新的URL(?),这些操作只有用户+浏览器可以执行。\textit{如果盗链网站硬是模仿浏览器行为,也没办法,只能在于减少而非杜绝盗链}
\item 动态网页生成动态URL。不属于《CDN技术详解》上的七条。\textit{动态URL的生成原理是使用了JS?}
\end{enumerate}


