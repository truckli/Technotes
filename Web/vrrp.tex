%!Mode:: "TeX:UTF-8"
\section{VRRP}
VRRP(Virtual Router Redundancy Protocol)协议是用于实现路由器冗余的协议,最新协议在RFC3768中定义,原来的定义RFC2338被废除,新协议相对还简化了一些功能。

VRRP协议是为消除在静态缺省路由环境下的缺省路由器单点故障引起的网络失效而设计的主备模式的协议,使得在发生故障而进行设备功能切换时可以不影响内外数据通信,不需要再修改内部网络的网络参数。VRRP协议需要具有IP地址备份,优先路由选择,减少不必要的路由器间通信等功能。

 VRRP协议将两台或多台路由器设备虚拟成一个设备,对外提供虚拟路由器IP(一个或多个),而在路由器组内部,如果实际拥有这个对外IP的路由器如果工作正常的话就是MASTER,或者是通过算法选举产生,MASTER实现针对虚拟路由器IP的各种网络功能,如ARP请求,ICMP,以及数据的转发等;其他设备不拥有该IP,状态是BACKUP,除了接收MASTER的VRRP状态通告信息外,不执行对外的网络功能。当主机失效时,BACKUP将接管原先MASTER的网络功能。

 配置VRRP协议时需要配置每个路由器的虚拟路由器ID(VRID)和优先权值,使用VRID将路由器进行分组,具有相同VRID值的路由器为同一个组,VRID是一个0~255的正整数;同一组中的路由器通过使用优先权值来选举MASTER,优先权大者为MASTER,优先权也是一个0~255的正整数。

 VRRP协议使用多播数据来传输VRRP数据,VRRP数据使用特殊的虚拟源MAC地址发送数据而不是自身网卡的MAC地址,VRRP运行时只有MASTER路由器定时发送VRRP通告信息,表示MASTER工作正常以及虚拟路由器IP(组),BACKUP只接收VRRP数据,不发送数据,如果一定时间内没有接收到MASTER的通告信息,各BACKUP将宣告自己成为MASTER,发送通告信息,重新进行MASTER选举状态。

MASTER选举过程:\\
如果对外的虚拟路由器IP就是路由器本身配置的IP地址的话,该路由器始终都是MASTER;
否则如果不具备虚拟IP的话,将进行MASTER选举,各路由器都宣告自己是MASTER,发送VRRP通告信息;
如果收到其他机器的发来的通告信息的优先级比自己高,将转回BACKUP状态;
如果优先级相等的话,将比较路由器的实际IP,IP值较大的优先权高;
不过如果对外的虚拟路由器IP就是路由器本身的IP的话,该路由器始终将是MASTER,这时的优先级值为255。

VRRP协议状态比较简单,就三种状态,初始化,主机,备份机。

\begin{verbatim}
                      +---------------+
           +--------->|               |<-------------+
           |          |  Initialize   |              |
           |   +------|               |----------+   |
           |   |      +---------------+          |   |
           |   |                                 |   |
           |   V                                 V   |
   +---------------+                       +---------------+
   |               |---------------------->|               |
   |    Master     |                       |    Backup     |
   |               |<----------------------|               |
   +---------------+                       +---------------+
\end{verbatim}
   
   \begin{description}
   \item[初始化]路由器启动时,如果路由器的优先级是255(最高优先级,路由器拥有路由器地址),要发送VRRP通告信息,并发送广播ARP信息通告路由器IP地址对应的MAC地址为路由虚拟MAC,设置通告信息定时器准备定时发送VRRP通告信息,转为MASTER状态;
否则进入BACKUP状态,设置定时器检查定时检查是否收到MASTER的通告信息。
\item[主(master)机]主机状态下的路由器要完成如下功能:
 设置定时通告定时器;
 用VRRP虚拟MAC地址响应路由器IP地址的ARP请求;
 转发目的MAC是VRRP虚拟MAC的数据包;
 如果是虚拟路由器IP的拥有者,将接受目的地址是虚拟路由器IP的数据包,否则丢弃;
 当收到shutdown的事件时删除定时通告定时器,发送优先权级为0的通告包,转初始化状态;
 如果定时通告定时器超时时,发送VRRP通告信息;
 收到VRRP通告信息时,如果优先权为0,发送VRRP通告信息;否则判断数据的优先级是否高于本机,或相等而且实际IP地址大于本地实际IP,设置定时通告定时器,复位主机超时定时器,转BACKUP状态;否则的话,丢弃该通告包;
\item[备机] 备机状态下的路由器要实现以下功能:
 设置主机超时定时器;
 不能响应针对虚拟路由器IP的ARP请求信息;
 丢弃所有目的MAC地址是虚拟路由器MAC地址的数据包;
 不接受目的是虚拟路由器IP的所有数据包;
 当收到shutdown的事件时删除主机超时定时器,转初始化状态;
 主机超时定时器超时的时候,发送VRRP通告信息,广播ARP地址信息,转MASTER状态;
 收到VRRP通告信息时,如果优先权为0,表示进入MASTER选举;否则判断数据的优先级是否高于本机,如果高的话承认MASTER有效,复位主机超时定时器;否则的话,丢弃该通告包;
\end{description}


当内部主机通过ARP查询虚拟路由器IP地址对应的MAC地址时,MASTER路由器回复的MAC地址为虚拟的VRRP的MAC地址,而不是实际网卡的MAC地址,这样在路由器切换时让内网机器觉察不到;而在路由器重新启动时,不能主动发送本机网卡的实际MAC地址。如果虚拟路由器开启的ARP代理(proxy\_arp)功能,代理的ARP回应也回应VRRP虚拟MAC地址。

VRRP应用举例:\\

\begin{verbatim}
            +-----------+      +-----------+
            |   Rtr1    |      |   Rtr2    |
            |(MR VRID=1)|      |(BR VRID=1)|
            |(BR VRID=2)|      |(MR VRID=2)|
    VRID=1  +-----------+      +-----------+  VRID=2
    IP A ---------->*            *<---------- IP B
                    |            |
                    |            |
  ------------------+------------+-----+--------+--------+--------+--
                                       ^        ^        ^        ^
                                       |        |        |        |
                                     (IP A)   (IP A)   (IP B)   (IP B)
                                       |        |        |        |
                                    +--+--+  +--+--+  +--+--+  +--+--+
                                    |  H1 |  |  H2 |  |  H3 |  |  H4 |
                                    +-----+  +-----+  +--+--+  +--+--+
     Legend:
              ---+---+---+--  =  Ethernet, Token Ring, or FDDI
                           H  =  Host computer
                          MR  =  Master Router
                          BR  =  Backup Router
                           *  =  IP Address
                        (IP)  =  default router for hosts
\end{verbatim}

这是通常VRRP使用拓扑,两台路由器运行VRRP互为备份,路由器1作为VRID组1的MASTER,IP地址A,VRID组2的BACKUP,路由器2作为VRID组2的MASTER,IP地址B,VRID组1的BACKUP,内部网络中一部分机器的缺省网关地址是IP地址A,一部分是IP地址B,正常情况下以A为网关的数据将走路由器1,以B为网关的数据将走路由器2,如果一台路由器发生故障,所有数据将走另一台路由器。











