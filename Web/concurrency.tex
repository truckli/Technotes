%!Mode:: "TeX:UTF-8"
\section{并发架构}
我们在设计一个服务器的软件架构的时候,通常会考虑几种架构:
多进程(隔离性,在linux下面的服务程序广泛采用,比如大名鼎鼎的apache)、
多线程(这种模型在windows下面比较常见)、
非阻塞/异步IO(callback) 以及Coroutine模型。

网络服务在处理数以万计的客户端连接时,往往出现效率低下甚至完全瘫痪,这被称为C10K问题。

多进程和多线程都有资源耗费比较大的问题,所以在高并发量的服务器端使用并不多。
这里我们重点来研究一下两种架构,基于callback和coroutine的架构。




\subsection{Callback- 非阻塞/异步IO}
这种架构的特点是使用非阻塞的IO,这样服务器就可以持续运转,而不需要等待,可以使用很少的线程,即使只有一个也可以。需要定期的任务可以采取定时器来触发。把这种架构发挥到极致的就是node.js,一个用javascript来写服务器端程序的框架。在node.js中,所有的io都是non-block的,可以设置回调。

这种架构的缺点是编程复杂,把以前连续的流程切成了很多片段。另外也不能充分发挥多核的能力。
\subsection{Coroutine-协程}
coroutine本质上是一种轻量级的thread,它的开销会比使用thread少很多。多个coroutine可以按照次序在一个thread里面执行,一个coroutine如果处于block状态,可以交出执行权,让其他的coroutine继续执行。使用coroutine可以以清晰的编程模型实现状态机。

协程和一般多线程的区别是,一般多线程由系统决定该哪个线程执行,是抢占式的,而协程是由每个线程自己决定自己什么时候不执行,并把执行权主动交给下一个线程。 协程是用户空间线程,操作系统其存在一无所知,所以需要用户自己去做调度,用来执行协作式多任务非常合适。

线程和协同程序的主要不同在于:在多处理器情况下,多线程程序同时运行多个线程;而协同程序是通过协作来完成,在任一指定时刻只有一个协同程序在运行,并且这个正在运行的协同程序只在必要时才会被挂起。这样Lua的协程就不能利用现在多核技术了。

Lua 协程有三个状态:suspended,running,dead。可以通过coroutine.status来查看协程状态。


让我们看看Lua语言的coroutine的例子。
\begin{verbatim}
> function foo()
 >>   print("foo", 1)
 >>   coroutine.yield()
 >>   print("foo", 2)
 >> end
> co = coroutine.create(foo) -- create a coroutine with foo as the entry
> = coroutine.status(co)
 suspended
 > = coroutine.resume(co) <--第一次resume
 foo     1
 > = coroutine.resume(co) <--第二次resume
 foo     2
 > = coroutine.status(co)
 dead
\end{verbatim}
Google go语言也对coroutine使用了语言级别的支持,使用关键字go来启动一个coroutine(从这个关键字可以看出Go语言对coroutine的重视),结合chan(类似于message queue的概念)来实现coroutine的通讯,实现了Go的理念 ”Do not communicate by sharing memory; instead, share memory by communicating.”。一个例子:
\begin{verbatim}
func ComputeAValue(c chan float64) {
     // Do the computation.
     x := 2.3423 / 534.23423;
     c <- x;
 }
func UseAGoroutine() {
    channel := make(chan float64);
    go ComputeAValue(channel);
    // do something else for a while
    value := <- channel;
    fmt.Printf("Result was: %v", value);
 }
\end{verbatim}


在业务流程实现上,coroutine确实是更理想的实现,基于callback的风格,代码确实不是那么清晰。
但是现实世界中,coroutine到目前为止并没有真正流行起来,第一,是因为支持的语言并不是很多,比较新的语言(python/lua/go/erlang)才支持,但是老一些的语言(java/c/c++)并没有语言级别的支持。第二个原因是因为coroutine的使用可能让一些糟糕的代码占用过多的内存,而且比较难于排查。另外在实现一个工作流的构架中,流的暂停和恢复的时机都是未知的,系统的状态并不能放在内存中存放,都必须序列化,所以coroutine本身要提供序列化的机制,才可以实现稳定的系统。我想这些就是coroutine叫好不叫座的原因。

\subsection{Windows IOCP}
IOCP全称I/O Completion Port,中文译为I/O完成端口。IOCP是一个异步I/O的API,它可以高效地将I/O事件通知给应用程序。与使用select()或是其它异步方法不同的是,一个套接字[socket]与一个完成端口关联了起来,然后就可继续进行正常的Winsock操作了。然而,当一个事件发生的时候,此完成端口就将被操作系统加入一个队列中。然后应用程序可以对核心层进行查询以得到此完成端口。

\subsection{事件驱动的缺点}
不管是Nginx还是Squid这种反向代理,其网络模式都是事件驱动。事件驱动其实是很老的技术,早期的select、poll都是如此。后来基于内核通知的更高级事件机制出现,如libevent里的epoll,使事件驱动性能得以提高。事件驱动的本质还是IO事件,应用程序在多个IO句柄间快速切换,实现所谓的异步IO。事件驱动服务器,最适合做的就是这种IO密集型工作,如反向代理,它在客户端与WEB服务器之间起一个数据中转作用,纯粹是IO操作,自身并不涉及到复杂计算。

Apache或者Resin这类应用服务器,之所以称他们为应用服务器,是因为他们真的要跑具体的业务应用,如科学计算、图形图像、数据库读写等。它们很可能是CPU密集型的服务,事件驱动并不合适。例如一个计算耗时2秒,那么这2秒就是完全阻塞的,什么event都没用。想想MySQL如果改成事件驱动会怎么样,一个大型的join或sort就会阻塞住所有客户端。这个时候多进程或线程就体现出优势,每个进程各干各的事,互不阻塞和干扰。当然,现代CPU越来越快,单个计算阻塞的时间可能很小,但只要有阻塞,事件编程就毫无优势。所以进程、线程这类技术,并不会消失,而是与事件机制相辅相成,长期存在。

\subsection{多核系统}

Squid-3.3使用wokers 支持基本的多核,管理员可以通过配置启动一个squid来派生多个worker进程利用所有可利用的CPU。
Nginx也采用类似机制,通过多个worker进程绑定到CPU上。























