%!Mode:: "TeX:UTF-8"
\section{Nginx}
Nginx(发音同engine x)是一款由俄罗斯程序员Igor Sysoev所开发轻量级的网页服务器、反向代理服务器以及电子邮件(IMAP/POP3)代理服务器。起初是供俄国大型的门户网站及搜索引擎Rambler使用。此软件BSD-like协议下发行,可以在UNIX、GNU/Linux、BSD、Mac OS X、Solaris,以及Microsoft Windows等操作系统中运行。

Nginx是一款面向性能设计的HTTP服务器,相较于Apache、lighttpd具有占有内存少,稳定性高等优势。与旧版本(<=2.2)的Apache不同,nginx不采用每客户机一线程的设计模型,而是充分使用异步逻辑,削减了上下文调度开销,所以并发服务能力更强。整体采用模块化设计,有丰富的模块库和第三方模块库,配置灵活。 在Linux操作系统下,nginx使用epoll事件模型,得益于此,nginx在Linux操作系统下效率相当高。同时Nginx在OpenBSD或FreeBSD操作系统上采用类似于epoll的高效事件模型kqueue。

\subsection{阶段}
Nginx将HTTP处理分为11个阶段(Phase):
\begin{description}
\item[PostRead]接收完请求头之后的第一个阶段,它位于uri重写之前,实际上很少有模块会注册在该阶段,默认的情况下,该阶段被跳过
\item[ServerRewrite]server级别的uri重写阶段,也就是该阶段执行处于server块内,location块外的重写指令
\item[FingConfig]重写URL后找到匹配的location块,其原理是从location组成的静态二叉树中快速检索。处理方法由官方提供不可更改。
\item[Rewrite]location级别的uri重写阶段,该阶段执行location基本的重写指令,也可能会被执行多次。同ServerRewrite完全相同,其checker函数也是\verb$ngx_http_core_rewrite_phase$。
\item[PostRewrite]location级别重写的后一阶段,用来检查上阶段是否有uri重写,并根据结果跳转到合适的阶段,并限制重写URL次数,防止死循环。处理方法由官方提供不可更改。
\item[PreAccess]访问权限控制的前一阶段,该阶段在权限控制阶段之前,一般也用于访问控制,比如限制访问频率,链接数等
\item[Access]访问权限控制阶段,比如基于ip黑白名单的权限控制,基于用户名密码的权限控制等
\item[PostAccess]访问权限控制的后一阶段,该阶段根据权限控制阶段的执行结果进行相应处理
\item[TryFiles]\verb$try_files$指令的处理阶段,如果没有配置\verb$try_files$指令,则该阶段被跳过。\verb$try_files$作用类似于rewrite,用于配置所请求内容的路径选择规则
\item[Content]内容生成阶段,该阶段产生响应,并发送到客户端
\item[Log]日志记录阶段,该阶段记录访问日志
\end{description}

在\verb$ngx_lua$中,有init, rewrite, access, set, content, log等阶段。

一般而言handler返回:
\begin{description}
\item [NGX\_OK]表示该阶段已经处理完成,需要转入下一个阶段
\item [NG\_DECLINED]表示需要转入本阶段的下一个handler继续处理
\item [NGX\_AGAIN, NGX\_DONE]表示需要等待某个事件发生才能继续处理(比如等待网络IO),此时Nginx为了不阻塞其他请求的处理,必须中断当前请求的执行链,等待事件发生之后继续执行该handler
\item [NGX\_ERROR]表示发生了错误,需要结束该请求
\end{description}


