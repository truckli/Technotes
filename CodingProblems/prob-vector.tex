%!Mode:: "TeX:UTF-8"

\section{向量分析问题举例}

\subsection{数组中最大的K个数}
参\cite{bop}2.5, \cite{ita}9.2-9.3(\cite{ita}称找到第k小的数为kth order statistics)。
方法零:全部排序,$O(NlogN)$, 适用于多次查询场景;或只对前k个数部分排序,$O(nk)$,可使用利用选择或交换的排序算法\cite{subsec:sortAthmClass}。
方法一:随机分割保留单侧。\cite{ita}9.2称之为RANDOMIZED-SELECT算法, \cite{pp}11.5.9和\cite{wikipedia}称之为选择算法。期望运行时间为线性,最差为平方时间。\cite{ita}9.3的SELECT算法通过精心挑选pivot来确保最差时间也为线性。涉及随机访问,不适合外存储操作。\cite{bop}2.5.3认为线性算法常数项太大,未必好。
方法二:值域二分查找。每次迭代需要线性时间,迭代次数为$log_{2}(V_{max}-V_{min}/\delta)$,$\delta$为元素间最小间隔(对于整数,$\delta$为1)。对于均匀分布的数据,总时间为NlogN。文件操作优化(\cite{bop}2.5.3):每次迭代,将当期搜索区间的样本复制到新的文件中(\cite{self}:磁盘可能频繁换道,但适合于两个磁盘或磁带,参\ref{subsec:dupNumberInFile}),当区间内的样本数能够全部载入内存时,则不需要再进行文件操作。\cite{self}认为,为减少磁盘换道,对文件的读取采用搅拌式。
方法三:前K个元素生成容量为K的大顶堆,从第K+1个元素开始遍历,更新堆的内容。时间为$O(NlogK)$适合外存储操作。如果K个数可以全部载入内存,则只遍历文件一遍,不涉及随机访问。如果K也很大,需要将k分割为多个可以载入内存的部分(\cite{bop}2.5.4)。
\label{subsec:orderStatistics}

方法四:如果值域有限且非常小,计数排序(\cite{bop}2.5.5)。

\subsection{找到数组中最大数和最小数}
\cite{bop}2.10。如分别独立找到最大值和最小值,需要2N次比较。其他方法可以达到1.5N次比较。
方法一:每邻近的一对数为一组,组内比较,较小者调整到奇数位置。进而分别在奇数位置和偶数位置寻找最小和最大。
方法二:每邻近的一对数为一组,遍历每个组,每组内用3次比较更新当前的最大值和最小值。
方法三:分治法。
其时间复杂度见:\ref{DivideComplex}。



\subsection{数组相邻差值最大值}
\cite{bop}2.11扩展1\\
所述相邻指值的相邻,并非数组中位置的相邻。
根据抽屉原理,相邻差值的最大值应不小于delta=(Vmax-Vmin)/(N-1)。将取值域分解为长为delta的若干桶,每隔桶内不会有我们希望的值。记录每个桶的最大值和最小值,然后遍历所有的桶即可。时间和空间复杂度都是线性。



\subsection{找出数组中和为给定值S的两个数}
\cite{bop}2.12, \cite{ms100}14\\
方法一:排序,对数组中每个数A,用二分法查找S-A是否在数组中。$O(NlogN)$。方法二:建立Hash表,使得用$O(1)$时间即可求出某个值是否在数组中,对数组中每个数A,查找S-A是否在数组中。时间与空间复杂度均为线性。方法三:双下标相向游动.先排序,然后从数组的头和尾分别寻找一个数,使其和为S。查找时间为线性。书中提到,很多题目要求返回两个数组下标的,适用类似方法,即在一个循环体里用两个变量反向遍历。

\subsection{子向量最大和}
\cite{bop}2.14, \cite{pp}第8章整章讲述该问题。\label{problem:maxsubvector}\\
方法零:枚举所有子向量$O(N^2)$,计算其和, 总时间$O(N^3)$。若将求和与枚举操作的第二个循环合并,或者预先计算累加数组,总时间$O(N^2)$。方法二:分治法,分别在两个砍半数组中求解局部最优,再求解跨界解,给定两个序列头部,分别向前、向后搜索合适的子数组尾部。总时间$O(NlgN)$。方法三:动态规划,寥寥数行代码,线性时间复杂度。遍历数组,更新已遍历部分以当前位置为结尾的最优解,和已遍历部分最优解,后者的更新依赖于前者。方法四:\cite{self}为便于理解方法三而创,全局最优解必为以某i为结尾的子数组,先用线性的时间和空间求出以各i为结束的子数组的最大值序列maxEnd[i],maxEnd[k]依赖于maxEnd[k-1]。再用线性时间找到maxEnd[i]数组中的最大值即可。方法五:线性时间分治法\cite{pp}8.7.8。分别在两个砍半数组中求出局部最优和跨界最优,则总的跨界最优借即为两个砍半数组跨界最优解之和。
\label{subsec:maxsubsum}

\subsection{总和最接近0的子向量}
\cite{pp}8.7.10,是\ref{problem:maxsubvector}的延伸。使用累加数组,$cum[-1]=0,cum[i]=\sum_{k=0}^{k=i}{x[i]}$,转化为求cum[n]中最接近元素的问题。可以在$O(NlogN)$时间内通过排序来完成。似乎可以仿照原题对动态规划算法,将评价指标更改即可。\label{problem:zerosubvec}

\subsection{总和最接近常数t的子向量}
\cite{pp}8.7.10,是\ref{problem:maxsubvector}的延伸。书上未给出答案。如果仿照\ref{problem:zerosubvec}使用累加数组,转化为求cum[n]中差值最接近t的问题。可以在$O(NlogN)$时间完成排序,通过$O(NlogN)$时间在有序数组中完成查找。似乎可以仿照原题对动态规划算法,将评价指标更改即可。

\subsection{环形数组子向量最大和}
\cite{bop}2.14扩展1\\
将解空间分为两部分,一部分不过尾,参考\ref{subsec:maxsubsum}。另一部分过尾。过尾者,分别从头和尾按照两个方向求出最优子序列。书上如此似乎未能妥善处理两个最优子序列重叠的情况。我认为应该求出数组总和,减去子向量长度最小值。

\label{subsec:maxroundsubsum}

\subsection{二维数组子向量最大和}
\cite{bop}2.15\\
方法零:枚举,$O(N^2 \cdot M^2) \cdot \textrm{二维求和复杂度}$,通过部分和缓存方法,可以将二维求和复杂度做到$O(1)$。所谓部分和就是从某元素到原点的连线为对角线的矩形区域进行求和。方法二:只在Y维度进行枚举,确定上下边,按照一维的方法(参\ref{subsec:maxsubsum})确定左右边,求出子数组之和,$O(NM \cdot min(N,M))$。按照这两种方法,可以将问题扩展为三维、四维。参考\ref{subsec:maxroundsubsum} ,可以求解首位相连情形。


\subsection{最长递增子序列(LIS问题)}
\begin{verbatim}
http://en.wikipedia.org/wiki/Longest_increasing_subsequence
\end{verbatim}
\cite{bop}2.16\\
同\ref{subsec:maxsubsum}节不同,此处序列不必连续。
此题如使用枚举方法,子序列共有$2^N$个,不合适。方法一:动态规划,考虑各序列末尾元素的位置。记LIS[i]为以i为末尾的最长递增子序列(书上说是前i个元素中的最长子序列),至少为1。$LIS[i]=max\{1,LIS[k]\}, k<i, array[k]<array[i]$。最终返回$max{LIS[i]}$,时间复杂度为$O(N^2)$,需存储LIS数组。方法二:动态规划,考虑各序列的长度和末尾元素的值。遍历数组,存储$TAIL[j]$为当前(在遍历过程中更新)时刻所有长度为j的递增序列的尾值的最小值(最佳值,因为尾值越小,新加入的元素越有可能参与本条递增序列),MAXL为当前的最长递增序列的长度(即当前数据集的最优解),所有的递增链的长度都不会超过该值,TAIL数组的有效坐标也不应超过该值。TAIL数组必定是单调增的,假设新加入的元素(位于array[i])的值介于TAIL[j]和TAIL[j+1]之间,意味着新元素可以追加到长度为j的序列上,形成更优的长度为j+1的序列,那么可以递推出LIS[i] = j+1, 且TAIL[j+1]=array[i]。时间复杂度$O(N^2)$。方法三:思路同方法二,但在选取j时,从LIS[i-1]开始,并使用二分查找。利用了TAIL为递增数组,及$LIS[i] \le LIS[i-1]+1$。本节及\ref{subsec:maxsubsum}启发我们,数组不仅可看作静态的数据集合,也可看作一个更大的动态增长的数据集的组成部分。

\subsection{数组均匀分割}
\cite{bop}2.18,要求长度为2N的正整数数组分成两个长度为N的数组,且和最接近\\记数组为array[k], 和的一半为S。
方法零:枚举法的复杂度$\left( \begin{array}{c} N \\ 2N \end{array} \right) = \frac{(2N)!}{N!N!}$,找出最接近S的组合。
    方法一:动态规划,遍历数组,当前下标记作$k$,更新集合类$V_i,i \le k$, $V_i$表示任意i个元素的和的可取值。最终研究$V_N$中的最优值。书中未提及如何据此找到对应分配方案。可能需要在$V_i$中添加附加信息。复杂度为$O(2^N)$。
方法二:如果取值域很小,用boolean数组isOK[i][s]表示是否能存在i个变量其和为s。三维遍历k,i,s, $i \le N,i \le k, s \le S$, 如果isOk[i-1][s-array[k]]为真,则isOk[i][s]为真。最终得到了isOk[N][]数组。复杂度$O(N^2 S)$。适用于$S$较小的情形。扩展问题是如果数组中有负数当如何。我想,将数组部分和取值空间映射为从零开始的正整数作为isOk[][]第二维度即可继续采用方法二。

\subsection{有序数组循环移位后求最小值}

大部分情形下可用在$O(\lg n)$时间二分缩进查找,但\textbf{有些特殊数组无法二分缩进}(如首、尾、中值相同,无法区分{1,1,1,0,1}和{1,0,1,1,1}还是{1,1,1,1,1}),此时可改用线性查找。
另外,(可选地)先判断数组如仍是有序的(A[1] < A[n]),可直接返回首值。


\subsection{约瑟夫环问题}
\cite{ms100}18,\cite{sword}45。圈有n个结点,击鼓传花式不断删除第m个结点。记该问题为f(n,m),有 $f(n,m)=[f(n-1,m)+m]\bmod n$

