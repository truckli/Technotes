%!Mode:: "TeX:UTF-8"

\section{数论与数字问题举例}
本节中的问题涉及非常大的数字,普通内置数据类型容易溢出,应自定义BigInt类型,可用数组或字符串实现。一般用字符‘9’表示数字9就可用了,如果为节省内存,可用uchar存储0到255之间的数字,但是这样不方便打印。如果需要打印数字,可用uchar存储0到99之间的数字,相当于100进制。

\subsection{浮点数的N次方运算}
\cite{sword}面试题11。
求$b^N$,这里限定N为整数,且不考虑大数问题。\textbf{本题容易在边界条件问题上出错,做对的人很少}。
首先如果$N<0$,则$b$不能是零(\textbf{浮点数与零的比较应采取误差判断},而不能直接用相等运算符),$b^N \gets 1.0 / b^{-N}$。
如果$N=0$,则$b$不能为零,其他$b$值直接返回1.0。如果N是正偶数,则$b^N \gets (b^{N/2})^2$,可用递归运算。如果N为正奇数,则
$b^N \gets b \cdot b^{N-1}$。

\subsection{打印N位以内的所有数字}
\cite{sword}面试题12。
首先要意识到涉及大数问题,用长度为N+1的数组构造大数,以字符形式表达每个数字。
解法一模仿数字的进位。解法二转为全排列问题,可用递归解决。



\subsection{N!十进制/二进制中末尾的0个数}
\cite{bop}2.2。\\
对于十进制,0的个数等于所有质因子中5的个数。每隔5,产生一个因子5;每隔25,又产生一个因子5;每隔125,625,\ldots,
对于二进制,类似。另外N!中质因子2的个数等于N-N的二进制表示中1的个数。
\subsection{选择M,使N*M只包含0和1}
\cite{bop}2.8。\\转化为选择符合条件的K,使得K整除N。遍历K,值域按照模划分为N份,每份只保留最小者。K从i位数变成i+1位,相当于用$10^i$加上已访问的所有数,那么,i+1位数模N的结果依次为$10^i mod N$加上已访问的数$mod N$的和再模N。直到mod值为零的集合不再为空。遍历结束的条件,书上提到循环节的概念,不甚明了。

\subsection{不使用判断和分支语句实现两个整数的最大值}
\cite{ccpppibible}317,号称华为面试题。\\
方法1:\verb$ (a+b+abs(a-b))/2$。竟然是用了库函数!

方法二:\verb$ #define IsMax(a,b) = unsigned(a-b) >> (sizeof(int)*8-1)$,
于是$max=IsMax(a,b)*a+IsMax(b,a)*b$。


\subsection{长内存块中bit1的个数}
\cite{pp}9.5.7\\
方法一:以字节或字为单位,统计每个单位中bit1的个数,再累加。百度2014年面试曾要求复杂度与字长无关,该法不适合。
方法二:查表法,对每个单位通过查表的方式获得1的个数,再累加。
方法三:统计一个单位的取值分布,对于每个值出现对次数,乘以该值对应的1的个数。

\subsection{高效实现C语言isupper函数}
\cite{pp}9.5.6\\
使用一个256元的表(可以完全纳入缓存),定义
\begin{verbatim}
#define isupper(c) (uppertable[c])
\end{verbatim}
大多数系统为表中每个元素存储几个bit,通过逻辑与操作来提取
\begin{verbatim}
#define isupper(c) (bigtable[c] & UPPER)
#define isalnum(c) (bigtable[c] & (UPPER|LOWER|DIGIT))
\end{verbatim}


\subsection{最大公约数}
\cite{bop}2.7。\label{subsec:gcd}
方法0:欧几里德算法,辗转相除.
方法1:欧几里德算法减法版本,避免除法,但增加了减法.\verb|int gcd(i,j){while(i!=j){if(i>j)i-=j;else j-=i;}}|。
方法2:gcd初设为1。若两数均even,则均右移1位,gcd左移1位;若其一为even,则右移至成odd;若均为odd,则其一赋为两者差,必为even。通过最低bit快速判断奇偶性。


\subsection{数字逆转}
即123->321,-123->-321。代码很短,不需区分正负:
\begin{verbatim}
int reverse (int x) {
        int r = 0;
        for (; x; x /= 10)
            r = r * 10 + x % 10;
        return r;
}
\end{verbatim}
关于溢出的处理是:放任。注意对INT\_MAX加1会得到INT\_MIN,且不出现任何错误(与errno无关)。


\subsection{丢失的数字}
如果一个整型数组除了一个数字之外每个数字都出现两次,如果找到丢失的数字?全部异或。
如果\textbf{除了两个数字}之外每个数字都出现了两次呢?全部异或后能得到至少一个bit为1,按该bit分割数组,子数组各自只包含一个落单数字了。
如果除了一个数字之外每个数字都\textbf{出现三次}呢?统计各bit出现频率,不为3的倍数的bit会出现在落单数字中。



















