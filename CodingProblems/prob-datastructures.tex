%!Mode:: "TeX:UTF-8"

\section{数据结构问题举例}

\subsection{$O(1)$时间内删除链表结点}
\cite{sword}13。将下一个结点的内容复制到本结点再删除下一个结点。\textbf{本题极易在边界处理上出错}。第一,如果要先判断该结点是否存在于链表,需要$O(n)$时间,\textbf{必须声明这个责任归调用者}。
第二,如果要删除的恰好是\textbf{尾结点},\textbf{则同样无法在$O(1)$时间内完成}。第三,如果要删除的是头结点,需要更改头指针。


\subsection{m个互异整数搜索结构}
\cite{pp}13介绍了有序数组,有序链表,BST,位向量,桶。
链表相对于数组,适用于待存放元素个数不明确的情形。虽然链表在插入数据时不需要移动数据,但其内存访问不连续且需要额外空间存放指针,对cache不友好,性能甚至不如数组。对于纯搜索过程,有序数组可以在log时间内完成,链表需要线性时间。桶实质是一种散列结构,用链表处理碰撞。


\subsection{实现快速返回最大值的队列}
\cite{bop}3.7。
方法零:按照传统方式,以数组或链表存储队列,两个指针指向头尾。MaxVal操作复杂度为$O(N)$,增、删复杂度为常数。个人认为可以保存最大值,增删是予以更新,Enqueue的复杂度为常数, dequeue复杂度线性, MaxVal复杂度为常数。方法一:使用堆。Enqueue和Dequeue为$O(logN)$, MaxVal为常数。类似于\cite{ita}中的优先级队列。方法二:\cite{bop}提出了一种$O(1)$时间MaxVal操作的栈结构,用链表连接不同状态时的最大值,并用两个这种栈串联实现了队列。


\subsection{有序文件多路归并}
\cite{pp}14.6.4.d。用堆(优先级队列)表示每个文件的下一个元素。从堆中选出最小元素,再从对应文件中补上。

\subsection{二叉树中最远节点}
\cite{bop}3.8。
经分析可知,最远两节点有两种情况:两个叶子节点;根节点和一个叶子节点,此时根结点左右子树其一必为空。总之,为某子树(或本树)左右子树最大高度之和(叶子结点高度为1)。可深度优先遍历,检查以各结点为根结点的子树的左右子树的高度之和,更新当前最优解。书上使用的是递归方法,要求自作非递归法。



\subsection{分层遍历二叉树}
\cite{bop}3.10。有两问。第一问是打印某一层结点。第二问是从上到下分层打印各结点,层内从左到右。扩展题是从下到上分层打印各结点,层内从左到右或从右到左。对于第一问,按照正常流程遍历树,判断层数合适则打印。其他问题,使用一个数组,将根结点压入,遍历数组,同时按照一定规则将各结点压入数组末尾。最终数组中元素排列符合要求。此题比较简单。



\subsection{稀疏矩阵存储}
\cite{pp}10.2提出用数组表示各列,指向行链表。同时提出如果编程语言不支持指针,可以把所有行依次相连为单一大数组,用另一个数组表示各列首元素在大数组中的位置。\cite{pp}10.6.2提出,依此按照x、y坐标排列各元素,以支持二分搜索。对于其他空间节省技术,经常是通过仅存储差量来实现,r如\cite{pp}10.6.4,10.6.5。
\label{problem:sparseMatrix}


