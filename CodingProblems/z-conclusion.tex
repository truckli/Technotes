%!Mode:: "TeX:UTF-8"


\section{编程问题总结}
\subsection{测试用例}
树类问题:完全二叉树,不完全二叉树,空树,单节点数,无左节点的树,五左子树的树。

本章包含了一些算法题目,几乎所有题目都可以用枚举的方法低效地完成,记作方法零。遇到题目,可以依次尝试枚举、分治和动态规划。

\subsection{大数据分析技巧}
将字符串哈希到N多小文件中以图分治,转为“小数据”问题。

如果要求统计频次,可使用hashmap、trie树;如果只要求统计存在性,可使用hashset,位图,bloom filter;如果统计重复性,可用2-Bitmap。
32位整数将近有43亿种取值,其位图需要512MB空间。


\subsection{空间节省技术}
\ref{problem:sparseMatrix}给出了稀疏矩阵存储方式。对于其他空间节省技术,经常是通过仅存储差量来实现,如\cite{pp}10.6.4,10.6.5。

\subsection{循环不变式}
二分搜索:待搜索值在l\dots u之间的区间中。

堆siftup:heap(1,n) except perhaps between i and its parent。

堆siftdown:heap(1,n) except perhaps between i and its (0,1,2) children

快速排序partition:

子数组最大和:

BST搜索:

\subsection{利用哨兵减少判断}
哨兵有两个意思\cite{wikipedia},一是结构化表示结尾的最末特殊值(字符串,文件,链表),二是用于消除末尾判断的特殊值。本节取第二个意思。

求最值:\cite{pp}9.5.8利用哨兵元素找出数组最大值,x[n]时刻保存当前最大值。 需要在末尾多分配一个元素对空间,即可访问x[n]。同时,如果在开头额外分配元素,则可访问x[-1]。

二分搜索:\cite{pp}9.3修改二分搜索算法,将x[-1]和x[n]看作假想的哨兵值(参\ref{codes:binsort})。

顺序搜索:\cite{pp}9.2问题3将哨兵值用在了顺序顺序搜索中,x[n]保存待搜索值。

有序链表和BST插入:参\ref{problem:listInsert}和\ref{problem:BSTInsert}。

堆siftup:首元素作哨兵,取极小值。

快速排序Partition:



\subsection{if-else灾难与switch膨胀}
转为为搜索问题(\cite{pp}3.7.1),或利用多态(\cite{refractor})。

\subsection{分治法时间复杂度}\label{DivideComplex}
\begin{displaymath}
    T(n)=aT(n/b)+f(n)
\end{displaymath}

\begin{math}
    \textrm{其中} a \ge 1, b > 1, f(n) \textrm{ is asymptotically positive}。
\end{math}

三种常见情况:
\begin{enumerate}
    \item 
\begin{displaymath}
    f(n)=O(n^{log_{b}a-\epsilon}), \epsilon>0, \textrm{则 }T(n)=\Theta(n^{log_{b}a})
\end{displaymath}

    \item 
\begin{displaymath}
    f(n)=O(n^{log_{b}a}lg^{k}n), \textrm{则 }T(n)=\Theta(n^{log_{b}a}lg^{k+1}n)
\end{displaymath}

    \item 
\begin{displaymath}
    f(n)=O(n^{log_{b}a+\epsilon}), \epsilon>0, \textrm{则}T(n)=\Theta(f(n))
\end{displaymath}

\end{enumerate}



作为特例,如果$a=b$,那么$f(n)=O(n^{-\epsilon})$意味着$T(n)$有线性复杂度;$f(n)=O(nlg^{k}n)$意味着$T(n)=\Theta(nlg^{k+1}n)$。














