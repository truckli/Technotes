%!Mode:: "TeX:UTF-8"

\section{大量数据分析问题举例}


\subsection{求众数}
求众数(Mode):
\cite{pp}11.5.1。
方法1:基于选择算法,参考\ref{subsec:orderStatistics},取轴值x, 根据快排的过程,小于x的放在左边,大于x的放在右边。同时统计x的出现次数T。 如果x左边的个数(不算X)多于T,向左递归;同理,如果x右边的个数对于T,向右递归。复杂度为$O(nlogn)$。
方法2:基于分布的统计,利用数组(如果数据取值范围为有限区间内的整数)或散列。时间复杂度为线性,空间复杂度较高。
方法3:先排序。复杂度主要来自排序操作。

求绝对众数:
\cite{bop}2.3,发帖水王问题。
方法:不断去除相异的两个值。
扩展为有三个ID,各自超过ID出现次数的四分之一。



\subsection{丢失的备份}
\cite{bop}1.5。每个结点都冗余备份为2份,因此活跃ID列表中每个ID会出现两次,但一些结点可能会失效,其ID不出现于列表。在已知有1个/2个结点出故障的情况下,找到其ID。如果只有1个ID失效,将所有活跃的ID相异或,将得到失效ID。如果两个ID失效,事先应计算好所有ID总和,减去活跃ID总和,可以得到一个方程。另一个方程可以基于所有ID的乘积、平方和、异或等。

\subsection{缺失的整数}
\cite{pp}2.1A,文件中包含不足40亿个32位整数(32位整数有42.9亿种取值),要求找到一个缺失的整数。如果内存无法容纳位图(32位整数需要512MB的位图空间),可以在值域二分搜索,每次迭代将当前值域的样本复制到另一个文件上,空域空间因小于值域空间,因此空域能保证缩半。根据等比数列求和公式,复杂度为线性。如果题目条件是所有32bit整数中只缺1个,可以异或。因为所有整数异或为0。参\ref{subsec:dupNumberInFile}。

\subsection{至少出现两次的整数}
\cite{pp}2.6.2,文件中包含多于43亿个32位整数(32位整数有42.9亿种取值),要求找到一个重复的整数。如果内存无法容纳位图(32位整数需要512MB的位图空间),可以在值域二分搜索,即查找小于中间值的数值是否冗余,然后值域缩半。每次迭代将当前值域的样本复制到另一个工作磁带上,但空域空间大于值域空间,不能保证空域缩半,因此最差情况下复杂度为$NlogN$。优化:(\cite{pp}2.6.2 Jim Saxe),如果值域容量为m,则只复制m+1个样本,保证空域缩半。另外(\cite{self}, 参\ref{subsec:orderStatistics}),当m足够小时,开始使用位图法或全部载入内存。另外(\cite{self}),即使空域不缩半,如果每次迭代都将空域按一个线性因子缩小,则平均时间也为线性(无限项等比级数之和)。当发现当前值域空间存在冗余时,即可丢弃当前空域位置之后的位置。这样,不需要复制额外的文件,只要更新文件搜索范围即可。进一步,对文件的访问可以采用搅拌式。

\label{subsec:dupNumberInFile}
