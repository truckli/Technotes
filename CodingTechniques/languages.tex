%!Mode:: "TeX:UTF-8"
\section{编程语言}

动态语言,是指程序在运行时可以改变其结构:新的函数可以被引进,已有的函数可以被删除等在结构上的变化。比如众所周知的ECMAScript(JavaScript)便是一个动态语言。除此之外如Ruby、Python等也都属于动态语言,而C、C++等语言则不属于动态语言。


\subsection{JIT}


即时编译(英语:Just-in-time compilation),又译及时编译、实时编译,动态编译的一种形式,是一种提高程序运行效率的方法。通常,程序有两种运行方式:静态编译与动态直译。静态编译的程序在执行前全部被翻译为机器码,而直译执行的则是一句一句边运行边翻译。

即时编译器则混合了这二者,一句一句编译源代码,但是会将翻译过的代码缓存起来以降低性能损耗。相对于静态编译代码,即时编译的代码可以处理延迟绑定并增强安全性。

即时编译器有两种类型,一是字节码翻译,二是动态编译翻译。

微软的.NET Framework,还有绝大多数的Java实现,都依赖即时编译以提供高速的代码执行。Mozilla Firefox使用的JavaScript引擎SpiderMonkey也用到了JIT的技术。Ruby的第三方实现Rubinius和Python的第三方实现PyPy也都通过JIT来明显改善了解释器的性能。

\url{http://en.wikipedia.org/wiki/Just-in-time_compilation}

In computing, just-in-time compilation (JIT), also known as dynamic translation, is compilation done during execution of a program – at run time – rather than prior to execution. Most often this consists of translation to machine code, which is then executed directly, but can also refer to translation to another format.
JIT compilation is a combination of the two traditional approaches to translation to machine code – ahead-of-time compilation (AOT), and interpretation – and combines some advantages and drawbacks of both. Roughly, JIT compilation combines the speed of compiled code with the flexibility of interpretation, with the overhead of an interpreter and the additional overhead of compiling (not just interpreting). JIT compilation is a form of dynamic compilation, and allows adaptive optimization such as dynamic recompilation – thus in principle JIT compilation can yield faster execution than static compilation. Interpretation and JIT compilation are particularly suited for dynamic programming languages, as the runtime system can handle late-bound data types and enforce security guarantees.

JIT compilation can be applied to a whole program, or can be used for certain capacities, particularly dynamic capacities such as regular expressions. For example, a text editor may compile a regular expression provided at runtime to machine code to allow faster matching – this cannot be done ahead of time, as the data is only provided at run time. Several modern runtime environments rely on JIT compilation for high-speed code execution, most significantly most implementations of Java, together with Microsoft's .NET Framework. Similarly, many regular expression libraries ("regular expression engines") feature JIT compilation of regular expressions, either to bytecode or to machine code.
A common implementation of JIT compilation is to first have AOT compilation to bytecode (virtual machine code), known as bytecode compilation, and then have JIT compilation to machine code (dynamic compilation), rather than interpretation of the bytecode. This improves the runtime performance compared to interpretation, at the cost of lag due to compilation. JIT compilers translate continuously, as with interpreters, but caching of compiled code minimizes lag on future execution of the same code during a given run. Since only part of the program is compiled, there is significantly less lag than if the entire program were compiled prior to execution.

