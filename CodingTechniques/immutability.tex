%!Mode:: "TeX:UTF-8"
\section{对象生命管理}

\subsection{Persistent数据结构}
In computing, a persistent data structure is a data structure that always preserves the previous version of itself when it is modified. Such data structures are effectively immutable, as their operations do not (visibly) update the structure in-place, but instead always yield a new updated structure. (A persistent data structure is not a data structure committed to persistent storage, such as a disk; this is a different and unrelated sense of the word "persistent.")

A data structure is partially persistent if all versions can be accessed but only the newest version can be modified. The data structure is fully persistent if every version can be both accessed and modified. If there is also a meld or merge operation that can create a new version from two previous versions, the data structure is called confluently persistent. Structures that are not persistent are called ephemeral.

These types of data structures are particularly common in logical and functional programming, and in a purely functional program all data is immutable, so all data structures are automatically fully persistent. Persistent data structures can also be created using in-place updating of data and these may, in general, use less time or storage space than their purely functional counterparts.

While persistence can be achieved by simple copying, this is inefficient in CPU and RAM usage, because most operations make only small changes to a data structure. A better method is to exploit the similarity between the new and old versions to share structure between them, such as using the same subtree in a number of tree structures. However, because it rapidly becomes infeasible to determine how many previous versions share which parts of the structure, and because it is often desirable to discard old versions, this necessitates an environment with garbage collection.

\subsection{Immutable对象}
In object-oriented and functional programming, an immutable object is an object whose state cannot be modified after it is created. This is in contrast to a mutable object, which can be modified after it is created. In some cases, an object is considered immutable even if some internally used attributes change but the object's state appears to be unchanging from an external point of view. For example, an object that uses memoization to cache the results of expensive computations could still be considered an immutable object.

Immutable objects are often useful because they are inherently thread-safe. Other benefits are that they are simpler to understand and reason about and offer higher security than mutable objects.

If an object is known to be immutable, it can be copied simply by making a copy of a reference to it instead of copying the entire object. Because a reference (typically only the size of a pointer) is usually much smaller than the object itself, this results in memory savings and a potential boost in execution speed.

The reference copying technique is much more difficult to use for mutable objects, because if any user of a reference to a mutable object changes it, all other users of that reference will see the change. If this is not the intended effect, it can be difficult to notify the other users to have them respond correctly. In these situations, defensive copying of the entire object rather than the reference is usually an easy but costly solution. The observer pattern is an alternative technique for handling changes to mutable objects.

\subsection{懒拷贝}
A lazy copy is a combination of both shallow copy and deep copy. When initially copying an object, a (fast) shallow copy is used. A counter is also used to track how many objects share the data. When the program wants to modify an object, it can determine if the data is shared (by examining the counter) and can do a deep copy if necessary.

Lazy copy looks to the outside just as a deep copy but takes advantage of the speed of a shallow copy whenever possible. The downside are rather high but constant base costs because of the counter. Also, in certain situations, circular references can cause problems.

Lazy copy 与 copy-on-write 不同。



\subsection{写时拷贝}
Copy-on-write (sometimes referred to as "COW") is an optimization strategy used in computer programming. Copy-on-write stems from the understanding that when multiple separate tasks use initially identical copies of some information (i.e., data stored in computer memory or disk storage), treating it as local data that they may occasionally need to modify, then it is not necessary to immediately create separate copies of that information for each task. Instead they can all be given pointers to the same resource, with the provision that on the first occasion where they need to modify the data, they must first create a local copy on which to perform the modification (the original resource remains unchanged).

Copy-on-write finds its main use in virtual memory operating systems; when a process creates a copy of itself, the pages in memory that might be modified by either the process or its copy are marked copy-on-write. When one process modifies the memory, the operating system's kernel intercepts the operation and copies the memory thus a change in the memory of one process is not visible in another's.

Another use involves the calloc function. This can be implemented by means of having a page of physical memory filled with zeros. When the memory is allocated, all the pages returned refer to the page of zeros and are all marked copy-on-write. This way, the amount of physical memory allocated for the process does not increase until data is written. This is typically done only for larger allocations.

COW may also be used as the underlying mechanism for disk storage snapshots such as those provided by logical volume management, Microsoft Volume Shadow Copy Service or file systems such as btrfs in Linux, and ZFS on Solaris 10, Solaris 11, Illumos, OmniOS, FreeBSD and Linux.

Copy-on-write is also used in maintenance of instant snapshot on database servers like Microsoft SQL Server 2005. Instant snapshots preserve a static view of a database by storing a pre-modification copy of data when underlying data are updated. Instant snapshots are used for testing uses or moment-dependent reports and should not be used to replace backups. On the other hand, snapshots enable database back-ups in a consistent state without taking them offline.

\subsection{String interning}

In computer science, string interning is a method of storing only one copy of each distinct string value, which must be immutable. Interning strings makes some string processing tasks more time- or space-efficient at the cost of requiring more time when the string is created or interned. The distinct values are stored in a string intern pool.

The single copy of each string is called its 'intern' and is typically looked up by a method of the string class, for example String.intern() in Java. All compile-time constant strings in Java are automatically interned using this method.

Objects other than strings can be interned. For example, in Java, when primitive values are boxed into a wrapper object, certain values (any boolean, any byte, any char from 0 to 127, and any short or int between −128 and 127) are interned, and any two boxing conversions of one of these values are guaranteed to result in the same object.


String interning是flyweight设计模式的一个应用实例。






























