%!Mode:: "TeX:UTF-8"
\section{软件测试}

\subsection{白盒测试}
\textbf{白盒测试(white-box testing)}也称\textbf{结构测试、逻辑驱动测试或基于程序本身的测试}。测试应用程序的内部结构或运作,而不是测试应用程序的功能(即黑盒测试)。在白盒测试时,以编程语言的角度来设计测试案例。测试者输入数据验证数据流在程序中的流动路径,并确定适当的输出,类似测试电路中的节点。测试者了解待测试程序的内部结构、算法等信息,这是从程序设计者的角度对程序进行的测试。

缺点:过分复杂,有时候不切实际,\textbf{它可能会忽视检测规格说明中未被实现的部分}。

主要有基本路径测试(Prime path testing)和逻辑覆盖两种技术设计测试用例。

\textbf{基本路径测试法}是在程序控制流图的基础上,通过分析控制构造的环路复杂性,导出基本可执行路径集合,从而设计测试用例的方法。设计出的测试用例要保证在测试中程序的每个可执行语句至少执行一次。

逻辑覆盖是以程序内部的逻辑结构为基础的设计测试用例的技术。
根据覆盖目标的不同和覆盖源程序语句的详尽程度,逻辑覆盖又可分为:语句覆盖、判定覆盖、条件覆盖、条件/判定覆盖、条件组合覆盖、点覆盖、边覆盖、路径覆盖等。


\subsection{黑盒测试}

\textbf{黑盒测试},软件测试的主要方法之一,也可以称为\textbf{功能测试、数据驱动测试或基于规格说明的测试}。测试者不了解程序的内部情况,不需具备应用程序的代码、内部结构和编程语言的专门知识。只知道程序的输入、输出和系统的功能,这是从用户的角度针对软件界面、功能及外部结构进行测试,而不考虑程序内部逻辑结构。测试案例是依应用系统应该做的功能,照规范、规格或要求等设计。测试者选择有效输入和无效输入来验证是否正确的输出。
此测试方法可适合大部分的软件测试,例如单元测试(unit testing)、集成测试(integration testing)以及系统测试(system testing)。

可采用等价类划分、边界值分析、错误推测法等技术设计测试用例。

等价类划分法是一种典型的、重要的黑盒测试方法,它将程序所有可能的输入数据(有效的和无效的)划分成若干个等价类。然后从每个部分中选取具有代表性的数据当做测试用例进行合理的分类,测试用例由有效等价类和无效等价类的代表组成,从而保证测试用例具有完整性和代表性。

使用边界值分析方法设计测试用例时一般与等价类划分结合起来。但它不是从一个等价类中任选一个例子作为代表,而是将测试边界情况作为重点目标,选取正好等于、刚刚大于或刚刚小于边界值的测试数据。

在测试程序时,人们可能根据经验或直觉推测程序中可能存在的各种错误,从而有针对性地编写检查这些错误的测试用例,这就是错误推测法。


