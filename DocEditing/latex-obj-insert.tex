%!Mode:: "TeX:UTF-8"

\section{Object Insertion in LaTeX}

\subsection{insertion of images}
graphicx package is meant here:
\begin{lstlisting}[language={[LaTex]Tex}]
\begin{figure}[htpb]
    \begin{center}
        \includegraphics[keepaspectratio,width=0.8\paperwidth]{a.png}
    \end{center}
    \caption{SQL language elements}
    \label{fig:SQL lan elem}
\end{figure}
\end{lstlisting}

\subsection{insertion of tables}

A pitfall is that the label must appear after the caption. 
I discovered this while on PHD dissertation.
\begin{lstlisting}[language={[LaTex]Tex}]
\begin{table}[tbp]
 \centering 
 \begin{tabular}{lcc}  
 \hline
A & B & HiliMG \\ 
 \hline  
C  & 32 & 0 \\         
D & 3.61Mbps & 3.75Mbps \\        
 \hline
 \end{tabular}
 \caption{HiliMG
 \label{tab:hilimgAndOpenloop}
 \end{table}
\end{lstlisting}


\subsection{insertion of special characters}
\verb+\textbackslash+


\subsection{insertion of code}
\begin{verbatim}
\usepackage{listings} 

\begin{lstlisting}[language=C]
int rank()
{ 
    return 0;
}
\end{lstlisting}
\end{verbatim}


\subsection{insertion of pesudo-code}

We can combinely use 
algorithm and algorithmic environments.
The required packages are:

\begin{verbatim}
\usepackage{algorithm}
\usepackage{algpseudocode}
\end{verbatim}

For Debian's texlive system, texlive-science package is needed here.
\begin{verbatim}
\begin{algorithm}
\caption{xxxxxxxx}\label{euclid}
\begin{algorithmic}[1]
\Procedure{StaticChunkedParse}{$msg$}\Comment{msg is the chunked message}
\State $ep\gets 0$
\While{\texttt{True}}
\State $chunkSize \gets msg[ep:ep+64]$
\State $chunkLen \gets hex2int(chunkSize)$
\If{$chunkLen = 0$}
\State \textbf{break}
\EndIf
\State $ep \gets ep + len(chunkSize)$
\State $output \gets output + msg[ep:ep+chunkLen]$
\State $ep \gets ep + chunkLen$
\EndWhile\label{euclidendwhile}
\State \textbf{return} $output$\Comment{output is the decoded content}
\EndProcedure
\end{algorithmic}
\label{alg:static}
\end{algorithm}
\end{verbatim}



















