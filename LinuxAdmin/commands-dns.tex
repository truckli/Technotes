%!Mode:: "TeX:UTF-8"
\section{DNS查询}
\subsection{域名解析工具}
dig取代了nslookup和host命令。

dig可指定查询类型:
\begin{verbatim}
dig ustc.edu.cn +short
dig ustc.edu.cn MX
dig ustc.edu.cn MX
dig bbs.veno2.com AAAA
dig bt.neu6.edu.cn AAAA
\end{verbatim}

反向查询:
\begin{verbatim}
dig -x 202.108.22.220
\end{verbatim}

dig可以指定DNS服务器:
\begin{verbatim}
dig www.hrbnu.edu.cn @159.226.59.158
dig www.hrbnu.edu.cn @202.96.199.133 
\end{verbatim}

在Python中,socket模块即可提供域名解析功能:
\begin{verbatim}
import socket; socket.gethostbyname('www.baidu.com')
\end{verbatim}


BIND(Berkeley Internet Name Daemon)是现今互联网上最常使用的DNS服务器软件,使用BIND作为服务器软件的DNS服务器约占所有DNS服务器的九成。BIND现在由互联网系统协会(Internet Systems Consortium)负责开发与维护。BIND又称\textbf{named}。

\subsection{查询本机公网IP}

Linux下查询本机的局域网IP是容易的,只需执行ifconfig命令查看即可。
如果本地有多个IP地址,为了查找某个连接所对应的本机IP地址,可利用socket的getsockname接口。
例如在Python下执行如下脚本可显示用于连接公网的本地IP地址:
\begin{verbatim}
import socket
s = socket.socket(socket.AF_INET, socket.SOCK_STREAM)
s.connect(('www.baidu.com', 80))
print s.getsockname()
s.close()
\end{verbatim}

而查询本机的公网IP,必须借助于外部的服务器。


某些Web服务提供了返回客户端IP地址的功能。其中一些页面为纯文本,适合用文本工具提取,如:
\begin{verbatim}
curl -s checkip.dyndns.org
wget http://ipinfo.io/ip -qO -
curl -s http://ipecho.net/plain
python -c 'from urllib2 import urlopen; print urlopen('http://ip.42.pl/raw').read()'
\end{verbatim}

其他一些服务提供的信息用Json等格式返回,可以用Python等脚本语言提取:
\begin{verbatim}
python -c "from json import load; from urllib2 import urlopen; 
print load(urlopen('http://jsonip.com'))['ip']"

python -c "from json import load; from urllib2 import urlopen; 
print load(urlopen('http://httpbin.org/ip'))['origin']" 

python -c "from json import load; from urllib2 import urlopen; 
print load(urlopen('https://api.ipify.org/?format=json'))['ip']"
\end{verbatim}


而大多数这类服务器用HTML来返回结果,适合用浏览器查看:
\begin{verbatim}
http://whatismyipaddress.com/
https://www.whatismyip.com/
https://www.iplocation.net/
http://www.whatsmyip.org/
http://www.myipaddress.com/what-is-my-ip-address/
\end{verbatim}

类似的国内网站包括:
\begin{verbatim}
http://ip.chinaz.com/
http://www.ip138.com/
http://www.whatismyip.com.tw/
\end{verbatim}


一些机构提供了基于DNS端口的客户端IP查询功能,如opendns.com和谷歌:
\begin{verbatim}
dig +short myip.opendns.com @resolver1.opendns.com
host myip.opendns.com resolver1.opendns.com
dig TXT +short o-o.myaddr.l.google.com @ns1.google.com
\end{verbatim}



如果设置了正向代理,这些服务返回的可能是代理服务器的地址。


