\section{virtualenv}

\subsection{virtualenv}
Just install the package with pip, create a virtual environment and activite it.
\begin{verbatim}
pip install virtualenv
virtualenv --python=/path/to/python/interpreter my_venv
source my_venv/bin/activate
deactivate
\end{verbatim}

We can copy the whole environment folder to another machine which is disconnected to the Internet. 
Dynamic libraries such as /usr/lib/libpython2.7.so.1.0  should be transfered together.


\subsection{virtualenvwrapper}

Installation:
\begin{verbatim}
    pip install virtualenvwrapper
\end{verbatim}

To start using virtualenvwrapper:
\begin{verbatim}
    export WORKON_HOME=~/PyEnv
    export PROJECT_HOME=~/PyPro
    source /usr/local/bin/virtualenvwrapper.sh
    lsvirtualenv
\end{verbatim}



Common commands:
\begin{verbatim}
    
mkvirtualenv my_env
workon my_env
deactivate

mkproject my_project
workon my_project
deactivate

rmvirtualenv my_project
cdvirtualenv

\end{verbatim}


\subsection{Anaconda as virtualenv}

Anaconda can act as virtualenv, even in offline environment:
\begin{verbatim}
    conda create --name test_env_name --offline python=2.7
\end{verbatim}