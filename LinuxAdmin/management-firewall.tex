%!Mode:: "TeX:UTF-8"
\section{iptables操作}
linux的包过滤功能,即linux防火墙,它由netfilter 和 iptables 两个组件组成。

netfilter 组件也称为内核空间,是内核的一部分,由一些信息包过滤表组成,这些表包含内核用来控制信息包过滤处理的规则集。

iptables 组件是一种工具,也称为用户空间,它使插入、修改和除去信息包过滤表中的规则变得容易。

iptables 默认有四个表Filter, NAT, Mangle, Raw。表包含若干个链(chain),链包含若干规则。

\begin{verbatim}
重启:service iptables restart

查看iptables状态:service iptables status

保存iptables配置:service iptables save

Iptables服务配置文件:/etc/sysconfig/iptables-config

Iptables规则保存文件:/etc/sysconfig/iptables

打开iptables转发:echo "1"> /proc/sys/net/ipv4/ip_forward

\end{verbatim}

iptables命令:
\begin{verbatim}
iptables [ -t 表名] 命令选项 [链名] [条件匹配] [-j 目标动作或跳转] 
\end{verbatim}
表名默认为Filter。

命令选项:
\begin{verbatim}
-A : 追加
-I pos: 在pos处插入。pos默认为1。
-D : 删除规则。后面要指明规则号或规则内容。
-R: 替换规则。
-L: 列举指定chain(默认为所有chain)的规则
-F: 清空指定chain(默认为所有chain)的规则
-N: 新建一条链
-X: 删除一条链
-P: 设置一条链的默认策略
-V: 显示版本号
-v: 显示详细信息
-h: 显示帮助信息
-n: 以数字形式显示结果
\end{verbatim}

条件匹配:
\begin{verbatim}
-p: 指定协议,如tcp, udp, icmp
-s: 源地址
-d: 目的地址
-i:输入端口
-o: 输出端口
-m: 扩展
\end{verbatim}

目标值:
\begin{verbatim}
ACCEPT:允许数据包通过。 
DROP:直接丢弃数据包,不给出任何回应信息。 
REJECT:拒绝数据包通过,必须时会给数据发送端一个响应信息。 
LOG:在/var/log/messages 文件中记录日志信息,然后将数据包传递给下一条规则。 
QUEUE:防火墙将数据包移交到用户空间 
RETURN:防火墙停止执行当前链中的后续Rules,并返回到调用链(the calling chain) 
\end{verbatim}



举例
\begin{verbatim}
查看iptables规则
iptables –L(iptables –L –v -n)

允许内网访问本机的某些端口
iptables -I INPUT 2 -i eth0 -p tcp -m multiport --dport 21,25,80,110,8082 -s 172.0.0.0/8 -j ACCEPT

开放本机的8082端口
iptables -I INPUT 2 -i eth0 -p tcp --dport 8082 -j ACCEPT

iptables -I INPUT 2 -i eth0 -p tcp --dport 80 -m state --state NEW,ESTABLISHED -j ACCEPT 

iptables -A INPUT -i eth0 -p tcp --dport 80 -m state --state NEW,ESTABLISHED -j ACCEPT 

iptabels -D INPUT 2 

iptables -R INPUT 3 -i eth0 -p tcp --dport 80 -m state --state NEW,ESTABLISHED -j ACCEPT 

设置默认策略
iptables -P INPUT DROP 

允许远程主机进行SSH连接
iptables -A INPUT -i eth0 -p tcp --dport 22 -m state --state NEW,ESTABLISHED -j ACCEPT 
iptables -A OUTPUT -o eth0 -p tcp --sport 22 -m state --state ESTABLISHED -j ACCEPT 

允许HTTP请求
iptables -A INPUT -i eth0 -p tcp --dport 80 -m state --state NEW,ESTABLISHED -j ACCEPT 
iptables -A OUTPUT -o eth0 -p tcp --sport 80 -m state --state ESTABLISHED -j ACCEPT 

限制ping 192.168.146.3主机的数据包数,平均2/s个,最多不能超过3个
iptables -A INPUT -i eth0 -d 192.168.146.3 -p icmp --icmp-type 8 -m limit --limit 2/second --limit-burst 3 -j ACCEPT 

限制SSH连接速率(默认策略是DROP)
iptables -I INPUT 1 -p tcp --dport 22 -d 192.168.146.3 -m state --state ESTABLISHED -j ACCEPT  
iptables -I INPUT 2 -p tcp --dport 22 -d 192.168.146.3 -m limit --limit 2/minute --limit-burst 2 -m state --state NEW -j ACCEPT 

\end{verbatim}



详见\url{http://drops.wooyun.org/tips/1424}

\url{https://lesca.me/archives/iptables-tutorial-structures-configuratios-examples.html}

\url{https://www.centos.bz/2011/06/iptables-basic-guide/}

\url{http://www.vpser.net/security/linux-iptables.html}

\url{https://www.frozentux.net/iptables-tutorial/cn/iptables-tutorial-cn-1.1.19.html}














