%!Mode:: "TeX:UTF-8"
\section{邮件配置:MTA,MRA,MUA}

邮件的账户,实质上就是邮件服务器主机上的用户账户。

\subsection{Postfix}
默认安装后,可能需修改/etc/postfix目录下的配置文件。
另外,要开放防火墙。


\begin{verbatim}
inet_interfaces = all
mynetworks_style = subnet
mynetworks = 168.100.189.0/28, 127.0.0.0/8, 172.0.0.0/8
\end{verbatim}

邮箱格式有mbox(默认)和Maildir两种风格。采用Maildir方式,设置用户的邮箱路径设置为:
\begin{verbatim}
home_mailbox = Maildir/
\end{verbatim}

修改配置文件后需要重新加载:
\begin{verbatim}
postfix reload
\end{verbatim}

但如果修改了inet\_interfaces参数则需要重启服务器:
\begin{verbatim}
service postfix restart
\end{verbatim}

也可如此重启postfix:
\begin{verbatim}
sudo postfix stop
sudo postfix start
\end{verbatim}

查看启动状态
\begin{verbatim}
postfix status
\end{verbatim}


查看参数配置
\begin{verbatim}
postconf 
\end{verbatim}


Postfix默认接受相同网络(通过mynetworks参数设置)的邮件客户IP,其他网络的客户端需要经过
\href{http://www.postfix.org/SASL_README.html}{SASL认证}。
Postfix支持Cyrus SASL以及Dovecot提供的SASL。
Cyrus SASL本身又包含多种认证方式,学习起来较为复杂。

如果遭遇到权限错误,需检查selinux设置。 



Postfix最好能够开启\href{http://www.postfix.org/TLS_README.html}{TLS支持}。



参考:

\href{https://help.ubuntu.com/community/PostfixBasicSetupHowto}{Ubuntu官方文档:PostfixBasicSetupHowto}

\href{Postfix Howtos and FAQs}{Postfix Howtos and FAQs}

\href{http://www.postfix.org/documentation.html}{Postfix Documentation}



\subsection{Dovecot}

Dovecot的配置文件位于/etc/dovecot。

在dovecot.conf中设置MRA所支持的协议(pop3, imap等):

\begin{verbatim}
# Protocols we want to be serving.
#protocols = imap pop3 lmtp
protocols = pop3 imap
\end{verbatim}

邮箱路径要同MTA的设置一致。可采用mbox风格:
\begin{verbatim}
mail_location = mbox:~/mail:INBOX=/var/mail/%u
mail_access_groups = mail
\end{verbatim}
或采用Maildir风格:
\begin{verbatim}
mail_location = maildir:~/Maildir
\end{verbatim}

如果采用mbox风格,且初次运行时并不存在~/.imap/INBOX路径,dovecot将创建它,并将其group设置为mail。
然后dovecot本身是以具体邮箱用户的身份运行,并无权限将文件的group设置为mail,因为该用户不属于mail组。
因此需要设置mail\_access\_groups。但如此会导致一些安全隐患。

常用的设置项还有
\begin{verbatim}
# Disable LOGIN command and all other plaintext authentications unless
# SSL/TLS is used (LOGINDISABLED capability). Note that if the remote IP
# matches the local IP (ie. you're connecting from the same computer), the
# connection is considered secure and plaintext authentication is allowed.
disable_plaintext_auth = no
# SSL/TLS support: yes, no, required. <doc/wiki/SSL.txt>
ssl = no
\end{verbatim}


查看Dovecot的日志文件存储路径:
\begin{verbatim}
% doveadm log find
\end{verbatim}
通常其日志文件为/var/log/maillog。

参考
\href{http://wiki2.dovecot.org/}{Dovecot官方Wiki}。




\subsection{Cyrus SASL认证配置}


Cyrus SASL的saslauthd守护进程支持多种认证渠道,包括/etc/shadow, PAM,
借助IMAP服务器等。除saslauthd外,Cyrus SASL也支持通过sasldb, sql, ldapdb三种插件来进行认证。

Cyrus SASL提供的认证机制包括:plain,login,DIGEST-MD5,CRAM-MD5等。

如果要采用plain和login机制,需要安装相应的包:
\begin{verbatim}
yum install cyrus-sasl-plain
\end{verbatim}



\begin{verbatim}
/etc/init.d/saslauthd start
\end{verbatim}

需要合理设置Maildir的权限,否则SELINUX可能会导致认证失败
\begin{verbatim}
setenforce 0
sestatus
\end{verbatim}



testsaslauthd可用于测试账号认证功能:
\begin{verbatim}
testsaslauthd -u username -p password
\end{verbatim}

配置Postfix的认证:
\begin{verbatim}

/etc/postfix/main.cf:
	smtpd_sasl_type = cyrus 
	smtpd_sasl_path = smtpd
    smtpd_sasl_auth_enable = yes
    broken_sasl_auth_clients = yes

\end{verbatim}


详情可参考:

\href{http://www.postfix.org/SASL_README.html}{Postfix官方文档: SASL Howto}。

\href{http://postfix.state-of-mind.de/patrick.koetter/smtpauth/index.html}{Patrick Ben Koetter: Postfix SMTP AUTH (and TLS) HOWTO}。



\subsection{Dovecot SASL认证配置}
前提条件是配置好Dovecot本身的认证机制,然后设置Postfix令其借助于该机制进行认证。


\begin{verbatim}

/etc/dotecov/conf.d/10-master.conf:
    service auth {
       ...
       unix_listener /var/spool/postfix/private/auth {
         mode = 0660
         # Assuming the default Postfix user and group
         user = postfix
         group = postfix        
       }
       ...
     }
 
/etc/dotecov/conf.d/10-auth.conf:
    auth_mechanisms = plain login

/etc/postfix/main.cf:
    smtpd_sasl_type = dovecot
    smtpd_sasl_path = private/auth
    smtpd_sasl_auth_enable = yes
    broken_sasl_auth_clients = yes
    
\end{verbatim}



\subsection{mailx发送邮件}

mail命令由软件包mailx(Debian系称作mailutils)提供。mail可连接本地的MTA发送邮件。

以下命令可用于发送邮件,命令行中给出收件人和主题,以shell为编辑器编写邮件正文,以EOT结束编辑。 
-s参数可给出,也可不给,留在编辑时写出主题。
\begin{verbatim}
mail -s MySubject root@zld_02.chanct.com
\end{verbatim}

也可以在命令行中直接给出邮件的正文内容:
\begin{verbatim}
echo “mail content”|mail -s test admin@aispider.com 
\end{verbatim}

还可以用文件作为正文内容:
\begin{verbatim}
mail -s test  admin@aispider.com < file
\end{verbatim}


\subsection{mailx查看邮件}
不带任何参数执行mail命令,将查看当前用户的邮件。
用户的邮件位于\verb|/var/spool/mail/用户名|目录下,这一路径由环境变量MAIL设置。
\begin{verbatim}
export MAIL=
\end{verbatim}

在邮件查看界面下可使用一些子命令。

\begin{description}
  \item [?] 给出命令提示 
  \item [file|folder] 查看邮箱概要
  \item [x] 退出mail界面,不执行变更 
  \item [q] 退出mail界面,执行变更 
  \item [t|type|p|print msglist] 显示邮件正文 
  \item [f|from msglist] 显示邮件概要
  \item [s|save msglist fname] 追加邮件至fname文件, 标记为已保存
  \item [copy msglist fname] 追加邮件至fname文件, 不标记为已保存
  \item [n|回车] 迭代至下一封邮件并查看正文 
  \item [v|visual] 用编辑器打开当前邮件 
  \item [d msglist] 删除邮件 

\end{description}

子命令中的msglist为邮件的id列表,如“1 3 7”表示第一、第三、第七封邮件,“2-4”表示第二、第三、第四封邮件。
如果msglist不给出,则认为是上一次被显示的邮件。



























