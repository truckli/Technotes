%!Mode:: "TeX:UTF-8"
\section{Ftp配置}


\subsection{Ftp命令}

登录某ftp服务器,ftp命令的参数可以是主机,也可不带参数,而在ftp命令界面下采用open命令:
\begin{lstlisting}
	ftp 1.1.1.1
\end{lstlisting}

close和disronnect命令关闭与远程机的连接,但是使用户留在本地计算机的ftp程序中。

bye命令都关闭用户与远程机的连接,然后退出用户机上的ftp程序。

cd, ls, pwd等命令是针对远程主机的。

'!ls'或'!dir'命令相当于本机下的ls命令。lcd则在本机切换目录。'!pwd'是本地端的pwd。

other useful commands:
\begin{lstlisting}
	open 1.1.1.1
	send FILENAME
	get FILENAME
	dir # list remote folder
\end{lstlisting}

The \verb|passive| command can turn on/off passive mode.



\subsection{vsftpd配置}
配置文件位于/etc/vsftpd目录下,ftpusers文件存放的是黑名单,user\_list可以是白名单或黑名单。
可在vsftpd.conf中设置userlist\_deny=NO,表示此为白名单,应将有登录权限的用户名追加于此文件。

通过设置SELinux,运行ftp访问用户目录:
\begin{lstlisting}
setsebool -P ftp_home_dir=1
\end{lstlisting}

Allow local login for test purposes:
\begin{lstlisting}
LOCAL_ENABLE = YES
\end{lstlisting}

Restrict PASV port to a small range:
\begin{lstlisting}
pasv_enable=YES
pasv_max_port=11021
pasv_min_port=11020
\end{lstlisting}



\subsection{tFtp配置}
Ubuntu下,安装tftp-hpa。在/etc/default/tftp-hpa目录下配置。
RHEL/CentOS环境下,安装tftp-server。守护进程名称为in.tftpd。

tftp的配置文件在/etc/xinetd.d/tftp,可修改disable=no,默认路径等参数。然后执行:
\begin{lstlisting}
    service xinetd restart
\end{lstlisting}

可用netstat命令查看69号端口是否开启。测试发现xinetd开启若干分钟后,in.tftpd守护进程才启动。此时可在另一台主机上测试:

\begin{lstlisting}
    tftp (serverip)
    tftp>get (some_file_name)
\end{lstlisting}

如果结果是
\begin{lstlisting}
Error code 0: Permission denied
\end{lstlisting}
可能是selinux问题。运行
\begin{lstlisting}
ls -alZ
\end{lstlisting}
结果应包含:
\begin{lstlisting}
 user_u:object_r:tftpdir_t
\end{lstlisting}

解决办法是:
\begin{lstlisting}
restorecon -Rv /tftpboot 
\end{lstlisting}




