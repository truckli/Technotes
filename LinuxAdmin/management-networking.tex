%!Mode:: "TeX:UTF-8"



\section{Network Traffic Measurement} \label{sec:netmeasure} 
vnstat, sar, slurm, ifstat, system-monitor等工具可查看网卡总流量。iptraf,iftop可查看连接的流量。

\subsection{简单测量}
最原始的办法,是连续两次使用date;ifconfig命令,计算一定时间间隔内的数据量。
也可以通过查看/proc/net/dev获取数据量。
在Gnome3下,可以使用一个叫做netspeed的gnome shell插件。

\subsection{vnstat工具}
\begin{shellcmd}
#-ru 0 使其以byte为单位,1使其以bit为单位.
vnstat -l -ru 0 #持续采样 
vnstat -tr #统计网速,5秒内的采样平均计算所得。
\end{shellcmd}

\subsection{iftop工具}
显示带宽使用情况。3列显示,分别表示过去2s,10s,40s内的统计带宽。
\begin{verbatim}
iftop -h | [-nNpbBP] [-i interface] [-f filter code] [-F net/mask]
\end{verbatim}
例如:
\begin{shellcmd}
#-B表示以byte而非bit为单位,-P显示端口号
sudo iftop -B -P 
\end{shellcmd}
工具默认自动将IP地址转换为主机名,会产生一定的DNS流量,扰乱测试。为讲其关闭,可使用-n命令。

\subsection{sar工具}
也可以使用sar工具.在Redhat/Fedora下,sar工具位于sysstat软件包中.
\begin{shellcmd}
#最后的数字表示刷新时间间隔,单位为秒
sar -n DEV 3 
\end{shellcmd}

经我验证,sar统计的字节数为以太网层,包括其头部和尾部(虽然Linux抓不到帧尾的FCS),不包括前导码和帧间隔。

\subsection{ifstat工具}
\begin{shellcmd}
ifstat -a
\end{shellcmd}

\subsection{ntop工具}
Ntop是一种监控网络流量工具,用ntop显示网络的使用情况比其他一些网络管理软件更加直观、详细。Ntop甚至可以列出每个节点计算机的网络带宽利用率。它是一个灵活的、功能齐全的,用来监控和解决局域网问题的工具;尤其当ntop与nprobe配合使用,其功能更加显著。它同时提供命令行输入和web页面,可应用于嵌入式web服务。跟 top 监视系统活动状况相似,ntop 是一个用来实时监视网络使用情况的工具。由于 ntop 具有 Web 界面模式,因此无论是配置还是使用都很容易在短时间之内快速上手。

\subsection{iptraf工具}
Interactive Colorful IP LAN Monitor。可查看每条连接的信息。
\begin{verbatim}
iptraf -i eth0
\end{verbatim}


\subsection{slurm工具}
 Simple Linux Utility for Resource Management,查看网络流量的一个工具。
 \begin{verbatim}
 slurm -i eth0
 \end{verbatim}

彩色curse节目,有部分文字是白色,在浅色背景下看不清楚。









