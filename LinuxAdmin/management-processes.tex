%!Mode:: "TeX:UTF-8"
\section{进程查看}
显示所有进程:
\begin{verbatim}
pstree -p #-p选项会打印出PID的值,否则只打印名字
ps -ef
\end{verbatim}


\section{进程控制}

按照NOHUP方式执行进程:
\begin{verbatim}
nohup ./cdn > /dev/null 2>&1 &
\end{verbatim}
noup:run a command immune to hangups, with output to a non-tty


杀进程:
\begin{verbatim}
kill [信号] pid
pkill [信号] pattern
fuser -km 文件名
\end{verbatim}

xkill杀死单一窗口
如需重启,可按住Alt SysRq,再依次按REISUB


\subsection{进程查找}

返回程序program的pid:
\begin{verbatim}
pidof program 
pgrep program 
ps -ef|grep program|awk '{print $2}'
\end{verbatim}
pgrep的-f参数使得查找的pattern不仅局限于程序名,而是整个命令行。 -u参数可以限制所属的user。

通过pid查看进程的命令行参数:
\begin{verbatim}
cat /proc/2884/cmdline
ps 3236|grep -v PID|awk '{print $5}'
\end{verbatim}

通过pid查看进程的打开的fd:
\begin{verbatim}
ls /proc/2884/fd
\end{verbatim}

通过pid查看进程的打开的文件:
\begin{verbatim}
lsof -p 2884
\end{verbatim}
通过程序名查看进程program的打开的文件:
\begin{verbatim}
lsof -c program 
\end{verbatim}

通过文件名查看打开该文件的进程
\begin{verbatim}
lsof ProberRun.log
fuser -m ProberRun.log
\end{verbatim}

显示打开了某个端口的进程
\begin{verbatim}
lsof -i:8082 #只包括本用户的进程, root用户除外
lsof -i |grep 8082
netstat -nlpt |grep 8082
\end{verbatim}




