%!Mode:: "TeX:UTF-8"
\section{用户与组}

\subsection{用户增加与删除}
useradd, adduser, userdel, deluser, usermod, delgroup

使用ueradd时,如果后面不添加任何参数选项,例如:\verb+#sudo useradd test+创建出来的用户将是默认“三无”用户:一无Home Directory,二无密码,三无系统Shell。
使用adduser时,创建用户的过程更像是一种人机对话,系统会提示你输入各种信息,然后会根据这些信息帮你创建新用户。
\begin{verbatim}
useradd -m -s /bin/bash user 
-m表示创建家目录,默认不创建。
注意-p选项以密文方式设置密码,似乎没什么用。
userdel -r user
删除用户及其家目录
\end{verbatim}


\subsection{设置用户的Bash工作环境}

通常在\verb|.bash_profile|中调用\verb|.bashrc|:
\begin{verbatim}
if [ -f ~/.bashrc ]; then . ~/.bashrc; fi
\end{verbatim}


\subsection{设置用户为sudoer}
执行visudo命令,在打开的文件中添加一行
\begin{verbatim}
userA ALL=(ALL)  ALL
userB ALL=(ALL)  NOPASSWD: ALL
\end{verbatim}
visudo相当于修改/etc/sudoers文件。userB执行sudo的时候不需要输入密码。

可以在bashrc或profile中添加如下内容,修改PATH
\begin{verbatim}
PATH=/usr/sbin:/usr/local/sbin:/sbin:$PATH
export PATH
\end{verbatim}



\subsection{用户所在组}

\subsubsection{查看用户所在组}
方法一:id命令和groups命令
方法二:/etc/group,格式:
\begin{verbatim}
group_name:password('x'):GID:user_list(separated by commas)
\end{verbatim}

\subsubsection{将用户加入组}
在Debian系上,采用命令
\begin{verbatim}
adduser username groupname 
\end{verbatim}

一般地,有
\begin{verbatim}
usermod -a -G <groupname> username
\end{verbatim}



\subsection{用户的shell}
/etc/passwd包含了家目录,shell等用户信息
查看系统安装的shell
\begin{verbatim}
cat /etc/shells
\end{verbatim}

查看当前shell
\begin{verbatim}
echo $SHELL
\end{verbatim}

更改某用户的shell
\begin{verbatim}
usermod -s /bin/zsh someuser
\end{verbatim}


\subsection{修改用户名和主机名}
用户名在/etc/shadow中修改,Ubuntu主机名需要在/etc目录下hostname和hosts两个文件下修改,Centos6.4主机名通过/etc/sysconfig/network文件修改。



\subsection{特殊文件权限}
setuid,setgid和粘贴位为特殊文件权限。

当一个可执行文件具有setuid这个权限时,文件的运行者将具有文件的所有者所拥有的权限。
通常passwd程序具有setuid权限。
\begin{verbatim}
-r-sr-sr-x   3 root     sys       104580 Sep 16 12:02 /usr/bin/passwd
\end{verbatim}

如果root用户所拥有的程序具有特殊文件权限:w
,则能够运行这个程序的用户将拥有root用户的uid,这将构成安全隐患。
另外,普通用户可以让自己拥有的程序具有特殊文件权限,其他用户运行了这个程序,将拥有该用户的权限。

若一个目录被设置setgid权限,在该目录下创建的文件以该目录的所有者为所有者,而非以创建者为所有者。
\begin{verbatim}  
-r-x--s--x   1 root     mail       63628 Sep 16 12:01 /usr/bin/mail
\end{verbatim}

  当粘贴位被设置于一个目录时,目录下的文件将被作防删除包含:只有文件的owner,目录的owner或root才可删除文件。
  /tmp目录常被设置粘贴位。
  
\begin{verbatim}  
drwxrwxrwt 7  root  sys   400 Sep  3 13:37 tmp
\end{verbatim}
  
  
 \subsection{ACL} 
 \begin{verbatim}
  setfacl -m d:u:myuser1:rx /srv/projecta
 \end{verbatim}
  
  
 \subsection{executing command as specified user} 

 \begin{verbatim}
 su USER  -c "CMD"
 \end{verbatim}
  
  
  
