%!Mode:: "TeX:UTF-8"
\section{日志信息}


\subsection{查看用户连接信息}
连接时间日志一般由/var/log/wtmp和/var/run/utmp这两个文件记录,不过这两个文件都无法使用tail或cat命令直接查看。该文件由系统自动更新。Linux提供了如 w, who, finger, id, last, lastlog,ac等命令读取这部分的信息。
\begin{description}
    \item[finger]user information lookup program.用户详细信息,甚至包括电话号码。 
    \item[w]Show who is logged on and what they are doing.
    \item[last]show listing of last logged in users.\verb+last|grep reboot+记录了开机时间。 
    \item[lastlog] reports the most recent login of all users or of a given user
    \item[users]print the user names of users currently logged in to the current host. 其实就是第一列。
    \item[ac]ac应是accounting的缩写。打印用户连接时间的总和,指多个pty
    \item[id]当前用户的id和组id
    \item[uptime]查看开机时间,其实就是w输出的第一行
\end{description}

The utmp file allows one to discover information about who is currently using the system. 

The  wtmp  file records all logins and logouts.

\subsection{命令历史记录}
.bash\_history 记录了Bash命令历史,但不包含最近执行的命令。

/var/log/apt/目录下的文件记录了apt相关的命令执行历史。

\subsection{进程清算(accounting)}

acct是一个系统调用:

\begin{verbatim}
 #include <unistd.h>
 int acct(const char *filename);
\end{verbatim}

The  acct()  system  call  enables  or  disables process accounting.  If called with the name of an existing file as its argument,
accounting is turned on, and records for each terminating process are appended to filename as it terminates.  An argument of  NULL
causes accounting to be turned off.

例如:
\verb+ acct("/var/log/pacct");+

accton为系统命令,清算一段时间内终结的进程, 记录于清算文件中, 默认为/var/log/account/pacct。lastcomm用来读取清算文件。
\verb+ accton [OPTION] on|off|filename+

例如:
\begin{verbatim}
sudo accton somefile
sudo accton off
lastcomm -f somefile
\end{verbatim}

默认的pacct清算似乎早已默认开启,使得\verb+sudo accton on+命令显得多余。


\subsection{内核日志}
dmesg: 显示内核信息

\subsection{系统与服务日志}
由syslog的服务管理,比如下面的日志文件都是由syslog日志服务驱动的:      
/var/log/lastlog :记录最后一次用户成功登陆的时间、登陆IP等信息 

/var/log/messages 包括整体系统信息,其中也包含系统启动期间的日志。此外,mail,cron,daemon,kern和auth等内容也记录在var/log/messages日志中。

/var/log/secure :Linux系统安全日志,记录用户和工作组变坏情况、用户登陆认证情况 

/var/log/btmp :记录Linux登陆失败的用户、时间以及远程IP地址 

/var/log/cron :记录crond计划任务服务执行情况

/var/log/dmesg — 包含内核缓冲信息(kernel ring buffer)。在系统启动时,会在屏幕上显示许多与硬件有关的信息。可以用dmesg查看它们。

/var/log/boot.log — 包含系统启动时的日志

syslog服务由配置文件/etc/syslog.conf或/etc/rsyslog.conf.

/etc/syslog.conf的内容格式为:
\begin{verbatim}
消息类型.错误级别    动作域
\end{verbatim}

消息类型(facility)包括:
auth, authpriv (for security information of a sensitive nature), cron, daemon, ftp, kern , 
lpr, mail, news, security (deprecated synonym for auth), syslog, user, uucp, 
local0, local1, \ldots,  local7。


错误级别包括: emerg, alert, crit, err, warning, notice, info, debug。优先级依次降低。

动作域即日志文件路径,如为星号则表示在各个日志文件、用户终端都输出。



\subsection{logger}
logger程序为syslog提供了shell接口。
-p选项设置了消息的facility和level,如 -p local3.info。

\subsection{日志转储}
Linux中使用logrotate命令进行日志转储。
可以配合cron计划任务轻松实现日志文件的定时转储, /etc/logrotate.conf提供了日志转储的相关配置。
各个应用自身的转储配置存放于/etc/logrotate.d中,被/etc/logrotate.conf所读取。
包括发送电子邮件。

\begin{verbatim}
/var/log/active-prober.log {
        su root root
        weekly
        missingok
        rotate 5
        size=500
        sharedscripts
        postrotate
                echo "Trimming active-prober.conf", $(date) >> /root/datefile
        endscript
}
\end{verbatim}











