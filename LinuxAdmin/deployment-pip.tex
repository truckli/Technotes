\section{Pip tricks}


\subsection{Selecting a fast source mirror}

\begin{verbatim}
pip install dnspython -i http://pypi.douban.com/simple --trusted-host pypi.douban.com
\end{verbatim}

Here is a list of mirrors in mainland China:
\begin{verbatim}
http://pypi.douban.com/  
http://pypi.hustunique.com/  
http://pypi.sdutlinux.org/  
http://pypi.mirrors.ustc.edu.cn/  
http://mirrors.aliyun.com/
\end{verbatim}


In file  ~/.pip/pip.conf:

\begin{verbatim}
[global]
trusted-host=mirrors.aliyun.com
index-url=http://mirrors.aliyun.com/pypi/simple
\end{verbatim}

\subsection{requirements freezing}

\begin{verbatim}
    pip freeze > requirements.txt
    pip install -r requirements.txt
\end{verbatim}

\subsection{create a local mirror}

\begin{verbatim}
    https://aboutsimon.com/blog/2012/02/24/Create-a-local-PyPI-mirror.html
\end{verbatim}

\subsection{package installation}

To download package wheels without installing it:
\begin{verbatim}
pip download --platform=manylinux1_x86_64 --python-version 27 
    --only-binary=:all: -r requirement.txt
\end{verbatim}
To download package source tars without installing it:
\begin{verbatim}
pip download --no-binary=:all: virtualenv
\end{verbatim}

In an offline environment, package installation with pre-downloaded installers is like:

\begin{verbatim}
pip install --no-index --find-links="~/download" fabric 
\end{verbatim}


\subsection{show package information}


\begin{verbatim}
[user@host]$ pip show fabric
Name: fabric
Version: 2.1.3
Summary: High level SSH command execution
Home-page: http://fabfile.org
Author: Jeff Forcier
Author-email: jeff@bitprophet.org
License: BSD
Location: /Users/mingzhe/anaconda/lib/python3.6/site-packages
Requires: cryptography, paramiko, invoke
\end{verbatim}

Direct dependancies of a package is listed. For a more detailed dependency graph on installed packages, pipdeptree tool can be used.

\begin{verbatim}
   pip install pipdeptree
   pipdeptree -p fabric 
\end{verbatim}}


\subsection{Anaconda as pip}

Anaconda can act as pip, even in offline environment:
\begin{verbatim}
    conda install fabric --offline python=2.7
    conda list
\end{verbatim}
By specifying Python version when creating a virtual environment we effectively have a way for Python2/Python3 coexistance.

Anaconda can be configured to used a customized download site in order to accelerate package downloading. An example of ~/.condarc is

\begin{verbatim}
channels:
 - https://mirrors.tuna.tsinghua.edu.cn/anaconda/pkgs/free/
 - defaults
show_channel_urls: true

\end{verbatim}


