%!Mode:: "TeX:UTF-8"
\section{常用操作命令}

\subsection{ls}
解释几个比较实用的选项。当不使用任何选项时,输出按照名字排序。S,t选项能更改排序的依据。r指示反序。

\textbf{l选项}产生类似如下输出:
\begin{verbatim}
srw-rw-rw-   1 root root             0 11月  3 15:45 log=
crw-------   1 root root       10, 237 11月  3 15:45 loop-control
-rwsr-xr-x   1 root root         45420  7月 27 01:07 /usr/bin/passwd*
\end{verbatim}
其中第1个字符表示文件类型。s表示socket,p表示FIFO(命名管道),b表示块设备,c表示字符设备,l表示符号链接,-表示普通文件,?表示未知类型。

在文件权限中,s表示设置用户(组)id位为1,且文件可执行。t表示粘住位为1,且文件可执行。相应的S和T表示文件不可执行。

第2个位段表示硬链接数。第3、4位分别表示用户和组,-n选项让二者用数字而非名字表示。

第5位单位为字节。对于设备文件(b和c),显示为主从设备号。

第6位表示时间戳。文件的三个时间属性为修改时间(mtime),访问时间(atime)和i节点状态改变时间(ctime)。-l选项的默认时间戳为mtime。使用u选项使得时间戳使用atime,使用-c选项表示时间戳为ctime。

最后一位为文件名。

常用选项:
\begin{description}
	\item[R]递归输出。相当于\verb|`--recursive'|
	\item[F]文件名后面加上一个符号,表示文件类型。如/表示目录,@表示符号链接,|表示FIFO,=表示socket。普通文件后面什么也没有。相当于\verb|`--classify'|
	\item[d]表示显示的为目录本身的信息,而非其子目录或包含文件的信息。相当于\verb|`--directory'|
	\item[t]表示按照时间戳排序。相当于\verb|`--sort=time'|
	\item[S]表示按照size排序。相当于\verb|`--sort=size'|
	\item[r]将排序结果反转。相当于\verb|`--reverse'|
	\item[c]使用ctime表示时间。如果同-t选项一同使用,则按照ctime排序。相当于\verb$`--time=ctime'$或\verb$`--time=status'$.
	\item[u]使用atime表示时间。如果同-t选项一同使用,则按照atime排序。相当于\verb$`--time=atime'$或\verb$`--time=access'$或\verb$`--time=use'$
	\item[i]显示inode号。
\end{description}

如何打印当前路径下的文件夹:
\begin{verbatim}
ls -d */
find */ -type d -prune
ls -l|grep ^d|awk '{print $9}'
\end{verbatim}

如何打印当前路径下的纯文件:
\begin{verbatim}
find . -maxdepth 1 -type f
ls -l|grep ^-|awk '{print $9}'
\end{verbatim}

\subsection{grep}

Linux系统中grep命令是一种强大的文本搜索工具,它能使用正则表达式搜索文本,并把匹配的行打印出来。
grep全称是Global Regular Expression Print,表示全局正则表达式版本,它的使用权限是所有用户。

grep的选项:
\begin{description}
	\item[-c]只输出匹配行的计数
	\item[-i]不区分大小写(用于单字符)
	\item[-n]显示匹配的行号
	\item[-v]不显示不包含匹配文本的所以有行
	\item[-s]不显示错误信息
	\item[-E] 使用扩展正则表达式
\end{description}

举例:
\begin{verbatim}
grep 'ipp|ssn' netstat.txt -E
\end{verbatim}


\subsection{find}
\begin{verbatim}
find [-H] [-L] [-P] [-D debugopts] [-Olevel] [path...] [expression]
\end{verbatim}
H,L,P选项控制是否进入符号链接。D,O选项也不太常用。
expression包含选项、测试、行动三部分,因此find命令的使用方式通常为如下形式:

\verb+find 起始目录 寻找条件 操作+

寻找条件可以用and,or,not连接起来,分别写作:

\verb+ -a -o 和 !+

如

\verb+ find ! -name 'tmp'+

\verb+-prune选项相当与-maxdetph 0+,表示只对path参数包含的路径进行操作,不进行递归。

需要说明的是:当使用很多的逻辑选项时,可以用括号把这些选项括起来。为了避免Shell本身对括号引起误解,在话号前需要加转义字符来去除括号的意义。例:

\verb+find \(–name ’tmp’ -a type c -user ’inin’ \)+


只显示当前目录下的文件夹:

\verb+ls -l | grep ^d+

\verb+find * -type d -prune+

\verb+find . -maxdepth 1 -mindepth 1 -type d+

只显示当前目录下的非文件夹:

\verb+ls -l | grep -v ^d+




\subsection{xargs}
执行一条命令,其参数从标准输入获取。

其用途之一是用于将参数分成多行输入,类似于行末添加了反斜杠。例如输入xargs file,分多行输入file命令的参数,用CtrlD终止输入。

用途之二是构造管道,第一个命令的输入并不连接到第二个命令的输入,而是其命令行参数。如

\begin{verbatim}
ls | xargs file
find /tmp -name core -type f -print0 | xargs -0 /bin/rm -f
\end{verbatim}

find,xargs和wc命令配合使用可以统计目录下源代码行数:
\begin{verbatim}
find . -name "*.c" -o -name "*.h"|xargs wc -l
\end{verbatim}
这条命令统计了.h和.c文件中包含的行数。


\subsection{rsync}
Rsync(remote synchronize)是一个远程数据同步工具,可通过LAN/WAN快速同步多台主机间的文件。Rsync使用所谓的“Rsync算法”来使本地和远程两个主机之间的文件达到同步,这个算法只传送两个文件的不同部分,而不是每次都整份传送,因此速度相当快。

Rsync本来是用于替代rcp的一个工具,目前由rsync.samba.org维护,所以rsync.conf文件的格式类似于samba的主配置文件。Rsync可以通过rsh或ssh使用,也能以daemon模式去运行,在以daemon方式运行时Rsync server会打开一个873端口,等待客户端去连接。连接时,Rsync server会检查口令是否相符,若通过口令查核,则可以开始进行文件传输。第一次连通完成时,会把整份文件传输一次,以后则就只需进行增量备份。

Rsync用于单向同步,有一款双向\textbf{同步工具Unison}。

Rsync的基本特点如下:
\begin{enumerate}
\item 可以镜像保存整个目录树和文件系统
\item 可以很容易做到保持原来文件的权限、时间、软硬链接等
\item 无须特殊权限即可安装
\item 优化的流程,文件传输效率高
\item 可以使用rsh、ssh等方式来传输文件,当然也可以通过直接的socket连接
\item 支持匿名传输
\end{enumerate}

举例:
\begin{verbatim}
rsync -av `pwd` /media/D/Yunio/Projects
rsync -av `pwd` limz@192.168.10.10:~/Projects
\end{verbatim}

\subsection{sz and rz}

The \textbf{sz} and \textbf{rz} tools also come in handy. 
They are provided by \textbf{lrzsz} yum package.

\subsection{port scanning}
To check whether port 53 is open on a host,
\begin{verbatim}
nmap -Pn -p 53 202.100.4.15
\end{verbatim}


\subsection{compression and decompression:rar, zip, tar}

\href{https://en.wikipedia.org/wiki/RAR_(file_format)}{RAR} is a proprietary
archive file format that supports data compression, error recovery and file spanning.
Decompression source code is available, but it's not free software due to the
restriction that it must not be used to reverse engineer the RAR compression algorithm

\begin{verbatim}
rar a jpg.rar *.jpg 
rar a lua.rar -r lua
rar x file.rar 
\end{verbatim}

Linux zip command usage:
\begin{verbatim}
 zip options archive inpath inpath ...
\end{verbatim}

Examples:

\begin{verbatim}
zip jpg.zip *.jpg 
zip -r foo.zip foo
zip -r foo foo
unzip file.zip 
\end{verbatim}

We can also zip and unzip with Python's standard
\href{https://docs.python.org/2/library/zipfile.html}{zipfile} module:


\begin{verbatim}
$ python -m zipfile
Usage:
    zipfile.py -l zipfile.zip        # Show listing of a zipfile
    zipfile.py -t zipfile.zip        # Test if a zipfile is valid
    zipfile.py -e zipfile.zip target # Extract zipfile into target dir
    zipfile.py -c zipfile.zip src ... # Create zipfile from sources
\end{verbatim}


The \textbf{tar} tool has file sub-commands:
\begin{itemize}
  \item -c archive creation 
  \item -x archive decompression
  \item -t archive content listing
  \item -r archive append
  \item -u archive update
\end{itemize}

Optional parameters:
\begin{itemize}
    \item -z:gzip
    \item -j:bz2
    \item -Z:compress
    \item -v:show processing verbosely
\end{itemize}

Arhive name follows \textbf{-f} argument.

\begin{verbatim}
tar –cvf jpg.tar *.jpg 
tar –czf jpg.tar.gz *.jpg   
tar –cjf jpg.tar.bz2 *.jpg 
tar –cZf jpg.tar.Z *.jpg 
tar -xzvf file.tar.gz 
tar -xjvf file.tar.bz2 
\end{verbatim}



\subsection{date命令}
\begin{verbatim}
date  //显示本地时间
date -R //按照RFC-2822格式显示本地时间
date -u  //显示UTC时间
data -s $timestr //设置当前时间
date -s 20151021 //设置日期,日内时间置为零时
date -s 12:23:23 //设置具体时间,不会对日期做更改
date -s “12:12:23 2006-10-10″ //这样可以设置全部时间
date +"Day : %d Month : %m Year : %Y" // +号用于定制时间显示格式
date +%s //显示UNIX时间戳
date -d tomorrow //-d 或 --date选项,显示指定的时间而非当前时间
date -d '@1445396904' //显示UNIX时间戳指定的时间
date -d '2015-11-26'
date -d '2015-11-26 19:00:00'
date -d '3 hours'
date -d '1 month ago'
date -d 'last year' 
date -d 'next wed' #下周三
\end{verbatim}


显示格式定义:
\begin{verbatim}
%Y: 年(如2015)
%m: 月(01..12)
%d: 日(01..31)
%H: 小时(01..24)
%M: 分钟(00..59)
%S: 秒数(00..59)
%s: 秒数(UNIX时间)
%w: 星期几(0..6, 0表示星期日)
%b: 月份(Oct, 10月)
%B: 月份全称(October, 十月)

\end{verbatim}

\subsection{netstat}

netstat用法:
\begin{verbatim}
netstat [类型] [选项]
\end{verbatim}
默认类型为打印所有Open socket信息,类型-r表示路由表信息,-i表示网络接口信息,-g表示多播组信息,-s表示各协议统计信息。-c选项控制打印间隔。

对于Open socket信息,重要选项为:
\begin{description}
    \item  [a] 显示所有的socket,而不仅仅是监听套接字(默认为l)
    \item  [t] TCP only
    \item  [u] UDP only
    \item  [n] 不将数字解析成名字,使用该选项能加速程序输出
    \item  [p] 显示相关进程的PID和名字
    \item  [4] tcp4 and udp4 only
    \item  [6] tcp6 and udp6 only
\end{description}

在Mac OS X下,netstat不能提供进程名称信息,可采用lsof -i 命令来获取该信息。


\subsection{Generating Terminal Output}
\textbf{wall} and \textbf{write} can do that.

\subsection{tcpdump and Wireshark}

\begin{verbatim}
tcpdump -i eth0 -w file.pcap host 192.168.130.10 'tcp' 
tcpdump -r file.pcap >> file.txt
\end{verbatim}
注意,待写入的文件必须以pcap为后缀,否则tcpdump不能运行,报权限错误。pcap文件也可以用wireshark打开。

一些常用选项为:
\begin{itemize}
    \item 
        c选项指定抓包数量
    \item 
        s选项指定抓包长度(从二层开始)
\end{itemize}

注意如果s选项指定的长度超过帧长,则只抓帧长。所谓帧长,如果是在Linux下运行pcap程序,是不包括L2 FCS的(以太网尾部CRC),这和Smartbit等测试仪表不一样。sar也将FCS加入到计数器中,虽然Linux下的pcap抓不到FCS。

抓某个地址的ping包:
\begin{verbatim}
 tcpdump -i em2 host 172.28.11.131 and icmp
\end{verbatim}

抓HTTP包:
\begin{verbatim}
tcpdump -i eth0 tcp and port 80
\end{verbatim}

抓SYN,FIN,RST包(Unix网络编程29.2节BPF的例子),tcp头第13个字节为标志位字节。
\begin{verbatim}
tcpdump -i any 'tcp and port 80 and tcp[13:1] & 0x7 != 0'
\end{verbatim}


只接收SYN标志位置位且目标端口是22或23的数据包(tcp首部开始的第13个字节)
\begin{verbatim}
tcpdump 'tcp[13] == 0x02 and (dst port 22 or dst port 23) '
tcpdump tcp[13] == 0x02 and \(dst port 22 or dst port 23\)
\end{verbatim}

只接收icmp的ping请求和ping响应的数据包
\begin{verbatim}
tcpdump 'icmp[icmptype] == icmp-echoreply or icmp[icmptype] == icmp-echo'
tcpdump 'tcp[tcpflags] & (tcp-syn|tcp-fin) != 0 and not src and dst net localnet'
\end{verbatim}

The following commands allow any user to manipulate tcpdump without root privilege:
\begin{verbatim}
groupadd pcap
usermod -a -G pcap user
chgrp pcap /usr/sbin/tcpdump
chmod 750 /usr/sbin/tcpdump
setcap cap_net_raw,cap_net_admin=eip /usr/sbin/tcpdump
\end{verbatim}


Wireshark会判断操作系统是否能抓到FCS,但声称判断未必准确,可以手工告诉Wireshark一定有FCS。
有个很实用的功能是对一个包右键选择follow tcp connection,能够筛选出该包所属连接的所有包。

 参考文章:
 * \href{http://www.rationallyparanoid.com/articles/tcpdump.html}{Tcpdump usage examples}
 * \href{https://danielmiessler.com/study/tcpdump/}{A tcpdump Primer with  Examples}

Traffic can be filtered in Wireshark, like
\begin{verbatim}
ip.src == 172.28.3.95 and icmp
\end{verbatim}

\subsection{crontab}

\begin{verbatim}
MINUTE HOUR DAY MONTH DAY-OF-WEEK CMD
\end{verbatim}

\url{https://crontab.guru/} provides an online-editor for crontab entries.

\clearpage




