%!Mode:: "TeX:UTF-8"


\section{Linux运行与服务}


\subsection{rc scripts}

init.d/存放各种服务器和程序的二进制文件存放目录。

对每个运行级别x,rcx.d/存放各个启动级别的执行程序连接目录,里头的东西都是指向init.d/的一些软连接。

CentOS/Ubuntu下的/etc/rc.local脚本将在所有其他init脚本之后执行,用户可将自己想要开机执行的指令追加到这个脚本里。

rc是runcom(runcom应该是运行命令的意思)的简称。

/etc/inittab is only used by upstart for the default runlevel.
System initialization is started by /etc/init/rcS.conf.
 Individual runlevels are started by /etc/init/rc.conf
 Terminal gettys are handled by /etc/init/tty.conf and /etc/init/serial.conf,
 with configuration in /etc/sysconfig/init.



\subsection{systemd}

systemd的启动级别配置文件位于 /lib/systemd/system/目录下。

更改默认启动级别:
\begin{verbatim}
systemctl set-default TARGET.target
\end{verbatim}
查看默认启动级别:
\begin{verbatim}
systemctl get-default
\end{verbatim}

\subsection{xrdp}
\begin{verbatim}
systemctl start xrdp.service
 systemctl enable xrdp.service
 chcon -t bin_t /usr/sbin/xrdp
chcon -t bin_t /usr/sbin/xrdp-sesman
\end{verbatim}

\subsection{Ubuntu系统服务}

Ubuntu使用service命令来临时启动和停止服务。service是一个脚本,调用start,stop等可执行文件。

Ubuntu上可以使用工具update-rc.d或sysv-rc-conf来管理服务,但经尝试似乎效果不理想。

\begin{verbatim}
sysv-rc-conf --list 
sysv-rc-conf apache2 on
\end{verbatim}

\subsection{RHEL系统服务}
service命令运行\verb+/etc/init.d+目录下的System V脚本或\verb+/etc/init+目录下的upstart jobs。
这些脚本应至少支持start, stop参数,有些支持status, restart参数。

service runs a System V init script or upstart job in as predictable environment as possible, removing most environment variables and with current working directory set to \verb+/+.  

重启网络:
\begin{verbatim}
service network restart
\end{verbatim}

查看所有服务的状态
\begin{verbatim}
service --status-all
将所有支持status命令的服务运行一遍
\end{verbatim}


\subsection{at工具}
在CentOS及Ubuntu系统上,at命令所在的软件包为at。

atd所执行的内容由标准输入提供,或者通过-f选项指定一个脚本。
例如,如果需要立即执行“wall haha”命令:

\begin{verbatim}
echo wall haha|at now
\end{verbatim}

\subsection{cron工具}

cron执行时,也就是要读取三个地方的配置文件:一是/etc/crontab,二是/etc/cron.d目录下的所有文件,三是每个用户的配置文件. 


以下命令编辑用户crontab表:
\begin{verbatim}
crontab -e
\end{verbatim}

表的列依次表示:分,时,日,月,周几,命令。

亦可直接编辑/var/spool/cron/目录下的文件,某用户对应的crontab文件的文件名为其用户名。
\begin{verbatim}
echo "8 8 * * * /usr/sbin/logrotate /etc/logrotate.d/active-prober.conf" >> /var/spool/cron/root
\end{verbatim}

以下命令查看crontab表:
\begin{verbatim}
crontab -l
\end{verbatim}
