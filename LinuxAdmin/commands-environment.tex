%!Mode:: "TeX:UTF-8"
\section{系统信息查看}

\subsection{环境变量}
Linux系统里的env命令可以显示当前用户的环境变量,还可以用来在指定环境变量下执行其他命令。
下面来比较一下set,env和export命令的异同:set命令显示当前shell的变量,包括当前用户的变量;env命令显示当前用户的变量;export命令显示当前导出成用户变量的shell变量。
每个shell有自己特有的变量(set)显示的变量,这个和用户变量是不同的,当前用户变量和你用什么shell无关,不管你用什么shell都在,
比如HOME,SHELL等这些变量,但shell自己的变量不同shell是不同的,比如BASH\_ARGC,BASH等,这些变量只有set才会显示,
是bash特有的,export不加参数的时候,显示哪些变量被导出成了用户变量,因为一个shell自己的变量可以通过export “导出”变成一个用户变量。

\subsection{外围设备}
lspci可列举出所有PCI设备。例如,下面的命令能够列举出所有的以太网设备:
\begin{verbatim}
lspci |grep -i ether
\end{verbatim}

如欲查看本机所有的网络接口,可查看/sys/class/net目录。



\subsection{主机名称}
hostname命令可以查询主机名称。但是很多主机的名称都取作localhost。
Ubuntu主机名需要在/etc目录下hostname和hosts两个文件下修改,Centos6.4主机名通过/etc/sysconfig/network文件修改。

\begin{lstlisting}
	HOSTNAME=vss-dev1   
\end{lstlisting}


\subsection{IP地址}

下面两个Python函数分别获得了指定网口的IP地址和同外界连接的IP地址。
\begin{verbatim}
import socket, fcntl, struct
 
def get_inf_ip_addr(ifname):
    s = socket.socket(socket.AF_INET, socket.SOCK_DGRAM)
    return socket.inet_ntoa(fcntl.ioctl(
        s.fileno(),
        0x8915,  # SIOCGIFADDR
        struct.pack('256s', ifname[:15])
    )[20:24])
    

def get_localhost_ip_addr():
    return [i[4][0] for i in socket.getaddrinfo(socket.gethostname(), None, 2, 1, 6)]

print get_inf_ip_addr("eth0")
print get_localhost_ip_addr()

\end{verbatim}


\subsection{Linux内核版本}
\begin{verbatim}
cat /proc/version:查询内核信息
uname -r:查询Linux内核版本号
uname -a:查询内核信息
\end{verbatim}
注意不要用file命令通过查看可执行文件的信息来判断当前主机的内核版本号,二者关系复杂

\subsection*{Linux发行版信息}
\begin{verbatim}
lsb_release -a 
cat /etc/issue
cat /etc/redhat-release
rpm -q redhat-release:查看Redhat release号
\end{verbatim}

\subsection*{查看字长:32/64位}
\begin{verbatim}
getconf LONG_BIT
getconf WORD_BIT
file /bin/ls
lsb_release -a
\end{verbatim}

\subsection*{CPU信息}
\subsubsection{CPU架构}
\begin{verbatim}
 lscpu|grep Arch
 uname -m
 arch
\end{verbatim}

\subsubsection*{CPU详细信息}
\begin{verbatim}
cat /proc/cpuinfo
lscpu
\end{verbatim}

\subsubsection*{CPU数与核数}
Linux下可以在/proc/cpuinfo中看到每个cpu的详细信息。但是对于双核的cpu,在cpuinfo中会看到两个cpu。常常会让人误以为是两个单核的cpu。
其实应该通过Physical Processor ID来区分单核和双核。而Physical Processor ID可以从cpuinfo或者dmesg中找到.
flags如果有ht说明支持超线程技术。判断物理CPU的个数可以查看physical id 的值,相同则为同一个物理CPU。
\begin{verbatim}
 cat /proc/cpuinfo |grep "physical id"
 lscpu|grep Core
\end{verbatim}

\subsubsection*{CPU型号}
\begin{verbatim}
cat /proc/cpuinfo |grep "model name"
\end{verbatim}

\subsection*{内存信息}
\begin{verbatim}
cat /proc/meminfo |grep MemTotal
sudo dmidecode |grep -A16 "Memory Device$"
free -m
\end{verbatim}

\subsection*{硬盘大小}
\begin{verbatim}
fdisk -l |grep Disk
\end{verbatim}
df命令主要用来查询文件系统信息,可以用来查看硬盘以挂载的分区的大小
\begin{verbatim}
df -h
\end{verbatim}

\subsection{主板型号}
\begin{verbatim}
sudo dmidecode | grep -A16 "System Information$"
cat /proc/pci
\end{verbatim}

\subsection{查看Linux系统安装的时间}

1. 查看 /lost+found 目录状态,因为这个目录一般很少用到,改动最少(很可能无任何改动),而其他目录比如 /bin, /home 等因为经常升级系统、创建用户等操作会修改目录状态。

2. 查看 bin, daemon, sys, adm 等这些帐号的建立时间:\verb$ # passwd -S bin$。这些帐号是在系统安装的时候创建的。然而有些发行版在安装时是直接从安装光盘拷贝这些账号和文件,此时该方法行不通。



\subsection{CPU温度}
CPU温度通过/sys/class/hwmon目录给出。
对于一个有双CPU槽的机器,第2个CPU的信息可能位于/sys/class/hwmon/hwmon2/device目录。
整个槽的温度位于\verb|temp1_input|文件,如果文件内容为560000,表示温度为56摄氏度。
该CPU第一个核的温度位于\verb|temp2_input|文件。

lm-sensors软件包提供了查看温度的便捷工具sensors。其在CentOS下的用法如:
\begin{verbatim}
yum install lm_sensors
sensors-detect
sensors
\end{verbatim}
在Ubuntu下如:
\begin{verbatim}
apt-get install lm-sensors
sensors-detect
service module-init-tools start
sensors
\end{verbatim}










