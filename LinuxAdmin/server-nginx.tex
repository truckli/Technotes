%!Mode:: "TeX:UTF-8"
\section{Nginx配置}

Linux系统下搭建Web服务器的一个简易方法是使用Python的SimpleHTTPServer模块:

\begin{verbatim}
python -m SimpleHTTPServer 8080
\end{verbatim}

CentOS的epel仓库中包含了一个nginx软件包,可通过yum从epel中安装nginx:
\begin{verbatim}
yum -y install epel-release
yum -y install nginx
service nginx start
\end{verbatim}

在Ubuntu 14.04下使用apt-get安装后,可执行文件在/usr/bin目录,
配置文件在/etc/nginx目录,网页文件在/usr/share/nginx/html目录。

核心的配置文件为nginx.conf,其可用include指令包含任意其他配置文件,默认包含conf.d子目录下的文件。

使用文件拷贝来安装配置Nginx(前提是configure时指定prefix局限在一个用户目录中),需要修改某些配置以便同别人区分开:
更改server块下listen配置,修改端口号;更改pid配置,修改pid文件路径;修改\verb$lua_packet_path$这种参数。



配置代理服务器
\begin{verbatim}
server {
        listen 8000;
        root html;
        index index.html index.htm;

        location / { 
                proxy_pass http://127.0.0.1:8082;    
        }
}

\end{verbatim}
