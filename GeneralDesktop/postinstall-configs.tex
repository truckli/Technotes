%!Mode:: "TeX:UTF-8"

\section{Basic Configurations After OS Installation}


\subsection{Ubuntu自动选源}
\begin{verbatim}
deb mirror://mirrors.ubuntu.com/mirrors.txt precise main restricted universe multiverse
deb mirror://mirrors.ubuntu.com/mirrors.txt precise-updates main restricted universe multiverse
deb mirror://mirrors.ubuntu.com/mirrors.txt precise-backports main restricted universe multiverse
deb mirror://mirrors.ubuntu.com/mirrors.txt precise-security main restricted universe multiverse
\end{verbatim}

\subsection{设置时区}
\begin{verbatim}
cp /usr/share/zoneinfo/Asia/Shanghai /etc/localtime 
\end{verbatim}

\subsection{网络对时}
一次性对时,可采用ntp服务或ntpdate程序:
\begin{verbatim}
ntpd -q -p cn.pool.ntp.org
ntpdate ntp.sjtu.edu.cn
ntpdate cn.pool.ntp.org
ntpdate 3.asia.pool.ntp.org
ntpdate time.windows.com
ntpdate 0.pool.ntp.org
ntpdate 1.pool.ntp.org
ntpdate time.nist.gov
ntpdate wwv.nist.gov
ntpdate s1a.time.edu.cn 
\end{verbatim}

另外,需要经常性地调整本机时间,确保不再次出现偏差。这需要在系统上部署ntp服务。
ntp服务维护了一个远程NTP服务器列表,自动完成经常性的对时功能,且对系统时间的修改比较缓和。

\begin{verbatim}
yum -y install ntp
service ntpd start
chkconfig ntpd on
\end{verbatim}

ntp服务的配置文件位于/etc/ntp.conf目录,可适当对该文件进行修改,如追加ntp服务器列表。

参考:

\href{pool.ntp.org中国服务器列表}{http://www.pool.ntp.org/zone/cn}

\href{http://tf.nist.gov/tf-cgi/servers.cgi} { NIST Internet Time Servers }



\subsection{CentOS network configuration}
/etc/sysconfig/network-scripts目录下设置各个网卡,注意ONBOOT选项。

一个例子,ifcfg-enp0s3:
\begin{verbatim}
TYPE=Ethernet
BOOTPROTO=none
DEFROUTE=yes
IPV4_FAILURE_FATAL=no
IPV6INIT=yes
IPV6_AUTOCONF=yes
IPV6_DEFROUTE=yes
IPV6_FAILURE_FATAL=no
NAME=enp0s3
UUID=00f4f6d9-f69d-4d52-92ae-b44adfc16038
DEVICE=enp0s3
ONBOOT=yes
IPADDR=172.28.3.95
PREFIX=24
GATEWAY=172.28.3.1
DNS1=159.226.59.158
DNS2=159.226.8.6
IPV6_PEERDNS=yes
IPV6_PEERROUTES=yes
\end{verbatim}


DNS可以在network-scripts中配置,也可在/etc/resolv.conf中配置。
如果选择后者,需在network-scripts中设置PEERDNS=NO。

如果将网卡设置成DHCP,只需将BOOTPROTO设置成dhcp,并删除IPADDR等信息。

\subsection{Mac OS X配置}
Go2Shell选择打开的终端
\begin{verbatim}
open -a Go2Shell --args config
\end{verbatim}

显示 Mac 隐藏文件的命令:
 
defaults write com.apple.finder AppleShowAllFiles -bool true
隐藏 Mac 隐藏文件的命令:
 
defaults write com.apple.finder AppleShowAllFiles -bool false

sshpass
brew install https://raw.github.com/eugeneoden/homebrew/eca9de1/Library/Formula/sshpass.rb


\subsection{重装备份}




重装系统时,需要备份的文件包括:
\begin{verbatim}
/etc/hosts
/etc/fstab
/etc/lightdm/lightdm.conf

.bash*
.vimrc
.gitconfig
.ssh/*
.config/Chromium/User*
\end{verbatim}


\subsection{安装版XP}
使用安装版CD安装Windows XP,大致需要四五十分钟。其中,二十分钟后会提示输入地域、账户、序列号等信息。余下不需要刻意关注。

在实验室电脑安装KVM版虚拟XP时,没有额外安装任何驱动程序。显示外观在全屏时比较模糊。设置以1280*800分辨率使得全屏时恰好覆盖整个屏幕。

如果是在PC上安装真实XP,第一件是安装网卡驱动和其他驱动,推荐使用自带网卡驱动的驱动精灵。此外,我至少需安装的软件包括360浏览器和安全卫士,输入法, Chrome浏览器。然后运行激活工具。

\subsection{克隆版XP安装}
要点有二:MBR和设置活动分区。

克隆版在复制C盘数据的过程中不更新MBR。因此,对于MBR不需要改动的情形,克隆版可以顺利安装。否则,就需要改动MBR。可以使用WinPE安装XP,PE下带有分区管理工具,在分区后可以先设置C盘为活动分区。然后使用本工具或另一款分区修复工具,更新MBR。



\subsection{VirtualBox安装XP虚拟机}
使用虚拟机安装克隆版XP经常发生蓝屏错误,提示processor.sys出现了问题。
可以通过PE系统修改\verb+C:/windows/system32/drivers/processor.sys+的名称,使系统找不到这个文件。

对于某些机构IP和MAC绑定上网的情形,VirtualBox联网方式设置成NAT方式。
在虚拟机里无需更改MAC地址,只需要设置正确的DNS服务器地址就好。

\subsection{虚拟机共享文件夹}
假如共享目录命名为\verb$public_dir$。
那么在VBox的Linux客户机中执行\verb$sudo /sbin/mount.vboxsf public_dir /mnt/$。

或者在/etc/fstab中设置自动加载:
\begin{verbatim}
sharename   mountpoint   vboxsf   defaults  0   0
\end{verbatim}
VirtualBox的详细设置方式见\url{https://www.virtualbox.org/manual/ch04.html#sharedfolders}。

在Vmware Linux Guest下,可在/mnt/hgfs下看到共享目录。如果看不到,执行:

\begin{verbatim}
sudo vmware-config-tools.pl
\end{verbatim}

Vmware虚拟机使用共享文件夹的方式详见\href{https://www.vmware.com/support/ws5/doc/ws_running_shared_folders.html}{VMware
Workstation 5.0 Using Shared Folders}。












