%!Mode:: "TeX:UTF-8"
\section{启动与登陆}
\subsection{ISO镜像与光盘}
在Ubuntu下将文件夹做成iso镜像的工具叫做genisoimage(原名mkisofs)。可引导光盘加载后对相应文件夹使用genisoimage工具,会失去引导性。如欲保留引导性,可以使用dd工具对裸设备进行操作。如:

\verb+dd if=/dev/sr0 of=deepin.iso+

如需将iso烧录入光盘,可以使用Brasero工具。这很可能是Ubuntu系统自带的。


\subsection{USB启动盘}

可以使用dd命令或使用一些图形工具。在维基百科上有一个制作LiveUSB的软件列表。包括:
\begin{itemize}
  \item Fedora Liveusb-creator,在Windows或Linux下制作Fedora的LiveUSB
  \item Ubuntu Liveusb Creator(在命令为usb-creator-gtk或usb-creator-kde)
  \item LinuxLive USB Creator,在Windows上制作Linux LiveUSB
  \item Unetbootin,在Windows或Linux下制作UNIX LiveUSB,Ubuntu软件仓库有提供
\end{itemize}
我自己用优盘实际创建启动盘,在Ubuntu下使用Ubuntu Liveusb Creator制作失败了。用dd和Unetbootin制作成功。

如果使用dd命令,先umount这个USB设备,其名称可能是sdb1。但dd命令要作用与sdb而不是sdb1上。
\begin{verbatim}
dd if=linux_image.iso of=/dev/sdb
\end{verbatim}
尝试使用dd命令将Windows安装镜像做成启动盘,未能成功,U盘在启动时未能引导,依然由硬盘进行了引导。

制作完成后,需要更改BIOS设置,让USB-HDD(或其他USB设备类型)的设备启动优先级高于HDD。优盘可能被当作硬盘处理,如果是这种优盘,开机时需要在BIOS设置中更改的不是设备启动顺序,而是HDD启动优先级。在实验室电脑上DEL键进入BIOS设置。


\subsection{修改Grub配置}
配置/etc/default/grub文件,然后运行update-grub,会自动生成/boot/grub/grub.cfg文件
\subsection{安装Grub}
如果安装了多个Linux系统,最后一个安装的系统的GRUB会覆盖之前安装的GRUB。
如果这不是所希望的,可以进入心目中的默认系统重新安装GRUB。如果最后安装的是Windows,
则GRUB会被破坏,不会在开机时显示GRUB界面,也需要设法进入Linux后重新安装GRUB。
进入所希望的Linux系统后,执行grub-install。
 grub-install 将GRUB镜像复制到/boot/grub, 并调用grub-setup来将grub安装到boot扇区。
 如果是从LiveCD进入的系统,执行grub-install前先chroot。
 \begin{verbatim}
   sudo mount /dev/sdaX /mnt/root
   sudo mount --rbind /dev /mnt/root/dev
   sudo mount --rbind /proc /mnt/root/proc
   sudo chroot /mnt/root grub-install /dev/sda
 \end{verbatim}
 
 
如果没有LiveCD,在进入系统前停留在了如下界面上:
\begin{verbatim}
GRUB loading
error:unknow filesystem
grub rescue>
\end{verbatim}
可以执行ls命令,找到Linux被安装在了哪个分区。如果确定了分区,比如,(hd0,5),则执行\verb+ls (hd0,5)+时会看到许多mod文件。
安装normal.mod:
\begin{verbatim}
grub rescue>set root=(hd0,5)
grub rescue>set prefix=(hd0,5)/boot/grub
grub rescue>insmod /boot/grub/normal.mod
\end{verbatim}
执行normal命令,可以恢复Grub界面。进入Linux后重新安装GRUB即可。


\subsection{登陆密码恢复}
方法如下:

1、重新启动,按ESC键进入Boot Menu,选择recovery mode(一般是第二个选项)。

2、在\#号提示符下用cat /etc/shadow,看看用户名。

3、输入passwd "用户名"(引号要有的哦)。

4、输入新的密码.

5、重新启动,用新密码登录。 






