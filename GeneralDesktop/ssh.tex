%!Mode:: "TeX:UTF-8"
\section{SSH}

\subsection{login and execute commands}
\begin{lstlisting}[language=bash]
   ssh HOSTNAME -t ``cd /somedir; bash'' 
\end{lstlisting}

\subsection{login without passwords}

First, generate a pair of RSA keys and store them on ~/.ssh.
\begin{verbatim}
ssh-keygen -t rsa
\end{verbatim}

Second, copy the public key file to a target host.
\begin{verbatim}
ssh-copy-id -i ~/.ssh/id_rsa.pub root@172.30.30.101
\end{verbatim}


\subsection{ssh tunneling}

Port forwarding:
\begin{verbatim}
 ssh -p 22 -L 8443:10.139.90.144:8443 -L 8080:10.139.90.148:8080 -L
13306:10.139.90.148:3306 -N -lroot 10.9.0.5
\end{verbatim}

\href{http://ju.outofmemory.cn/entry/123978}{svn使用socks代理}

\begin{verbatim}
http-proxy-host = 127.0.0.1
http-proxy-port = 12007
http-compression = yes
\end{verbatim}

\subsection{screen}

screen is used to manage shell sessions.

Enter a new or existing session:

\begin{verbatim}
 screen -dR screen_name
\end{verbatim}

Show existing sessions:

\begin{verbatim}
 screen -ls
\end{verbatim}
 

Enter an existing session:

\begin{verbatim}
 screen -r screen_name
\end{verbatim}
 
 
Session quit(detach):
\begin{verbatim}
 screen -d screen_name
 screen -d 
\end{verbatim}

To terminate, i.e., delete the current session, type \textbf{Ctrl + D}, or use:
\begin{verbatim}
screen -X -S screen_name quit 
\end{verbatim}



To activate a screen command, type \textbf{Ctrl+A}, following a command key:

\begin{description}
\item[D] quit a session
\end{description}


screen cli options descriptions:

\begin{description}
\item[-d] Detach the elsewhere running screen
\item[-R] Reattach if possible, otherwise start a new session
\item[-r session] Reattach to a detached screen process
\item[-X] Execute <cmd> as a screen command in the specified session
\item[-S sockname]   Name this session <pid>.sockname instead of <pid>.<tty>.<host>.
\end{description}












