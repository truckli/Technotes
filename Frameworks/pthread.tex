%!Mode:: "TeX:UTF-8"
\section{pthread接口}

\subsection{数据类型}
\begin{verbatim}
pthread_t:线程句柄
pthread_attr_t:线程属性
pthread_mutex_t:互斥量
pthread_cond_t:条件变量
\end{verbatim}
每个线程都有errno副本。

\subsection{操纵函数}
\begin{verbatim}
pthread_create():创建一个线程
pthread_exit():终止当前线程
pthread_cancel():中断另外一个线程的运行
pthread_join():阻塞当前的线程,直到另外一个线程运行结束
pthread_attr_init():初始化线程的属性
pthread_attr_setdetachstate():设置脱离状态的属性(决定这个线程在终止时是否可以被结合)
pthread_attr_getdetachstate():获取脱离状态的属性
pthread_attr_destroy():删除线程的属性
pthread_kill():向线程发送一个信号
\end{verbatim}


\subsection{同步函数}
\begin{verbatim}
pthread_mutex_init() 初始化互斥锁
pthread_mutex_destroy() 删除互斥锁
pthread_mutex_lock():占有互斥锁(阻塞操作)
pthread_mutex_trylock():试图占有互斥锁(不阻塞操作)。即,当互斥锁空闲时,将占有该锁;否则,立即返回。
pthread_mutex_unlock(): 释放互斥锁
pthread_cond_init():初始化条件变量
pthread_cond_destroy():销毁条件变量
pthread_cond_signal(): 唤醒第一个调用pthread_cond_wait()而进入睡眠的线程
pthread_cond_wait(): 等待条件变量的特殊条件发生
Thread-local storage(或者以Pthreads术语,称作线程特有数据):
pthread_key_create(): 分配用于标识进程中线程特定数据的键
pthread_setspecific(): 为指定线程特定数据键设置线程特定绑定
pthread_getspecific(): 获取调用线程的键绑定,并将该绑定存储在 value 指向的位置中
pthread_key_delete(): 销毁现有线程特定数据键
pthread_attr_getschedparam();获取线程优先级
pthread_attr_setschedparam();设置线程优先级
\end{verbatim}

\subsection{工具函数}
\begin{verbatim}
pthread_equal(): 对两个线程的线程标识号进行比较
pthread_detach(): 分离线程
pthread_self(): 查询线程自身线程标识号
\end{verbatim}








