%!Mode:: "TeX:UTF-8"
 
\section{进程/线程与CPU核的绑定}
\verb$sched_setaffinity$实现进程与核的绑定。

\begin{lstlisting}[language=C++]
int pthread_setaffinity_np(pthread_t thread, size_t cpusetsize, const  cpu_set_t *cpuset);
int pthread_getaffinity_np(pthread_t thread, size_t cpusetsize, cpu_set_t  *cpuset);
\end{lstlisting}
分别设置和获取线程的亲和性。

\verb$cpu_set_t$类似于select中的\verb$fd_set$可以理解为cpu集,也是通过约定好的宏来进行清除、设置以及判断:
\begin{lstlisting}[language=C++]
//初始化,设为空
void CPU_ZERO (cpu_set_t *set); 
//将某个cpu加入cpu集中 
void CPU_SET (int cpu, cpu_set_t *set); 
//将某个cpu从cpu集中移出 
void CPU_CLR (int cpu, cpu_set_t *set); 
//判断某个cpu是否已在cpu集中设置了 
int CPU_ISSET (int cpu, const cpu_set_t *set); 
\end{lstlisting}

cpu集可以认为是一个掩码,每个设置的位都对应一个可以合法调度的 cpu,而未设置的位则对应一个不可调度的 CPU。
换而言之,线程都被绑定了,只能在那些对应位被设置了的处理器上运行。通常,掩码中的所有位都被置位了,也就是可以在所有的cpu中调度。       
      
以下为测试代码:

\begin{lstlisting}[language=C++]
#define _GNU_SOURCE
#include <stdio.h>
#include <stdlib.h>
#include <string.h>
#include <unistd.h>
#include <pthread.h>
#include <sched.h>

void *myfun(void *arg)
{
    cpu_set_t mask;
    cpu_set_t get;
    char buf[256];
    int i;
    int j;
    int num = sysconf(_SC_NPROCESSORS_CONF);<span style="white-space:pre">	</span>//统计cpu个数
    printf("system has %d processor(s)\n", num);

    for (i = 0; i < num; i++) {
        CPU_ZERO(&mask);
        CPU_SET(i, &mask);
        if (pthread_setaffinity_np(pthread_self(), sizeof(mask), &mask) < 0) {
            fprintf(stderr, "set thread affinity failed\n");
        }
        CPU_ZERO(&get);
        if (pthread_getaffinity_np(pthread_self(), sizeof(get), &get) < 0) {
            fprintf(stderr, "get thread affinity failed\n");
        }
        for (j = 0; j < num; j++) {
            if (CPU_ISSET(j, &get)) {
                printf("thread %d is running in processor %d\n", (int)pthread_self(), j);
            }
        }
        j = 0;
        while (j++ < 100000000) {
            memset(buf, 0, sizeof(buf));
        }
    }
    pthread_exit(NULL);
}

int main(int argc, char *argv[])
{
    pthread_t tid;
    if (pthread_create(&tid, NULL, (void *)myfun, NULL) != 0) {
        fprintf(stderr, "thread create failed\n");
        return -1;
    }
    pthread_join(tid, NULL);
    return 0;
}
\end{lstlisting}





