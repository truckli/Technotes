%!Mode:: "TeX:UTF-8"
\section{Java Frameworks}


\href{https://en.wikipedia.org/wiki/Java_Platform,_Enterprise_Edition}{Java
Platform, Enterprise Edition or Java EE} is a widely used enterprise computing
The platform provides an API and runtime environment for developing and running
enterprise software, including network and web services, and other large-scale, multi-tiered, scalable, reliable, and secure network applications. 
Java EE extends the Java Platform, Standard Edition (Java SE), providing an API for object-relational mapping, distributed and multi-tier architectures, and web services.

The \href{https://en.wikipedia.org/wiki/Spring_Framework}{Spring Framework} is
an application framework and inversion of control (IoC) container for the Java platform.
The framework's core features can be used by any Java application, but there are extensions for building web applications on top of the Java EE platform. 
Although the framework does not impose any specific programming model, 
it has become popular in the Java community as an alternative to, replacement for, or even addition to the Enterprise JavaBeans (EJB) model. 
The Spring Framework is open source.


\href{https://en.wikipedia.org/wiki/Enterprise_JavaBeans}{Enterprise JavaBeans (EJB)} 
is a managed, server software for modular construction of enterprise
software, and one of several Java APIs.
EJB is a server-side software component that encapsulates the business logic of an application.
The EJB specification is a subset of the Java EE specification. 
An EJB web container provides a runtime environment for web related software components, including computer security, Java servlet lifecycle management, transaction processing, and other web services.

The complexity issue continued to hinder EJB's acceptance.
\href{https://en.wikipedia.org/wiki/Hibernate_(framework)}{Hibernate} (for persistence and object-relational mapping) and 
\href{https://en.wikipedia.org/wiki/Spring_Framework}{Spring Framework} (which provided an alternate and far less verbose way to encode
business logic) were created to replace it and grew in popularity, despite
lacking the support of big businesses.


\href{https://en.wikipedia.org/wiki/Hibernate_(framework)}{Hibernate ORM} 
(Hibernate in short) is an object-relational mapping framework for the Java language. 
It provides a framework for mapping an object-oriented domain model to a relational database. 
Hibernate solves object-relational impedance mismatch problems by replacing direct, persistent database accesses with high-level object handling functions.
Hibernate's primary feature is mapping from Java classes to database tables; and
mapping from Java data types to SQL data types. Hibernate also provides data query and retrieval facilities. It generates SQL calls and relieves the developer from manual handling and object conversion of the result set.

\href{https://en.wikipedia.org/wiki/Apache_Struts_2}{Apache Struts 2} is an
open-source web application framework for developing Java EE web applications. 
It uses and extends the Java Servlet API to encourage developers to adopt a model–view–controller (MVC) architecture.
 It favors convention over configuration, is extensible using a plugin architecture, and ships with plugins to support REST, AJAX and JSON.

The integration of Struts, Spring and Hibernate is often abbreviated as
\textbf{SSH}.



\href{https://en.wikipedia.org/wiki/Non-blocking_I/O_(Java)}{Non-blocking I/O}
(usually called \textbf{NIO}, and sometimes called "New I/O") is a collection of
Java programming language APIs that offer features for intensive I/O operations.
It was introduced with the J2SE 1.4 release of Java by Sun Microsystems to complement an existing standard I/O.

\href{https://en.wikipedia.org/wiki/Netty_(software)}{Netty} is a NIO
client-server framework for the development of Java network applications such as protocol servers and clients. 
The asynchronous event-driven network application framework and tools are used to simplify network programming such as TCP and UDP socket servers.
Netty includes an implementation of the reactor pattern of programming.

\href{https://en.wikipedia.org/wiki/Apache_MINA}{Apache MINA} (Multipurpose
Infrastructure for Network Applications) is an open source Java network application framework. MINA can be used to create scalable, high performance network applications. 
MINA provides unified APIs for various transports like TCP, UDP, serial communication. 
It also makes it easy to make an implementation of custom transport type. 
MINA provides both high-level and low-level network APIs.










