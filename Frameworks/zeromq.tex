%!Mode:: "TeX:UTF-8"
\section{Zeromq}

\href{https://en.wikipedia.org/wiki/ZeroMQ}{ZeroMQ} (also known as zmq) looks
like an embeddable networking library but acts like a concurrency framework. It gives you sockets that carry atomic messages across various transports like in-process, inter-process, TCP, and multicast. You can connect sockets N-to-N with patterns like fan-out, pub-sub, task distribution, and request-reply. It's fast enough to be the fabric for clustered products. 
Its asynchronous I/O model gives you scalable multicore applications, built as asynchronous message-processing tasks. 
 It provides a message queue, but unlike
 \href{https://en.wikipedia.org/wiki/Message-oriented_middleware}{message-oriented
 middleware}, a ZeroMQ system can run without a dedicated message broker. The
 library's API is designed to resemble that of Berkeley sockets.

 A basic
 \href{http://zguide.zeromq.org/py:hwclient}{client}
 and \href{http://zguide.zeromq.org/py:hwserver}{server} 
 program in Python can be found in the official site.

 