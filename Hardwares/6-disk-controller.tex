%!Mode:: "TeX:UTF-8"
\section{磁盘控制器技术}


\subsection{IDE与ATA}
IDE(Integrated Drive Electronics)是一种计算机系统接口,主要用于硬盘和CD-ROM,
本意为“把控制器与盘体集成在一起的硬盘”,
与ATA(Advanced Technology Attachment)关系密切。
数年以前PC主机使用的硬盘,大多数都是IDE兼容的,只需用一根电缆将它们与主板或适配器连起来就可以了,而目前主要接口为SATA接口。
一般说来,ATA是一个控制器技术,而IDE是一个匹配它的磁盘驱动器技术,但是两个术语经常可以互用。
ATA是一个花费低而性能适中的接口,主要是针对台式机而设计的,销售的大多数ATA控制器和IDE磁盘都是更高版本的,
称为ATA - 2和ATA - 3,与之匹配的磁盘驱动器称为增强的IDE。


把盘体与控制器集成在一起的做法,减少了硬盘接口的电缆数目与长度,数据传输的可靠性得到了增强,硬盘制造起来变得更容易,
因为厂商不需要再担心自己的硬盘是否与其他厂商生产的控制器兼容,对用户而言,硬盘安装起来也更为方便。
ATA是用传统的40-pin并行数据线连接主板与硬盘的,外部接口速度最大为133MB/s,因为并行线的抗干扰性太差,且排线占空间,不利散热,而逐渐被SATA所取代。
ATA主机控制器芯片差不多集成到每一个生产的系统板,提供连接4个设备的能力。
ATA控制器已经变得非常廉价和常见,但在SATA技术日益发展下,没有ATA的主版已经出现,
而且Intel在新型的芯片组中已经不默认支持ATA接口,主机版厂商需要另加芯片去对ATA作出支持(通常是为了兼容旧有硬盘和光盘驱动器)。

普遍情况下,一块主板只有两个IDE接口,每个接口可以挂两个IDE设备。
但同一个接口的两个设备是共用带宽的,对速度的影响非常大。所以稍有常识的人,都会把硬盘和光驱分开两条IDE线连接到主板上 
这样,IDE有个很大的问题,就是虽然一块主板可以连接4个设备,但事实上只要超过两个,速度就大大下降。
更大的问题是,同一条线上两个设备要严格按主/从设置才能正常运行。

并行ATA在支持设备热插拔方面能力有限,这一点对服务器方面的应用非常重要。因为服务器通常采用RAID的方式,
任何一块硬盘坏了都可以热拔插更换,而不影响数据的完整性,确保服务器任何情况下都正常开着。
具有热插拔支持功能的SCSI和光纤通道占据了企业级应用的几乎全部市场,
并行ATA空有价格优势而不能获得一席之地,主要原因就是它不支持热拔插。 


\subsection{SCSI}
小型计算机系统接口(SCSI,Small Computer System Interface)是一种用于计算机及其周边设备之间(硬盘、软驱、光驱、打印机、扫描仪等)系统级接口的独立处理器标准。
SCSI标准定义了命令、通信协定以及实体的电气特性(换成OSI的说法,就是占据了实体层、链接层、通信层、应用层),
最大部份的应用是在存储设备上(例如硬盘、磁带机),但SCSI可以连接的设备也包括扫描仪、光学设备(像CD、DVD)、打印机等等,

系统中的每个SCSI设备都必须有自己唯一的ID(标识号),在8-bit总线上,这个号码是0~7;在16-bit总线上,这个号码从0~15。
SCSI Adapter系统默认ID为7。
SCSI链的最后一个SCSI设备要用终结器,中间设备是不需要终结器的。
一旦中间设备使用了终结器,那么SCSI卡就无法找到以后的SCSI设备了。

SCSI-1是最初版本的SCSI,现已过时。
SCSI-1具有8位BUS,数据传输率为40 Mbps(5 MB/sec)。
SCSI-2是基于CCS的SCSI-1改进版本。
在Fast SCSI和Wide SCSI的支持下,SCSI-2在原SCSI-1的基础上传输速率得到了提高。
Fast SCSI的传输速率为10 MB/sec,当配合16位BUS时,其传输速率为20 MB/sec(Fast-Wide SCSI)。
当今,SCSI-3单元采用Ultra-Wide和Ultra SCSI类型的驱动器。
Ultra SCSI具有8位BUS,其传输速率为20 MB/sec。
Ultra-Wide SCSI具有16位BUS,其传输速率达到40 MB/sec。 
SCSI-3在SCSI-2基础上有了很多提高,如串行SCSI。通过6芯同轴电缆,其传输速率达到100 MB/sec。
SCSI-3解决了旧SCSI版本中存在的终结和延迟问题,
此外通过即插即用(plug-and-play)操作,自动分配SCSI ID和终结,使SCSI安装更为容易。
与SCSI-2支持8台设备相比,SCSI-3能支持32台设备。

同SCSI相比,IDE还具有性能价格比高、适用面广等特点。
个人电脑用户不但需要配置的外设不多,而且对速度要求也不高,因此选用IDE接口比SCSI更合适些。
SCSI相比于IDE的优势包括:
1)
IDE的工作方式需要CPU的全程参与,CPU读写数据的时候不能再进行其他操作,
而SCSI接口,则完全通过独立的高速的SCSI卡来控制数据的读写操作。
IDE接口为改善这个问题也做了很大改进,已经可以使用DMA模式而非PIO模式来读写,
对CPU的占用可大大减小。尽管如此,比较SCSI和IDE在CPU的占用率,还是可以发现SCSI仍具有相当的优势。
2)SCSI的扩充性比IDE大,一般每个IDE系统可有2个IDE通道,总共连4个IDE设备,而SCSI接口可连接7—15个设备,比IDE要多很多,而且连接的电缆也远长于IDE。
3)虽然SCSI设备价格高些,与IDE相比,SCSI的性能更稳定、耐用,可靠性也更好。
4)SCSI还允许在对一个设备传输数据的同时,另一个设备对其进行数据查找。
这就可以在多任务操作系统如Linux、WindowsNT中获得更高的性能。


\subsection{SATA}
Serial ATA(SATA, 串行ATA,Serial Advanced Technology Attachment),
是串行SCSI(SAS:Serial Attached SCSI)的孪生兄弟,两者的排线兼容,SATA硬盘可接上SAS接口。
它是一种电脑总线,主要功能是用作主板和大量存储设备(如硬盘及光盘驱动器)之间的数据传输之用。

在数据传输上这一方面,SATA的速度比以往更加快捷,并支持热插拔。
另一方面,SATA总线使用了嵌入式时钟频率信号,具备了比以往更强的纠错能力,能对传输指令(不仅是数据)进行检查,提高了数据传输的可靠性。
不过,SATA和以往最明显的分别,是用上了较细的排线,有利机箱内部的空气流通,某程度上增加了整个平台的稳定性。
SATA与原来的IDE相比有很多优越性,最明显的就是数据线从80 pin变成了7 pin,
而且IDE线的长度不能超过0.4米,而SATA线可以长达1米,安装更方便,利于机箱散热。
每个设备都直接与主板相连,独享150M字节/秒带宽,设备间的速度不会互相影响。
热拔插对于普通家庭用户来说可能作用不大,但对于服务器却是至关重要。
事实上,SATA在低端服务器应用上取得的成功,远比在普通家庭应用中的影响力大。
SATA提高了错误检查的能力,除了对CRC对数据检错之外,还会对命令和状态包进行检错,
因此和并行ATA相比提高了接入的整体精确度,使串行ATA在企业RAID和外部存储应用中具有更大的吸引力。 


现时,SATA分别有SATA 1.5Gbit/s、SATA 3Gbit/s和SATA 6Gbit/s三种规格。
SATA 1.5Gb/s为第一代SATA接口,坊间的非官方名称为SATA-1。
SATA 3Gb/s在2004年正式推出,坊间的非官方名称为SATA-2(SATA-II),符合ATA-7规范,传输速度可达3.0Gbit/s。
在SATA2.0扩展规范所带来的一系列新功能中,NCQ(Native Command Queuing,原生命令队列)功能最令人关注。
SATA 6Gb/s 在2009年5月26日SATA-IO 完成 SATA 3.0 最终规格发布,比上一代提升一倍速率至6Gb/s,
此外增加多项新技术,包涵新增 NCQ 指令以改良传输技术,并减低传输时所需耗电量。
SATA不依赖于系统总线的带宽,而是内置时钟,每一代SATA升级带宽的增加都是成倍的,这点和PATA的一级级算术级数增长是不同的。
所谓3Gb/s的算法,3000MHz的频率 x 每次发送一个数据 x 80\%(8b/10b的编码) / 8 bits per byte = 300Mbytes/s,
同理1.5Gb/s也是这样可算成150MB/s,
也就是一般我们在买硬盘时,有时候会看到SATA 150MB/s / 300MB/s,有时候又会看到SATA 1.5Gb/s / 3Gb/s的缘故。
以USB 3.0而言,它拥有5Gbps的带宽,折算为500Mbytes/s,所以USB 3.0的带宽比SATA 3.0的600MB/s还来的小。

SATA的诸多先进性总体上对个人电脑用户意义不是太大,
它最大的意义的反而是适应了入门级企业应用的需要。 
但在企业级应用方面,它又仍然在很多方面有待改进:
单线程的机械底盘(不适应服务器应用程序大量非线性的读取请求。所以SATA硬盘用来做视频下载服务器还不错,用在网上交易平台则力不从心),
形同虚设的热拔插功能(SATA硬盘虽然可以热拔插,但SATA组成的阵列在某块硬盘损坏的时候,不能象SCSI、FC和SAS那样,具有SAF-TE机制用指示灯显示,知道具体坏的是哪一块)。
SATA 1.0控制器的传输速度效率不高,虽然标称具有150MB/s的峰值速度,事实上最快的SATA硬盘速度也只有60MB/s。
虽然SATA硬盘相对于SCSI硬盘来说很便宜,但整个的SATA方案并不便宜。
主要原因是SATA 1.0控制器的每个接口只能连接一个硬盘,8个硬盘组成的阵列需要8个接口,把每个接口300多元的花费算进去,就不便宜了。 

SATA国际组织(Serial ATA International Organization)正在着手制定下一代SATA标准,
定名为SATA Express,带宽最高可达8Gbps和16Gbps。 要达到最高的16Gbps带宽(现在最快的SATA 3.0标准带宽为6Gbps),
SATA Express标准将会如其名称所描述的,把SATA软件架构和PCI-Express高速界面结合在一起。
SATA国际组织称SATA Express标准将会带来新一代更快的存储装置和对应的主板接口,并且还能兼容现有的SATA设备。 S
ATA国际组织主席Mladen Luksic称该标准将使固态与混合硬盘受益于新一代PCI-Express 3.0的高带宽从而打破性能瓶颈,
标准的具体细节将在年内制定完成。 
SATA国际组织同时表示除SATA Express外,还有针对集成在主板上的嵌入式单芯片SSD存储解决方案的SATA µSSD标准,面向移动设备如平板电脑等。
SATA Express还处于标准制定阶段,可能会是SATA 3.2规范的一部分。
它其实就是PCI-E物理层上的SATA链接层,同时保持了对SATA 3/6Gbps等旧版规范的兼容(当然性能也会受影响),
传输带宽最初预计的范围是8-16Gbps,基本已经确定会达到10Gbps,实际传输速度也有1GB/s,比SATA 6Gbps要快将近70%。





\subsection{SAS}

SAS(Serial Attached SCSI)是并行SCSI接口之后开发出的全新接口。
此接口的设计是为了改善存储系统的效能、可用性和扩充性,
提供与串行ATA (Serial ATA,缩写为SATA)硬盘的兼容性。

SAS的接口技术可以向下兼容SATA。
SAS系统的背板(Backpanel)既可以连接具有双端口、高性能的SAS驱动器,也可以连接高容量、低成本的SATA驱动器。
因为SAS驱动器的端口与SATA驱动器的端口形状看上去类似,所以SAS驱动器和SATA驱动器可以同时存在于一个存储系统之中。
但需要注意的是,SATA系统并不兼容SAS,所以SAS驱动器不能连接到SATA背板上。
由于SAS系统的兼容性,IT人员能够运用不同接口的硬盘来满足各类应用在容量上或效能上的需求,
因此在扩充存储系统时拥有更多的弹性,让存储设备发挥最大的投资效益。


第一代SAS为数组中的每个驱动器提供 3.0 Gbps(300 MB/s)的传输速率(现在主流Ultra 320 SCSI速度为320MB/s)。
第二代SAS为数组中的每个驱动器提供 6.0 Gbps(600 MB/s)的传输速率。

SAS由3种类型协议组成,根据连接的不同设备使用相应的协议进行数据传输:
串行SCSI协议 (SSP) — 用于传输SCSI命令。
SATA通道协议 (STP) — 用于传输SATA数据。
SCSI管理协议 (SMP) — 用于对SAS设备的维护和管理。

SAS的接口技术可以向下兼容SATA。
具体来说,二者的兼容性主要体现在物理层和协议层的兼容。
在物理层,SAS接口和SATA接口完全兼容,SATA硬盘可以直接使用在SAS的环境中,
从接口标准上而言,SATA是SAS的一个子标准,因此SAS控制器可以直接操控SATA硬盘,
但是SAS却不能直接使用在SATA的环境中,因为SATA控制器并不能对SAS硬盘进行控制;
在协议层,SAS由3种类型协议组成,根据连接的不同设备使用相应的协议进行数据传输。其中串行SCSI协议(SSP)用于传输SCSI命令;SCSI管理协议(SMP)用于对连接设备的维护和管理;
SATA通道协议(STP)用于SAS和SATA之间数据的传输。
因此在这3种协议的配合下,SAS可以和SATA以及部分SCSI设备无缝结合。

存储设备的反应速度,除了各环节间的配合与操作系统的影响之外,硬盘的反应速度其实具有关键性的地位。
企业级的工作站或存储设备,一般来说,都采用光纤信道(Fibre Channel,FC)与SCSI硬盘作为内部的存储媒体。
但是随着SCSI硬盘在扩增性上的限制,SAS(Serial Attached SCSI)硬盘崭露头角。
服务器厂商有越来越多采用SAS硬盘作为内部的存储媒体,小型负载的应用可以采用SAS硬盘,可兼具预算与效能的考虑。


SAS目前的不足主要有以下方面:
1)硬盘、控制芯片种类少:只有希捷、迈拓以及富士通等为数不多的硬盘厂商推出了SAS接口硬盘.
2)硬盘价格太贵:比起同容量的Ultra 320 SCSI硬盘,SAS硬盘要贵了一倍还多。
3)实际传输速度变化不大:SAS硬盘的接口速度并不代表数据传输速度,受到硬盘机械结构限制,现在SAS硬盘的机械结构和SCSI硬盘几乎一样。
目前数据传输的瓶颈集中在由硬盘内部机械机构和硬盘存储技术、磁盘转速所决定的硬盘内部数据传输速度,也就是80MBsec左右,SAS硬盘的性能提升不明显。
4)用户追求成熟、稳定的产品. 虽然SAS接口服务器和SCSI接口产品在速度、稳定性上差不多,但目前的技术和产品都还不够成熟。
不过随着英特尔等主板芯片组制造商、希捷等硬盘制造商以及众多的服务器制造商的大力推动,SAS的相关产品技术会逐步成熟,价格也会逐步滑落,早晚都会成为服务器硬盘的主流接口。

衡量一种技术的优劣通常有4个基本指标,即性能、可靠性、可扩展性和成本。
回顾串行磁盘技术的发展历史,从光纤通道,到SATA,再到SAS,几种技术各有所长。
光纤通道最早出现的串行化存储技术,可以满足高性能、高可靠和高扩展性的存储需要,但是价格居高不下;
SATA硬盘成本倒是降下来了,但主要是用于近线存储和非关键性应用,毕竟在性能等方面差强人意;
SAS应该算是个全才,可以支持SAS和SATA磁盘,很方便地满足不同性价比的存储需求,是具有高性能、高可靠和高扩展性的解决方案。










