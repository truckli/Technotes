%!Mode:: "TeX:UTF-8"
\section{GDB调试器使用要略}


如果想对elfname程序进行调试,用-g选项编译,调用gdb的方式如:
\verb+gdb elfname+或\verb+gdb --args elfname arg1 arg2+\ldots
也可以只输入gdb,在交互界面上设置:
\begin{verbatim}
  file elfname
  set args argv1 argv2
\end{verbatim}


基本的用法,可以在进入gdb后执行help,具体的帮助信息,如关于断点的用法,输入help b。
退出gdb:quit或q。

每次执行\textbf{run},会从头开始运行程序。run简称\textbf{r}。

\subsection{查看源码}
\textbf{list简称l}。list 显示当前位置10行代码。

l funcname:显示某函数附近的10行代码。

l lineno: 显示第lineno行及其上下文10行代码。


\subsection{断点}
首先需要明确,如果在第18行设置断点,指的是在第18行执行之前中止,而非之后。

设置断点:breakpoint命令,简称\textbf{b},参数为[文件名:]行号或函数名,可以用list命令辅助b命令的参数选择。
如无参数,为在当前行设置断点。

b 45 if varname > 10: 在第45行之前设置条件断点,如果变量varname大于10则中止。

delete 2或del 2或d 2会删除2号断点, disable 2会冻结2号断点,enable 2会激活2号断点。

i b: 查看当前所有断点的状态。

遇到断点时执行cont或\textbf{c}时程序继续执行直至结束或受阻。

\subsection{步进}
next命令简称\textbf{n},用于执行下一条语句。执行完后显示的代码行为尚未执行的下一条语句。

n 5可以前进5行。

Return键用于重复执行上一条命令。

step命令简称\textbf{s},与next的区别是会进入函数体。

\subsection{查看上下文信息}

backtrace, 简称\textbf{bt},查看堆栈层次信息。最深层的frame号为零,main为最大的frame号。

frame简称\textbf{f}。\textbf{f frameno}会跳到frameno指定的frame层次。

info简称\textbf{i}:
\begin{description}
    \item[i line]  显示当前行号。
    \item[i args]  显示当前函数参数。在main函数中则为程序参数。
    \item[i source] 查看当前源文件信息。info sources查看所有源文件信息。
    \item[i locals]  显示当前函数局部变量。
    \item[i variables]  显示全局和静态变量。
    \item[i threads]  显示线程信息。thread ID命令跳转到ID所示的线程上。
    \item[i macros]  显示宏定义。i macro macroname某个宏定义。必须使用 -gdwarf-2 -g3编译选项才能看到宏定义。
\end{description}



\subsection{查看指定变量}
print varname 查看变量。print简称\textbf{p}。

例如,对于char c[5] = {97,98,99,100,101},
执行p c[2]显示 99 'c';p /x c[2]会显示为16进制0x63。
print /x c[2]@3会显示c[2]开始的3个连续变量{0x63, 0x64, 0x65}。

display varname 添加自动显示变量。
display添加的变量会在每次程序中止时再次显示。

\subsection{Patching:就地修补}
set variable varname = varvalue命令用于就地修改变量的值。
那么,下面的命令在断点处修改变量的值,下次run时生效。
\begin{verbatim}
commands 2
>set variable n = n+1
>cont
>end
\end{verbatim}


\subsection{观察值}
watch varname命令:当程序执行时发现varname被修改时就中止;
rwatch varname:当varname被读时中止;而awatch是在varname被读或者写时都中止。

注意watch只能观察当前堆栈frame的变量。所以一般要配合断点来使用。







































