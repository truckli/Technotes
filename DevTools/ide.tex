%!Mode:: "TeX:UTF-8"

\section{IDE选用}



Python IDE包括PyCharm,WingIDE,PyDev(Eclipse),Komodo IDE,Eric等。有人尤其推荐PyCharm,WingIDE和PyDev。

Eric是基于Qt的IDE,在Debian系Linux发行版上依赖于包python-qscintilla2。

IDLE常常是python安装是自带的开发工具。

在Linux上选择C/C++的IDE,有CodeBlocks,Eclipse CDT, Greany,MonoDevelop,Anjuta,Komodo Edit,NetBeans,KDevelop, CodeLite等。

源代码阅读器包括:Visual Studio Express edition(免费),Source Navigator,Understand等。

我有跨平台工作需要,决定使用Vim,QtCreator和Eclipse。如只在Windows上阅读代码,使用Source Insight。Vim的优势在于极强的定制型,以为IDE的定制性毕竟有限,有些设计可能让人觉得别扭。


\subsection{用QtCreator编辑Python脚本}
QtCreator可以编辑Python,进行语法高亮,但缺乏自动补全等高级功能。也可通过配置“外部工具(工具-选项-环境)”来调用系统中安装的python解释器以运行Python脚本。
配置方式如下:
1. 添加Category并在该Category上添加Tool
2. 配置这个Tool。设置Python解释器路径、参数(\verb|%{CurrentDocument:FilePath}|)、工作目录(\verb|%{CurrentDocument:Path}|)、环境变量(\verb|QT_LOGGING_TO_CONSOLE=1|)。

\subsection{QtCreator和Source Insight比较}
使用QtCreator可以开启vi模式,对于熟悉vim的人很顺手。点击一个标识符号,自动对该标识符的示例加以高亮。

QtCreator的缺点之一在于Find Usages这个Context功能的局限性,它比不上SI的Jump to Caller命令:似乎只能检索已经打开的文件,而不是项目中的全部文件(这可能只是个bug 而非设计);无法区分不同类型的同名成员变量。有时为了查找符合所有的occurrence,须使用字符串查找功能。这毕竟不够高效。


Si不如QtCreator之处一是不免费,二是浏览代码目录不方便,三是匹配括号不方便(快捷键\verb$Ctrl-Shift-]$),四是没有代码折叠。


\subsection{条件编译编辑}
某些代码因为条件编译等原因变成了inactive code。
例如,如果程序中没有定义名为NEVER的宏,则QtCreator认为\verb$#ifdef NEVER$ 和 \verb$ #endif $ 之间的代码为不活跃代码,将其背景设置为灰色,避免对读者的干扰。
然而SI确不作这种假定,而是允许用户手工设置这段代码是否为不活跃代码。
两种工具的设计各有有缺。
因为宏定义不仅仅仅是通过源代码给出,也可在编译器选项中给出,QtCreator做出的假定不妥。
解决办法是,如果需要,可主动修改代码,定义NEVER这个宏,然后保存本文件。




\subsection{Vim搭建Python IDE}
Vim插件pythonComplete可用于Python代码自动补全。对于VIM 7.3以上版本,这个插件是自带的。需要在.vimrc上添加如下配置:
\begin{verbatim}
filetype plugin on
autocmd FileType python set omnifunc=pythoncomplete#Complete
\end{verbatim}

\subsection{非vi编辑器的vi风格模式}

Visual Studio with ViEmu
NetBeans with jVi

Sublime Text has a vintage mode for vi style editing.

Check out excellent Vrapper plugin for Eclipse.

It seems the eclim plugin can help you embed the real GVim into Eclipse.

Qt Creator has a "vim mode" for editing, but it currently lacks some abilities; as well, I feel handicapped without the settings I have in my .vimrc.

There is also freeware Vimplugin for Eclipse — it embeds Vim into Eclipse, but you lose all navigation and code-completion functionality that Eclipse provides, so its usefulness is disputable.




















