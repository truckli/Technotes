%!Mode:: "TeX:UTF-8"

\section{Valgrind}

Valgrind是运行在Linux上一套基于仿真技术的程序调试和分析工具,它包含一个内核──一个软件合成的CPU,和一系列的小工具,每个工具都可以完成一项任务──调试,分析,或测试等。Valgrind可以检测内存泄漏和内存违例,还可以分析cache的使用等,灵活轻巧而又强大,能直穿程序错误的心脏,真可谓是程序员的瑞士军刀。

Valgrind包含如下工具:
\begin{description}
  \item[Memcheck]最常用的工具,用来检测程序中出现的内存问题:
    \begin{enumerate}
        \item 对未初始化内存的使用; 
        \item 读/写释放后的内存块; 
        \item 读/写超出malloc分配的内存块; 
        \item 读/写不适当的栈中内存块; 
        \item 内存泄漏,指向一块内存的指针永远丢失; 
        \item 不正确的malloc/free或new/delete匹配; 
        \item memcpy()相关函数中的dst和src指针重叠。
     \end{enumerate} 
    \item[Callgrind]和gprof类似的分析工具,但它对程序的运行观察更是入微。Callgrind收集程序运行时的一些数据,建立函数调用关系图,还可以有选择地进行cache模拟。
    \item[Cachegrind]Cache分析器,它模拟CPU中的一级缓存I1,Dl和二级缓存,能够精确地指出程序中cache的丢失和命中。如果需要,它还能够为我们提供cache丢失次数,内存引用次数,以及每行代码,每个函数,每个模块,整个程序产生的指令数。这对优化程序有很大的帮助。
    \item[Helgrind]仍处于实验阶段。它主要用来检查多线程程序中出现的竞争问题。Helgrind寻找内存中被多个线程访问,而又没有一贯加锁的区域,这些区域往往是线程之间失去同步的地方,而且会导致难以发掘的错误。
    \item[Massif]堆栈分析器,它能测量程序在堆栈中使用了多少内存,告诉我们堆块,堆管理块和栈的大小。
\end{description}

Valgrind检测的程序需用-g选项编译。gcc的-g选项能够生成额外调试信息。

Memcheck用法:
\begin{verbatim}
   valgrind --leak-check=full ./appname args...
   或 valgrind --tool=memcheck ./appname args...
\end{verbatim}


Callgrind会输出很多,而且最后在当前目录下生成一个文件: callgrind.out.pid。用callgrind\_annotate来查看它:
\begin{verbatim}
   valgrind --tool=callgrind ./appname args...
   callgrind_annotate callgrind.out.pid
\end{verbatim}
















