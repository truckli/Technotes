%!Mode:: "TeX:UTF-8"

\section{文件分析工具}

\subsection{统计代码行数}
例如,统计.h文件包含的非空行数:
\begin{verbatim}
  find . -name "*.h"|xargs cat|grep -v ^$|wc -l 
\end{verbatim}
如果不需要空行过滤功能,就不需要使用cat和grep,可用这样写:
\begin{verbatim}
   wc -l `find . -name "*.h"`|tail -n1
\end{verbatim}
匹配多种文件类型:
\begin{verbatim}
find . -name "*.c" -o -name "*.h"|xargs wc -l
\end{verbatim}
这条命令统计了.h和.c文件中包含的行数。注意如果没有xargs,就变成了统计文件数目。

\subsection{文本文件分析}
diff用于逐行比较。其他工具包括awk,sed等。

对于源代码,有indent工具。


\subsection{十六进制读写}

od, hexdump等工具将文件按照8进制、16进制、ASCII等形式打印出来。hexdiff同时打开两个文件,进行比较。

hexer工具能够以vi风格的界面编辑数据文件,而ghex则基于Gnome窗口系统。其他工具大多基于curses家族,包括hexedit, lfhex, dhex等。

vim使用\verb$:%!xxd$变为十六进制显示模式,再追加一个\verb$-r$命令则复原。

\subsection{目标文件分析}

strings:寻找文件中的字符串。可用于分析可执行文件。

size:显示目标文件中各段的大小(text,data,bss)。

readelf显示ELF文件各种信息。

nm或objdump -t:打印目标文件中的符号表。

objdump -h 给出文件中各段的头部信息。

objdump -D反汇编,-S将反汇编程序同源代码对照显示。









