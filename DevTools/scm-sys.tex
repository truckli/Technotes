%!Mode:: "TeX:UTF-8"

\section{Build and SCM systems}

\href{https://about.gitlab.com/}{GitLab}, the software, is a web-based Git repository manager with wiki and issue tracking features.
 
\href{https://jenkins-ci.org/}{Jenkins} is an open source continuous integration tool written in Java.
 It is a server-based system running in a servlet container such as Apache Tomcat. 
 It supports SCM tools including AccuRev, CVS, Subversion, Git, Mercurial, Perforce, Clearcase and RTC, 
 and can execute Apache Ant and Apache Maven based projects as well as arbitrary shell scripts and Windows batch commands.

\href{http://ant.apache.org/}{Apache Ant} is a software tool for automating
software build processes, which originated from the Apache Tomcat project in early 2000.
 It was a replacement for the unix make build tool, and was created due to a number of problems with the unix make.
It is similar to Make but is implemented using the Java language, requires the Java platform, and is best suited to building Java projects.
 The main known usage of Ant is the build of Java applications. 
 Ant supplies a number of built-in tasks allowing to compile, assemble, test and run Java applications. 
Ant can also be used effectively to build non Java applications, for instance C or C++ applications.

\href{https://maven.apache.org/}{Maven} is a build automation tool used primarily for Java projects.
Maven addresses two aspects of building software: First, it describes how
software is built, and second, it describes its dependencies. 
Contrary to preceding tools like Apache Ant, it uses conventions for the build procedure, and only exceptions need to be written down. 
An XML file describes the software project being built, its dependencies on other external modules and components, the build order, directories, and required plug-ins.


\subsection{Qmake,CMake}
qmake是一个协助简化跨平台进行项目开发的构建过程的工具程序,Qt附带的工具之一 。qmake能够自动生成Makefile、Microsoft Visual Studio 项目文件 和 xcode 项目文件。不管源代码是否是用Qt写的,都能使用qmake,因此qmake能用于很多软件的构建过程。

手写Makefile是比较困难而且容易出错,尤其在进行跨平台开发时必须针对不同平台分别撰写Makefile,会增加跨平台开发复杂性与困难度。qmake会根据项目文件(.pro)里面的信息自动生成适合平台的 Makefile。开发者能够自行撰写项目文件或是由qmake本身产生。qmake包含额外的功能来方便 Qt 开发,如自动的包含moc 和 uic 的编译规则。

CMake是个开源的跨平台自动化建构系统,它用配置文件控制建构过程(build process)的方式和Unix的Make相似,只是CMake的配置文件取名为CmakeLists.txt。Cmake并不直接建构出最终的软件,而是产生标准的建构文件(如Unix的Makefile或Windows Visual C++的projects/workspaces),然后再依一般的建构方式使用。这使得熟悉某个集成开发环境(IDE)的开发者可以用标准的方式建构他的软件,这种可以使用各平台的原生建构系统的能力是CMake和SCons等其他类似系统的区别之处。CMake可以编译源代码、制做程序库、产生适配器(wrapper)、还可以用任意的顺序建构可执行文件。CMake支持in-place建构(二进文件和源代码在同一个目录树中)和out-of-place建构(二进文件在别的目录里),因此可以很容易从同一个源代码目录树中建构出多个二进文件。CMake也支持静态与动态程序库的建构。

“CMake”这个名字是"cross platform make"的缩写。虽然名字中含有"make",但是CMake和Unix上常见的“make”系统是分开的,而且更为高级。


%!Mode:: "TeX:UTF-8"

\subsection{Scons}

\begin{verbatim}
  Program('program', ['prog.c', 'file1.c', 'file2.c'])
  Program('program', 
 	 ['hello.cpp', 'bloom_filter.c'], 
 	 CPPPATH = ['/home/truckli/Coding/wheels/Boost/boost_1_59_0'],
 	 LIBS='boost_system', 
 	 LIBPATH=['/home/truckli/Coding/wheels/Boost/boost_1_59_0/stage/lib'])

\end{verbatim}

\begin{verbatim}
env = Environment(CPPPATH = ['/home/truckli/Coding/wheels/Boost/boost_1_59_0'])
env.Program('program', ['hello.cpp', 'bloom_filter.c'], LIBS='boost_system', LIBPATH=['/home/truckli/Coding/wheels/Boost/boost_1_59_0/stage/lib'])
\end{verbatim}










